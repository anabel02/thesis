\begin{opinion}
    La programación es vital para el desarrollo de la computación, para el mundo moderno y para los estudiantes de nuestras carreras.  Por otra parte, la enseñanza de la programación ha recibido muchísima atención de parte de los investigadores de todo el mundo, como se puede apreciar en el capítulo de los preliminares de esta tesis.  

    En este trabajo se propone una metodología para asistir a los estudiantes de MATCOM con el estudio independiente de esta habilidad.  También se propone una herramienta computacional que implementa la propuesta realizada.   Este trabajo está dirigido específicamente a los estudiantes de nuestras carreras, y toma en consideración las características del curso que se imparte, y la bibliografía básica del mismo.

    Para realizar este trabajo, Anabel tuvo que estudiar contenidos que no forman parte de su plan de estudio, y lo hizo de una manera excelente.  

    Creo que estamos en presencia de un trabajo excelente desarrollado por una excelente científica de la computación y una excelente persona.

    \vspace{1cm}


    \begin{flushright}
    \underline{\hspace{6.5cm}}\\
    MSc. Fernando Raul Rodriguez Flores

    Facultad de Matemática y Computación
    
    Universidad de la Habana

    Febrero, 2025
    \end{flushright}
\end{opinion}