\documentclass[12pt,oneside]{uhthesis}
\usepackage{subfigure}
\usepackage[ruled,lined,linesnumbered,titlenumbered,algochapter,spanish,onelanguage]{algorithm2e}
\usepackage{amsmath}
\usepackage{amssymb}
\usepackage{amsbsy}
\usepackage{caption,booktabs}
\captionsetup{ justification = centering }
%\usepackage{mathpazo}
\usepackage{float}
\setlength{\marginparwidth}{2cm}
\usepackage{todonotes}
\usepackage{listings}
\usepackage{xcolor}
\usepackage{multicol}
\usepackage{graphicx}
\usepackage{framed}
\usepackage{tikz}
\usepackage[utf8]{inputenc}
\usepackage[spanish]{cleveref}
\usetikzlibrary{arrows.meta, positioning}
\usetikzlibrary{3d, shapes.geometric}
\crefname{section}{sección}{secciones}

\tikzset{
    startstop/.style = {rectangle, rounded corners, minimum width=2cm, minimum height=0.8cm, text centered, draw=black, fill=yellow!10, align=center, font=\footnotesize},
    process/.style = {rectangle, minimum width=2cm, minimum height=0.8cm, text centered, draw=black, fill=yellow!10, align=center, font=\footnotesize},
    decision/.style = {diamond, minimum width=2cm, minimum height=0.8cm, text centered, draw=black, fill=yellow!10, align=center, font=\footnotesize},
    arrow2/.style = {thick,->,>=stealth, font=\footnotesize}
}
\floatstyle{plaintop}
\restylefloat{table}
% \usepackage[backend=biber, style=apa]{biblatex} % Opciones correctas
\addbibresource{Bibliography.bib} % Archivo .bib
% \setlength{\parskip}{\baselineskip}%
\renewcommand{\tablename}{Tabla}
\renewcommand{\listalgorithmcfname}{Índice de Algoritmos}
%\dontprintsemicolon
\SetAlgoNoEnd

\definecolor{codegreen}{rgb}{0,0.6,0}
\definecolor{codegray}{rgb}{0.5,0.5,0.5}
\definecolor{codepurple}{rgb}{0.58,0,0.82}
\definecolor{backcolour}{rgb}{0.95,0.95,0.92}

\lstdefinestyle{mystyle}{
    backgroundcolor=\color{backcolour},   
    commentstyle=\color{codegreen},
    keywordstyle=\color{purple},
    numberstyle=\tiny\color{codegray},
    stringstyle=\color{codepurple},
    basicstyle=\ttfamily\footnotesize,
    breakatwhitespace=false,         
    breaklines=true,                 
    captionpos=b,                    
    keepspaces=true,                 
    % numbers=left,                    
    % numbersep=5pt,                  
    showspaces=false,                
    showstringspaces=false,
    showtabs=false,                  
    tabsize=4,
    inputencoding=utf8,
    literate=%
    {á}{{\'a}}1
    {é}{{\'e}}1
    {í}{{\'i}}1
    {ó}{{\'o}}1
    {ú}{{\'u}}1
    {Á}{{\'A}}1
    {É}{{\'E}}1
    {Í}{{\'I}}1
    {Ó}{{\'O}}1
    {Ú}{{\'U}}1
    {ñ}{{\~n}}1
    {Ñ}{{\~N}}1
    {¡}{{\textexclamdown}}1
    {¿}{{\textquestiondown}}1
}

\lstset{style=mystyle}

\title{Sistema para apoyar el estudio independiente en Programación}
\author{\\\vspace{0.25cm}Anabel Benítez González}
\advisor{\\\vspace{0.25cm}MSc. Fernando Rodríguez Flores}
\degree{Licenciado en Ciencia de la Computación}
\faculty{Facultad de Matemática y Computación}
\date{10 de febrero de 2025\\\vspace{0.25cm}\href{https://github.com/anabel02/programming-tutor}{github.com/anabel02/programming-tutor}}
\logo{Graphics/uhlogo}
\makenomenclature

\renewcommand{\vec}[1]{\boldsymbol{#1}}
\newcommand{\diff}[1]{\ensuremath{\mathrm{d}#1}}
\newcommand{\me}[1]{\mathrm{e}^{#1}}
\newcommand{\pf}{\mathfrak{p}}
\newcommand{\qf}{\mathfrak{q}}
%\newcommand{\kf}{\mathfrak{k}}
\newcommand{\kt}{\mathtt{k}}
\newcommand{\mf}{\mathfrak{m}}
\newcommand{\hf}{\mathfrak{h}}
\newcommand{\fac}{\mathrm{fac}}
\newcommand{\maxx}[1]{\max\left\{ #1 \right\} }
\newcommand{\minn}[1]{\min\left\{ #1 \right\} }
\newcommand{\lldpcf}{1.25}
\newcommand{\nnorm}[1]{\left\lvert #1 \right\rvert }
\renewcommand{\lstlistingname}{Ejemplo de código}
\renewcommand{\lstlistlistingname}{Ejemplos de código}

\begin{document}

\frontmatter
\maketitle

\begin{dedication}
    A mi mamá.
\end{dedication}
\begin{acknowledgements}
    A mis padres, por darme todo. En especial a mi mamá, por su apoyo incondicional, por estar siempre ahí para escucharme y entenderme. 
    
    A mi hermana por ser cómplice en mis boberías, a mi hermano, mis primos, mis tíos, mis abuelos, por su cariño.

    A Amanda, por estar ahí en todo momento, por todas las risas y llantos compartidos, especialmente en primer año, y por soñar juntas con este día desde el principio.  

    A Omar, por su amor e infinita paciencia, por ser el mejor compañero de viaje. 
    
    A Liuvca y Badiola, por abrirme las puertas de su hogar y recibirme siempre con tanto cariño.

    A Alex, Raudel y Ciclanejo, por haber estado desde primer año, compartiendo este camino. A Juanky, por todos los chistes malos que siempre me sacan una risa. A Enzo, por hacer que los proyectos fueran menos estresantes.

    A Javier, Daniel y Loraine por su guía en estos años.

    A mi tutor, Fernando, por su paciencia y dedicación, por hacer posible esta tesis y, junto a Cartaya, animarme a salir de mi zona de comfort y descubrir el placer de dar clases.

    Al C121, mis estudiantes, por todo lo que aprendí de ellos en estos meses.

    Gracias a todos los que, de una forma u otra, formaron parte de este camino.
\end{acknowledgements}
\begin{opinion}
    La programación es vital para el desarrollo de la computación, para el mundo moderno y para los estudiantes de nuestras carreras.  Por otra parte, la enseñanza de la programación ha recibido muchísima atención de parte de los investigadores de todo el mundo, como se puede apreciar en el capítulo de los preliminares de esta tesis.  

    En este trabajo se propone una metodología para asistir a los estudiantes de MATCOM con el estudio independiente de esta habilidad.  También se propone una herramienta computacional que implementa la propuesta realizada.   Este trabajo está dirigido específicamente a los estudiantes de nuestras carreras, y toma en consideración las características del curso que se imparte, y la bibliografía básica del mismo.

    Para realizar este trabajo, Anabel tuvo que estudiar contenidos que no forman parte de su plan de estudio, y lo hizo de una manera excelente.  

    Creo que estamos en presencia de un trabajo excelente desarrollado por una excelente científica de la computación y una excelente persona.

    \vspace{1cm}


    \begin{flushright}
    \underline{\hspace{6.5cm}}\\
    MSc. Fernando Raul Rodriguez Flores

    Facultad de Matemática y Computación
    
    Universidad de la Habana

    Febrero, 2025
    \end{flushright}
\end{opinion}
\begin{resumen}
	La programación es una habilidad esencial en el mundo moderno. En el ámbito académico, los cursos introductorios de programación son fundamentales para la formación de profesionales en Ciencia de la Computación, ya que proporcionan las bases para el desarrollo de habilidades avanzadas. Sin embargo, el estudio independiente en estos cursos enfrenta desafíos, como la dificultad de los estudiantes para gestionar el tiempo, la acumulación de conceptos no comprendidos y la falta de orientación sobre por dónde comenzar, lo que puede generar frustración y desmotivación.

	Este trabajo propone el diseño e implementación de un sistema de apoyo al estudio independiente para la asignatura de Programación en la carrera de Ciencia de la Computación de la Facultad de Matemática y Computación de la Universidad de La Habana (\mbox{MATCOM}). El sistema aborda problemas identificados, como la dependencia excesiva del apoyo docente y la falta de retroalimentación oportuna. Incluye funcionalidades como un sistema de preguntas y respuestas basado en Generación Aumentada con Recuperación (RAG), que permite a los estudiantes resolver dudas en tiempo real, y un mecanismo de propuesta de ejercicios adaptados al nivel de conocimiento del estudiante. Además, ofrece pistas durante la resolución de problemas y soluciones detalladas una vez completados, con el objetivo de fomentar el pensamiento crítico y la autonomía.

	El sistema se implementa como un bot en Telegram, una plataforma ampliamente utilizada por los estudiantes de \mbox{MATCOM}. Esta herramienta busca mejorar el rendimiento académico y promover habilidades como la autogestión del aprendizaje y la resolución de problemas. Este trabajo ofrece una solución adaptada a las necesidades específicas de los estudiantes de programación, con potencial para ser replicada en otros contextos académicos.
\end{resumen}

\begin{abstract}
	Programming is an essential skill in the modern world. Introductory programming courses are fundamental for the training of professionals in Computer Science, as they provide the foundation for developing advanced skills. However, independent study in these courses faces challenges, such as students' difficulty in managing time, the accumulation of misunderstood concepts, and the lack of guidance on where to begin, which can lead to frustration and demotivation.

	This work proposes the design and implementation of a system to support independent study for the Programming course in the Computer Science program at the Faculty of Mathematics and Computing of the University of Havana (\mbox{MATCOM}). The system addresses identified issues, such as excessive reliance on instructor support and the lack of timely feedback. It includes functionalities such as a question-and-answer system based on Retrieval-Augmented Generation (RAG), which allows students to resolve doubts in real time, and a mechanism for proposing exercises tailored to the student's knowledge level. Additionally, it provides hints during problem-solving and detailed solutions upon completion, aiming to foster critical thinking and autonomy.

	The system is implemented as a bot on Telegram, a platform widely used by \mbox{MATCOM} students. This tool seeks to improve academic performance and promote skills such as self-regulated learning and problem-solving. This work offers a solution tailored to the specific needs of programming students, with the potential to be replicated in other academic contexts.
\end{abstract}
\include{FrontMatter/Contents}

\mainmatter

\chapter*{Introducción}\label{chapter:introduction}
\addcontentsline{toc}{chapter}{Introducción}

La programación se ha convertido en una habilidad fundamental en el mundo contemporáneo, impulsando avances tecnológicos y transformando industrias enteras. Su importancia radica en su capacidad para resolver problemas complejos, optimizar procesos y fomentar la innovación en áreas como la inteligencia artificial, la ciencia de datos y el desarrollo de software~\cite{wing2006computational}. Además, la programación no solo es esencial para los profesionales de la tecnología, sino que también se ha integrado en disciplinas diversas, desde la medicina hasta las ciencias sociales, ampliando su impacto en la sociedad~\cite{resnick2009scratch}. En este contexto, comprender y dominar los principios de la programación no solo es valioso, sino necesario para enfrentar los desafíos del siglo XXI y contribuir al progreso global.

En el ámbito académico, la programación es un componente fundamental en la formación de profesionales en Ciencia de la Computación. Los cursos introductorios de programación son el primer contacto de los estudiantes con los conceptos y herramientas básicas, por lo que deben proporcionar una base sólida para el desarrollo de habilidades más avanzadas en este campo~\cite{Sarsa_2022}.

El estudio independiente en cursos introductorios de programación enfrenta desafíos que pueden afectar el rendimiento académico de los estudiantes. Según~\cite{proskuraLytvynova2020}, la falta de supervisión constante por parte del docente puede llevar a confusiones en la priorización de contenidos y en la gestión del tiempo, especialmente cuando los estudiantes deben equilibrar el aprendizaje de la programación con otras responsabilidades académicas. Además, en~\cite{overklift2019} se señala que la ausencia de un docente disponible para resolver dudas durante el estudio puede resultar en la acumulación de conceptos no comprendidos, generando frustración y desmotivación.

Varios estudios han abordado problemas relacionados con la enseñanza y el aprendizaje de la programación. Por ejemplo, en~\cite{Gabbay2022, Hanafi2023, Messer2024} se estudia el uso de plataformas de autoevaluación que permiten a los estudiantes recibir retroalimentación inmediata sobre sus códigos, lo que ha demostrado ser efectivo para el rendimiento del estudiante.

Por otro lado, en~\cite{dong2025buildaitutoradapt} se sugiere la implementación de sistemas de tutoría virtual que integren \textit{chatbots} inteligentes para resolver dudas en tiempo real. Estos sistemas, basados en modelos de lenguaje de gran tamaño (LLMs), como \textit{ChatGPT}~\cite{chatgpt}, proporcionan respuestas adaptándose a las necesidades individuales de los estudiantes.

En la asignatura Programación de la carrera Ciencia de la Computación en la Facultad de Matemática y Computación de la Universidad de La Habana (\mbox{MATCOM}), los estudiantes enfrentan desafíos en el proceso de aprendizaje. Los profesores han observado que existe una dependencia significativa del apoyo docente para que los estudiantes logren avanzar en la disciplina. Esta situación limita el desarrollo de habilidades esenciales como la resolución de problemas, la autogestión del aprendizaje y la capacidad de trabajar de manera independiente, competencias clave en el ámbito de la computación.

A partir de lo expuesto anteriormente, el objetivo general de este trabajo es diseñar e implementar un sistema que apoye a los estudiantes de Ciencia de la Computación de \mbox{MATCOM} en el estudio independiente de la asignatura Programación.

Los objetivos específicos de esta investigación son:
\begin{itemize}
    \item Realizar una revisión de la bibliografía existente sobre sistemas de apoyo al aprendizaje de programación y metodologías de estudio independiente.
    \item Identificar las necesidades específicas de los estudiantes de \mbox{MATCOM} en relación con el estudio independiente de programación.
    \item Recopilar ejercicios de Programación organizados por temas y elaborar soluciones detalladas que expliquen los procedimientos a los estudiantes.
    \item Diseñar un sistema que incluya funcionalidades como proporcionar retroalimentación inmediata sobre los ejercicios, resolver dudas comunes de los estudiantes y guiar en la resolución de problemas.
    \item Implementar un prototipo del sistema a través de un bot de Telegram.
    \item Proponer recomendaciones para mejorar el sistema.
\end{itemize}

Este trabajo propone un sistema diseñado específicamente para apoyar el estudio independiente de la asignatura Programación en la carrera Ciencia de la Computación en \mbox{MATCOM}. Este sistema se enfoca en integrar recursos adaptados al plan de estudios de la facultad con mecanismos de seguimiento del progreso del estudiante. También se abordan problemas identificados en el proceso de estudio independiente, como la falta de retroalimentación oportuna y la dependencia excesiva del apoyo docente. El sistema busca no solo facilitar el aprendizaje de la programación, sino también motivar a los estudiantes a través de una herramienta que les permita avanzar a su propio ritmo y recibir apoyo constante.

Para resolver los problemas identificados, el sistema propuesto integra un sistema de preguntas y respuestas basado en Generación Aumentada con Recuperación (RAG), que permite a los estudiantes acceder a respuestas a sus dudas sobre el contenido. Además, cuenta con un mecanismo de propuesta de ejercicios que selecciona problemas acordes al nivel de conocimiento del estudiante sobre un tema específico. Durante la resolución de los ejercicios, los estudiantes pueden solicitar pistas que los orienten sin revelar la solución completa, fomentando así el pensamiento crítico y la autonomía. Una vez resuelto el ejercicio, el sistema proporciona retroalimentación y acceso a la solución, lo que permite a los estudiantes comprender sus errores y aprender de ellos. Al ofrecer esta guía, el sistema busca reducir la dependencia del apoyo docente constante, promoviendo la autoevaluación y el aprendizaje autónomo, habilidades esenciales para el desarrollo académico y profesional de los estudiantes.

Además, dado que la mayoría de los estudiantes de la facultad están familiarizados con el uso de Telegram~\cite{Telegram}, se realiza una implementación de este sistema como un bot en dicha plataforma.

La estructura de este documento es la siguiente: en el capítulo~\ref{chapter:background} se presentan algunos de los fundamentos de la enseñanza de programación, se analiza la importancia del estudio independiente en el aprendizaje de la programación y se describe el curso de Programación que se imparte en primer año de Ciencia de la Computación en \mbox{MATCOM}. En el capítulo~\ref{chapter:rag} se explica el funcionamiento del sistema de Generación Aumentada con Recuperación que se usa en la implementación de la herramienta. En el capítulo~\ref{chapter:exercises} se describe el proceso de recopilación y diseño de un conjunto de ejercicios, así como el enfoque adoptado para presentar sus soluciones y pistas que guíen al estudiante durante el proceso de solución. En el capítulo~\ref{chapter:proposal} se presenta una propuesta del sistema detallando el flujo de cada funcionalidad. En el capítulo~\ref{chapter:implementation} se exponen los detalles de implementación del bot de Telegram. Finalmente se presentan las conclusiones, recomendaciones y trabajo futuro.
\include{MainMatter/Background}
\chapter{Generación Aumentada con Recuperación}\label{chapter:rag}

Los modelos de lenguaje de gran tamaño (LLMs) han demostrado un éxito notable en diversas tareas relacionadas con el procesamiento del lenguaje natural~\cite{makridakis2023largelanguagemodels}. Estos modelos están entrenados con grandes volúmenes de datos y utilizan miles de millones de parámetros para generar contenido en tareas como responder preguntas, traducir idiomas y completar oraciones; sin embargo, aún enfrentan limitaciones. Un problema destacado es la generación de ``alucinaciones''~\cite{zhang2023sirenssongaiocean}.

La Generación Aumentada con Recuperación (RAG, por sus siglas en inglés) es un enfoque que combina LLMs con sistemas de recuperación de información para optimizar las respuestas generadas al incorporar datos de una base de conocimiento autorizada~\cite{gao2024retrievalaugmentedgenerationlargelanguage}. RAG se utiliza para extender las capacidades de los LLMs a dominios específicos o bases de conocimiento internas de una organización sin requerir un reentrenamiento del modelo~\cite{gao2024retrievalaugmentedgenerationlargelanguage}.

El flujo de trabajo de RAG tiene tres pasos: recuperación, aumento de la consulta y generación de la respuesta. A continuación se describe cada uno.
\begin{enumerate}
     \item Recuperación: En este primer paso, se realiza una búsqueda a partir de una consulta del usuario para seleccionar la información más relevante de una base de datos o fuente de conocimiento.
    \item Aumento de la consulta: La información recopilada se integra con la consulta original del usuario, lo que genera un contexto para la siguiente fase del proceso.
    \item Generación de la respuesta: Finalmente, se suministra a un LLM la consulta ampliada con el contexto, con el objetivo de producir una respuesta más precisa y relevante que la que se habría obtenido utilizando únicamente la consulta inicial del usuario.
\end{enumerate}

En las siguientes secciones, se brindan detalles de cada uno de estos pasos y se describe el proceso de construcción del componente de recuperación. 

\section{Construcción del componente de recuperación}

El proceso de construcción del componente de recuperación consta de tres etapas: dividir el corpus en fragmentos, codificar los fragmentos y construir la base de datos vectorial.

Un corpus es un conjunto de datos o documentos que se utilizan como fuente de información para un sistema de recuperación. Este conjunto puede incluir textos en diversos formatos, como PDF, HTML, Word o Markdown.

Cuando el corpus está en formatos como los mencionados, el proceso comienza con la limpieza y extracción de los datos en bruto. Estos datos se transforman a un formato uniforme de texto plano, garantizando una base homogénea para las etapas posteriores.

\subsection{División del corpus}

Las técnicas de división de documentos consisten en fragmentar textos extensos en segmentos más pequeños~\cite{gong2020recurrentchunkingmechanismslongtext}. Esto se basa en dos principios. En primer lugar, cada fragmento debe ser semánticamente independiente y contener una idea clara, para minimizar la ambigüedad; por ejemplo, el término ``naranja'' puede referirse a una fruta o a un color, dependiendo del contexto en el que se utilice. En segundo lugar, trabajar con fragmentos más pequeños optimiza el uso de recursos en la etapa de codificación, ya que procesar documentos extensos puede ser computacionalmente costoso, mientras que dividirlos en partes más cortas permite un análisis menos demandante en términos de memoria y tiempo.

El principal desafío de las técnicas de división es encontrar el tamaño óptimo del fragmento para lograr un mejor equilibrio entre la semántica del texto y la eficiencia de codificación.

Existen tres tipos de técnicas de división:
\begin{itemize}
    \item División con longitud fija: Esta técnica es la más sencilla y consiste en dividir los documentos en fragmentos secuenciales de una longitud predeterminada. Por ejemplo, al especificar un parámetro de longitud, cada segmento contiene una cantidad fija de \textit{tokens}.
    \item División semántica: En este enfoque, los fragmentos se generan respetando las estructuras semánticas del texto. La segmentación puede realizarse utilizando delimitadores naturales, como el punto, la coma o el carácter de nueva línea, que marcan límites lógicos dentro del contenido. Herramientas de procesamiento de lenguaje natural como NLTK~\cite{nltk2001} y spaCy~\cite{spacy2016} facilitan este tipo de división al proporcionar métodos para identificar y dividir oraciones o párrafos.
    \item División basada en contenido: Este método utiliza características estructurales específicas del texto para realizar la segmentación. Es efectivo cuando los documentos contienen patrones o formatos definidos. Por ejemplo, el código de programación puede segmentarse en bloques de funciones.
\end{itemize}

Una vez que el texto ha sido dividido en fragmentos, se procede a la codificación de cada uno de ellos.

\subsection{Codificación de fragmentos}

La codificación se refiere a convertir datos textuales en representaciones vectoriales. La vectorización permite representar textos de manera que documentos con significados similares se encuentren más cerca entre sí en el espacio vectorial. 

Según la densidad de los vectores resultantes, existen dos tipos de métodos de codificación: codificación dispersa, donde la mayoría de los valores en el vector son ceros, y codificación densa, donde todos o casi todos los valores tienen contenido.

En la codificación dispersa se representa el texto creando vectores de alta dimensionalidad donde la mayoría de los elementos son ceros. Algunos ejemplos de codificación dispersa son:
\begin{itemize} 
    \item Codificación one-hot~\cite{harris2010digital}: Representa cada palabra mediante un vector binario de alta dimensionalidad, cuyo tamaño es igual al del vocabulario. En este vector, el valor correspondiente a la palabra es marcado como 1, mientras que el resto de los valores se mantienen en 0. Este método es sencillo, pero poco eficiente para representar relaciones semánticas entre palabras.
    \item Bolsa de palabras (BoW)~\cite{harris1954distributional}: Expande la codificación one-hot al contar la frecuencia de cada palabra en un documento. Aunque mejora la capacidad de representar la información textual, BoW no captura la estructura sintáctica ni el orden de las palabras. 
    \item TF-IDF (Term Frequency-Inverse Document Frequency)~\cite{rajaraman2011data}: Refina el modelo BoW ajustando la frecuencia de las palabras en función de su importancia relativa dentro de un corpus. Aumenta el peso de las palabras frecuentes en un documento, pero menos comunes en el resto del corpus, lo que ayuda a resaltar términos más significativos para la descripción del contenido de los documentos. 
\end{itemize}

En la codificación densa, se generan vectores en los que cada dimensión captura características semánticas del texto. Estos vectores, conocidos como \textit{embeddings}, representan matemáticamente palabras, frases o documentos en un espacio vectorial de alta dimensionalidad. En este espacio, cada entidad textual se asocia a un vector denso que refleja tanto relaciones semánticas como contextuales entre las palabras. Esto permite que términos con significados similares estén representados por vectores cercanos, facilitando la identificación de patrones y relaciones en el lenguaje.

Los \textit{embeddings} facilitan la búsqueda de información al permitir que los sistemas de recuperación se basen en la similitud semántica del contenido en lugar de realizar búsquedas por coincidencias exactas de palabras clave. Por ejemplo, si un usuario consulta por ``fármacos que alivian el dolor'', el sistema tiene la capacidad de identificar información relevante sobre ``medicamentos analgésicos'', incluso cuando esas palabras exactas no están presentes en los datos.

Los embeddings pueden generarse de diversas maneras, siendo las más comunes las basadas en redes neuronales profundas, especialmente \textit{transformers}~\cite{vaswani2017attention}, por ejemplo:

\begin{itemize} 
    \item BERT (Bidirectional Encoder Representations from Transformers) y variantes \cite{rajaraman2011data}: Estos modelos procesan el texto de manera bidireccional, lo que significa que consideran tanto el contexto anterior como el posterior a cada palabra. Al hacerlo, pueden captar relaciones entre palabras que se encuentran lejos unas de otras en el texto, ya que no dependen únicamente de la secuencia lineal, sino que analizan cómo las palabras se influyen mutuamente desde ambos lados. Esta técnica contrasta con modelos anteriores, que solo consideraban el contexto en una sola dirección.
    \item Encoders basados en LLM: Aprovechan la capacidad representacional avanzada de modelos de LLMs, como el modelo \textit{text-embedding-ada-002}~\cite{openai2022textembada}. Estos encoders producen embeddings de alta calidad que capturan complejas relaciones semánticas.
\end{itemize}

Una vez que los fragmentos han sido codificados en \textit{embeddings}, el siguiente paso es organizar estos vectores en una base de datos vectorial~\cite{wu2024retrievalaugmentedgenerationnaturallanguage}. Aunque es posible almacenarlos en bases de datos tradicionales u otros sistemas de almacenamiento, las bases de datos vectoriales resultan más adecuadas debido a su optimización para operaciones como la búsqueda por similitud y el cálculo de distancias en espacios multidimensionales~\cite{han2023comprehensivesurveyvectordatabase}.

\subsection{Indexación y construcción del \textit{Datastore}}

La indexación en bases de datos vectoriales es el proceso de organizar los vectores almacenados para facilitar su búsqueda eficiente. Consiste en construir estructuras de datos que permiten reducir el tiempo necesario para encontrar los vectores más similares a un vector de consulta, optimizando así las consultas \textit{k-nearest neighbors} (KNN) \cite{johnson2021billion}. La indexación se adapta a la métrica de similitud seleccionada.

Las métricas de similitud son fundamentales en los sistemas de recuperación, ya que cuantifican la relación entre el vector de consulta y los vectores almacenados en la base de datos. Estas métricas proporcionan un valor numérico que refleja la proximidad entre los vectores en un espacio vectorial. Algunas métricas de similitud comunes son:

\begin{itemize}
    \item Similitud Coseno: Evalúa el coseno del ángulo entre dos vectores en un espacio vectorial, considerando únicamente su orientación relativa y no su magnitud. La fórmula es:
    \[
    \text{Similitud coseno}(A, B) = \frac{A \cdot B}{\|A\| \|B\|},
    \]
    donde \(A \cdot B\) representa el producto escalar de \(A\) y \(B\), y \(\|A\|\) y \(\|B\|\) son las magnitudes de los vectores.
    
    \item Distancia Euclidiana: Mide la distancia directa entre dos elementos en un espacio multidimensional. Es útil para medir la proximidad entre vectores, aunque puede verse afectada por las escalas de los datos. La fórmula es:
    \[
    \text{Distancia euclidiana}(A, B) = \sqrt{\sum_{i=1}^{n} (A_i - B_i)^2},
    \]
    donde \(A_i\) representa el $i$-ésimo componente del vector \(A\).
    
    \item Distancia Manhattan: La distancia Manhattan, también conocida como L1, mide la suma de las diferencias absolutas entre las coordenadas de dos vectores. La fórmula es:
    \[
    \text{Distancia Manhattan}(A, B) = \sum_{i=1}^{n} |A_i - B_i|.
    \]
\end{itemize}

El \textit{datastore} en bases de datos vectoriales organiza la información en pares \textit{llave-valor}. Las llaves representan identificadores únicos de los \textit{embeddings} mientras que los valores almacenan información asociada relevante, como descripciones o metadatos. Un aspecto clave es determinar qué almacenar como valores; por ejemplo, en tareas de preguntas y respuestas, una estrategia efectiva es almacenar los \textit{embeddings} de preguntas como llaves y los pares \textit{pregunta-respuesta} como valores; esto facilita el proceso de generación al utilizar las recuperaciones como demostraciones para los modelos.

Una vez construido el componente de recuperación, el siguiente paso es entender cómo se realiza la consulta al sistema para obtener los resultados más relevantes.

\section{Consulta al componente de recuperación}

El proceso de consulta al componente de recuperación incluye tres pasos: codificación de la consulta, búsqueda de los fragmentos similares y post-procesamiento de la información recuperada.

Primeramente, para alinearse con el espacio de \textit{embeddings} preconstruido, se codifican las consultas utilizando el mismo codificador empleado durante la etapa de codificación del corpus.

Por ejemplo, si un usuario realiza la consulta:
``¿Qué es la fotosíntesis?'', esta se transforma en un vector de \textit{embedding} que representa semánticamente su contenido dentro del espacio vectorial preconstruido. Este espacio fue generado previamente al codificar un corpus que contiene textos como:
\begin{itemize} 
    \item ``La fotosíntesis es el proceso mediante el cual las plantas convierten la luz solar en energía química.'' 
    \item ``Durante este proceso, las plantas utilizan dióxido de carbono y agua para producir glucosa y liberar oxígeno.''
\end{itemize}

Una vez que la consulta ha sido codificada, el siguiente paso es la búsqueda de fragmentos similares en la base de datos. Esto se realiza utilizando algoritmos diseñados para identificar los \(k\) fragmentos más cercanos a la consulta en el espacio de \textit{embeddings} de la base de datos. Los métodos de búsqueda de vecinos más cercanos (KNN) permiten recuperar datos en función de su similitud con la consulta inicial y, por eso, se utilizan comúnmente~\cite{johnson2021billion}.

Después de recuperar los fragmentos similares a la consulta inicial, el \newline post-procesamiento de resultados busca refinar la información recuperada con el fin de presentarla de manera más relevante y alineada con las posibles necesidades del usuario. Este paso es importante, ya que los resultados obtenidos mediante algoritmos de búsqueda pueden contener información irrelevante, redundante o de menor calidad. Una de las técnicas de post-procesamiento más destacadas es \textit{re-ranking}~\cite{mortaheb2025rerankingcontextmultimodalretrieval}.

Luego de recuperar los textos relevantes, el siguiente paso consiste en ampliar la consulta original incorporando esta información, creando así una consulta extendida que sirva como entrada para el modelo generador.

\section{Fusiones de recuperación}

Las fusiones de recuperación se centran en cómo combinar la información recuperada con la consulta inicial para mejorar la calidad de las respuestas generadas. El objetivo de este proceso es utilizar la información recuperada de manera que enriquezca la respuesta final, brindando un contexto adicional que permita una mayor precisión y relevancia en la generación de respuestas.

Una técnica común es la concatenación de texto. Este enfoque consiste en unir los fragmentos o documentos recuperados con la consulta inicial, obteniendo de esta manera la consulta extendida. Este enfoque es especialmente útil cuando se trabaja con LLMs, los cuales operan como sistemas cerrados y generalmente se accede a ellos a través de APIs (Application Programming Interface), por ejemplo: OpenAI API \cite{openai_api} y Gemini API \cite{gemini_model}.

La consulta extendida se envía al LLM, lo que permite generar una respuesta más completa y contextualizada basada en la información relevante.

El diseño de plantillas de \textit{prompts} es importante en las fusiones de recuperación. Estas plantillas permiten integrar la información recuperada con la consulta inicial, facilitando el proceso de generación de respuestas. La optimización de estas plantillas mejora la interacción entre el sistema y el usuario, logrando que las respuestas sean más precisas y adecuadas al contexto~\cite{sahoo2024systematicsurveypromptengineering}.

Con la misma consulta ``¿Qué es la fotosíntesis?'' y los fragmentos recuperados, el sistema puede utilizar una plantilla de \textit{prompt} para integrar la información con la consulta inicial, adaptándola a un contexto educativo:
``Eres un consultor para estudiantes de biología de secundaria. Tu tarea es explicar conceptos de manera clara y sencilla. Pregunta: ¿Qué es la fotosíntesis? Contexto: La fotosíntesis es el proceso por el cual las plantas convierten la luz solar en energía química. Este proceso ocurre en los cloroplastos y utiliza dióxido de carbono, agua y luz solar. También produce oxígeno como subproducto, que es esencial para la vida en la Tierra. Respuesta:''.

Esta plantilla no solo integra la información recuperada, sino que también proporciona un marco contextual para ajustar el tono y el nivel de detalle de la respuesta, asegurando que sea adecuada para el tipo de usuario del sistema.

Después de tener la consulta extendida, el siguiente paso en el sistema RAG es generar la respuesta en lenguaje natural.

\section{Generación}

El proceso mediante el cual un LLM genera una respuesta comienza con la recepción de un \textit{prompt}. La respuesta generada depende de dos factores principales: el conocimiento paramétrico del modelo, adquirido durante su entrenamiento, y la información proporcionada en el \textit{prompt}.

El conocimiento paramétrico se refiere a los patrones, hechos y relaciones que el modelo ha aprendido durante su fase de entrenamiento, al ser expuesto a grandes volúmenes de texto y datos. Sin embargo, este conocimiento puede volverse desactualizado con el tiempo. Por ejemplo, un modelo entrenado con datos previos a ciertos eventos significativos puede carecer de información actualizada sobre esos eventos.

El \textit{prompt} es el mensaje o entrada proporcionado al modelo para generar una respuesta. Su función es establecer el tema o guiar al modelo hacia la información solicitada. La estructura del \textit{prompt} influye directamente en la respuesta generada, ya que define qué tipo de información será relevante y cómo debe presentarse. Una formulación precisa del \textit{prompt} ayuda al modelo a generar una respuesta que se ajuste mejor a las expectativas del usuario \cite{sahoo2024systematicsurveypromptengineering}.

En un sistema RAG, el \textit{prompt} aumentado, que incluye tanto la consulta original como los fragmentos recuperados, se presenta al LLM como entrada. La generación de la respuesta en esta etapa combina el conocimiento paramétrico del modelo con la información recuperada, lo que permite obtener respuestas más precisas y contextualizadas, en comparación con las generadas por el modelo sin mecanismos de recuperación \cite{gao2024retrievalaugmentedgenerationlargelanguage}.

RAG complementa las respuestas del LLM con información recuperada, permitiendo generar respuestas basadas en datos específicos y añadir referencias que permiten al usuario verificar su veracidad. 

El bot de Telegram que se presenta en el \Cref{chapter:implementation}, utiliza RAG para responder preguntas sobre conceptos del curso de Programación. Las respuestas incluyen referencias a la bibliografía oficial del curso.

Como parte de este trabajo, fue necesario recopilar ejercicios agrupados por los temas del curso de Programación de la carrera de Ciencia de la Computación en \mbox{MATCOM}, diseñar soluciones y desarrollar un conjunto de pistas progresivas para cada ejercicio. En el siguiente capítulo, se aborda esta parte del trabajo.
\chapter{Estrategia para la confección de ejercicios}\label{chapter:exercises}

Los ejercicios son una herramienta fundamental en el proceso de aprendizaje de la Programación, ya que permiten a los estudiantes aplicar y consolidar los conocimientos adquiridos, tal como se analizó en el capítulo~\ref{chapter:background}.

En este capítulo, se describe el proceso de recopilación y diseño de los ejercicios, así como el enfoque adoptado para presentar sus soluciones. Estas soluciones están redactadas de manera detallada, con el objetivo de servir como guía para que los estudiantes verifiquen sus respuestas y comprendan los pasos necesarios para resolver cada problema. Además, cada ejercicio tiene un conjunto de pistas, diseñadas para orientar a los estudiantes en la resolución de problemas sin revelar la solución completa.

Todos los ejercicios con sus respectivas soluciones y pistas están disponibles en el repositorio público: \href{https://github.com/anabel02/programming-exercise-lab}{github.com/anabel02/programming-exercise-lab}.

\section{Recopilación y organización de los ejercicios}\label{sec:exercises}

Los ejercicios que se presentan como parte de este trabajo han sido diseñados con el objetivo de asegurar que cada problema aporte al desarrollo de habilidades clave en los estudiantes. Cada ejercicio se alinea con los objetivos del curso. Esta alineación permite a los estudiantes entender cómo cada concepto se relaciona con los demás a lo largo de su formación.

Los problemas están enfocados a que los estudiantes no solo mejoren su fluidez en el lenguaje de programación, sino que también aprendan a aplicar sus conocimientos de manera práctica.

Los ejercicios se han recopilado de diversas fuentes, incluyendo:
\begin{itemize}
    \item Materiales de clases prácticas de Programación de años anteriores.
    \item Plataformas en línea de ejercicios como GeeksForGeeks~\cite{geeksforgeeks} y Leetcode~\cite{leetcode}.
    \item Libro de texto de la asignatura~\cite{katrib_programar}, que contiene problemas clásicos adaptados al currículo.
    \item Problemas creados específicamente para este trabajo, basados en experiencias docentes y necesidades de los estudiantes.
\end{itemize}

Los ejercicios recopilados se han organizado por temas para facilitar su integración en las clases prácticas. Los temas principales son los siguientes: 

\begin{itemize}  
    \item Operaciones básicas:  
    Ejercicios centrados en el uso de tipos de datos, operaciones aritméticas y manipulación de variables. Estos ejercicios introducen la secuencialidad de los programas.  

    \item Condicionales:  
    Problemas relacionados con el uso de estructuras de control condicionales (\texttt{if}, \texttt{else}, \texttt{else if}, \texttt{switch case}). Se enfoca en las operaciones \textit{booleanas} y la aplicación de lógica para seleccionar y ejecutar bloques de código según diferentes condiciones.

    \item Métodos:
    Ejercicios dedicados al diseño y uso de métodos, destacando su aplicación para estructurar soluciones mediante la modularidad del código.  

    \item Ciclos:  
    Problemas enfocados en estructuras iterativas (\texttt{for}, \texttt{while}, \texttt{do-while}), con énfasis en el control de iteraciones para resolver tareas repetitivas.

    \item Lectura de \textit{arrays}:  
    Ejercicios sobre el manejo de \textit{arrays}, incluyendo iteraciones, búsquedas y acceso a datos.  

    \item Modificación de \textit{arrays}:  
    Problemas que involucran operaciones como agregar, eliminar, ordenar o modificar elementos en \textit{arrays}.  

    \item Resolución de problemas:  
    Ejercicios diseñados para desarrollar habilidades en la descomposición y análisis de problemas mediante algoritmos.

    \item \textit{Arrays} bidimensionales:  
    Problemas relacionados con la manipulación de estructuras matriciales.  

    \item Tableros:  
    Ejercicios basados en la representación de tableros utilizados en juegos o simulaciones, con tareas que implican evaluación de estados en estructuras matriciales.
\end{itemize}  

Dentro de cada tema, los ejercicios se clasifican según su nivel de dificultad, buscando garantizar una progresión adecuada en el aprendizaje del estudiante. Los ejercicios se clasifican en tres niveles de dificultad: básico, intermedio y avanzado.

Los ejercicios básicos se enfocan en evaluar los conocimientos fundamentales adquiridos por los estudiantes sobre el tema. Son problemas claramente definidos en los que los estudiantes deben centrarse en la implementación. Se enfocan principalmente en la comprensión y aplicación de la sintaxis del lenguaje de programación.

Los ejercicios intermedios exigen una mayor integración de conceptos aprendidos previamente, fomentando el razonamiento lógico y la resolución de problemas.

Los ejercicios avanzados requieren resolver problemas que combinan varios conceptos y exigen análisis detallado. Los estudiantes deben explorar diferentes enfoques y estrategias para encontrar soluciones.

El ejemplo presentado en la figura~\ref{fig:basic} de la página~\pageref{fig:basic}, refleja la estructura utilizada para diseñar los ejercicios. Esta estructura tiene como objetivo garantizar claridad y consistencia e incluye un título, una descripción, definiciones necesarias, formato de entradas y salidas y ejemplos de ejecuciones. A continuación se describen estos elementos.

\begin{itemize}
    \item Título del ejercicio: Se incluye un título para identificar a los ejercicios.
    
    \item Descripción del problema:  
    Se indica la orden del ejercicio y el problema que se debe resolver. Por ejemplo: ``Escribe un programa que determine si un entero es primo o no''

    \item Definiciones: 
    Se incluye explicaciones breves de conceptos relacionados con el ejercicio que los estudiantes puedan no conocer o necesitar recordar.

    \item Definición formal:
    Siempre que es posible, se incluye una formulación matemática del problema para facilitar su comprensión teórica, reforzar el razonamiento lógico y abstracto, y evidenciar la conexión de Programación con otras asignaturas de primer año.
    
    \item Entradas y salidas:
    Se especifican los datos que el programa debe recibir como entrada y los resultados que debe generar como salida.

    \item Ejemplos:
    Se incluyen casos prácticos con sus respectivas entradas y salidas, acompañados de explicaciones cuando es necesario, para facilitar la comprensión del comportamiento esperado del programa.
\end{itemize}

\begin{figure}[h!]
    \centering
    \begin{minipage}{12cm}
    \begin{framed}
        \textbf{¿Es primo?}

        Escribe un programa que determine si un entero es primo o no. Un número entero positivo \( n \) se dice que es \textit{primo} si tiene exactamente dos divisores distintos: \( 1 \) y el propio número \( n \). Es decir, \( n \) es primo si y solo si no existen otros divisores \( d \) tal que \( 1 < d < n \) y \( d \) divide a \( n \). Formalmente, podemos escribir:
        \[
        n \text{ es primo} \iff  \forall \, d \in \mathbb{Z}^+ \, \text{se cumple que} \, d \mid n \, \implies \, d = 1 \text{ o } d = n,
        \]
        donde \( \mathbb{Z}^+ \) representa el conjunto de los números enteros positivos, y \( d \mid n \) denota que \( d \) divide a \( n \), es decir, \( n \) es divisible por \( d \).\\
        
        \textbf{Entrada:} Un número entero positivo \( n \). 
        
        \textbf{Salida:} Un valor booleano que indica si \( n \) es primo.\\
        
        \textbf{Ejemplos:}

        \centering
        \begin{tabular}{|c|c|}
        \hline
        Entrada & Salida \\
        \hline
        10      & false  \\
        29      & true   \\
        15      & false  \\
        31      & true   \\
        \hline
        \end{tabular}
    \end{framed}
    \end{minipage}
\caption{Ejemplo de ejercicio básico.}\label{fig:basic}
\end{figure}

Estos ejercicios se acompañan de soluciones que proporcionan explicaciones paso a paso para ayudar a los estudiantes a comprender los conceptos involucrados. En la siguiente sección, se describe el enfoque adoptado para la elaboración de estas soluciones.

\section{Solución de ejercicios}

El objetivo de estas soluciones es proporcionar una guía adicional para los estudiantes, ayudándoles a comprender no solo los resultados finales, sino también los enfoques seguidos para alcanzarlos. A través del análisis de cada solución, los estudiantes podrán observar cómo se aplican los conceptos teóricos en ejercicios prácticos.

Se recomienda que los estudiantes intenten resolver cada ejercicio por sí mismos antes de consultar las soluciones propuestas. Este enfoque fomenta la práctica activa y el aprendizaje autónomo, permitiendo que los estudiantes refuercen sus habilidades al enfrentarse directamente con el problema. Una vez que hayan logrado resolverlo, podrán utilizar las soluciones propuestas como una herramienta para comparar su solución y analizar posibles áreas de mejora.

Al revisar las soluciones proporcionadas, es recomendable que los estudiantes analicen no solo el código final, sino también las decisiones tomadas durante su desarrollo. Así, las soluciones actúan no solo como un medio de validación, sino también como una oportunidad para mejorar la comprensión teórica.

Las soluciones están redactadas de manera que los pasos sean comprensibles y sigan una secuencia lógica. Cada solución incluye comentarios o notas que explican el razonamiento detrás de las decisiones tomadas.

En la figura~\ref{fig:solution} de la página~\pageref{fig:solution} se presenta un ejemplo de solución para ilustrar cómo aplicar los conceptos discutidos en la clase de ciclos a un ejercicio práctico. Este ejemplo proporciona una solución al problema de determinar si un número es primo, mostrando tanto una implementación inicial como una versión optimizada.

\begin{figure}[h!]
    \centering
    \begin{minipage}{12cm}
    \begin{framed}
        \textbf{Solución: ¿Es primo?}

        Dado que un número \(n\) es primo si solo es divisible por 1 y por sí mismo, y además se conoce que si \(d\) es divisor de \(n\) entonces \(d \leq n\), para determinar si un número \(n\) es primo, basta con verificar si es divisible por algún número \(d\) tal que \(2 \leq d < n\). Si encontramos algún divisor en ese rango, podemos concluir que \(n\) no es primo.

        Una primera implementación podría iterar desde \(2\) hasta \(n - 1\), verificando si \(n\) es divisible por algún número:
      
    \begin{lstlisting}
    bool IsPrime(int n)
    {
        int d = 2;
        while (d < n)
        {
            if (n % d == 0) return false;
            else d++;
        }
        return true;
    }
    \end{lstlisting}
        
        La solución anterior es correcta, aunque realiza iteraciones innecesarias, ya que basta con comprobar divisores hasta la raíz cuadrada de \(n\). Esto se debe a que si \(n\) tiene un divisor mayor que \(\sqrt{n}\), necesariamente debe tener otro menor que \(\sqrt{n}\). Reduciendo el rango de búsqueda, podemos mejorar la eficiencia del algoritmo:
        
    \begin{lstlisting}
    bool IsPrime(int n)
    {
        int d = 2;
        int sqr = (int)Math.Sqrt(n);
        while (d <= sqr)
        {
            if (n % d == 0) return false;
            d++;
        }
        return true;
    }
    \end{lstlisting}
    \end{framed}
    \end{minipage}
\caption{Ejemplo de solución.}\label{fig:solution}
\end{figure}

Las soluciones propuestas no son respuestas definitivas, sino ejemplos de las decisiones tomadas durante el proceso de resolución de cada ejercicio. El objetivo de presentar las soluciones es que el estudiante pueda identificar los pasos necesarios para resolver un problema, aprender a estructurar el código y conocer buenas prácticas para abordar problemas de programación. Además, se busca facilitar la identificación de errores comunes en el desarrollo de soluciones y fomentar la capacidad de analizar diferentes formas de abordar un mismo problema.

Antes de consultar las soluciones, es fundamental que los estudiantes desarrollen la capacidad de resolver problemas de manera independiente. Para apoyar al estudiante en caso de que encuentre dificultades al resolver un ejercicio, se han diseñado pistas que guían a los estudiantes paso a paso en la resolución.

\section{Pistas para la resolución de ejercicios}

Las pistas son una herramienta complementaria diseñada para guiar a los estudiantes en la resolución de los ejercicios propuestos. A diferencia de las soluciones completas, las pistas no proporcionan el código directamente, sino que ofrecen orientación paso a paso para que los estudiantes puedan avanzar en la resolución de problemas por sí mismos. 

Cada pista dentro de un ejercicio está numerada según el orden en que debe ser consultada. Por ejemplo, para el ejercicio presentado en la figura~\ref{fig:basic}, que consiste en identificar si un número es primo o no, se brindan las siguientes pistas:
\begin{enumerate}
    \item Conocemos que un número primo es un número entero mayor que 1 que solo es divisible por 1 y por sí mismo. ¿Cómo puedes verificar si un número es divisible por otro?
    \item Sea \(n\) un número entero, todos sus divisores son menores o iguales que \(n\). ¿Cómo puedes verificar la divisibilidad desde 2 hasta \(n - 1\)?
    \item Si el número es 2, es primo. Si es menor que 2 no es primo. Para números \(n\) mayores que 2, solo necesitas hacer un bucle desde \(d = 2\) hasta \(d = n -1\) verificando si n es divisible entre d, o sea comprobar que \(n \% d == 0\).
\end{enumerate}

Al proporcionar ayuda gradual, el estudiante debería sentirse menos abrumado por la complejidad del problema, lo que facilita su avance. Este sistema de pistas escalonadas fomenta que el estudiante avance a su propio ritmo, lo que contribuye a desarrollar confianza en sus habilidades. Además, al descubrir la solución por sí mismo, puede interiorizar los conceptos y técnicas aplicadas, lo que refuerza su comprensión.

Como parte de este trabajo, se ha desarrollado y recopilado una colección de 113 ejercicios, organizados según los temas mencionados en la sección~\ref{sec:exercises}. Estos ejercicios están diseñados específicamente para las clases prácticas de las primeras 8 semanas del curso de Programación de Ciencia de la Computación en \mbox{MATCOM}.

Con el objetivo de facilitar el aprendizaje y la resolución de problemas, se han incorporado pistas a 18 ejercicios correspondientes al tema de arrays. Asimismo, se ha redactado la solución de 40 ejercicios, proporcionando un recurso adicional para el estudio y la autoevaluación de los estudiantes.

En el siguiente capítulo se presenta un sistema que proporciona recomendaciones de ejercicios según el tema y nivel de dificultad en que se encuentre el estudiante. Este recurso se propone para apoyar aún más el proceso de estudio, permitiendo un aprendizaje adaptativo y centrado en las necesidades individuales.
\chapter{Descripción del sistema propuesto}\label{chapter:proposal}

En este capítulo se presenta una propuesta de un sistema diseñado para apoyar el estudio independiente en el curso de Programación de la carrera de Ciencia de la Computación de \mbox{MATCOM}.

El sistema determina el nivel de cada estudiante y, en función de este, propone ejercicios adaptados a sus necesidades. Los estudiantes pueden solicitar pistas para resolver los ejercicios. El sistema registra el estado de cada actividad: si el estudiante ha comenzado a trabajar en ella, si tiene dudas o si ha completado el ejercicio, permitiendo un seguimiento detallado de su progreso.

Por otro lado, los profesores pueden monitorear el avance de los estudiantes, identificar dificultades comunes y ajustar el contenido o las estrategias de enseñanza según sea necesario. Esta información les ayuda a brindar retroalimentación oportuna y a personalizar el apoyo ofrecido.

La implementación de estas funcionalidades se realizó mediante un bot de Telegram. Sin embargo, es importante destacar que estas tareas también podrían realizarse con herramientas más simples, como hojas de cálculo o archivos de texto. De hecho, durante varios años, algunos profesores de \mbox{MATCOM} han gestionado procesos similares utilizando estas herramientas básicas~\cite{rodriguez}. En este sentido, la aplicación no es solo una solución tecnológica, sino una propuesta para mejorar la experiencia de aprendizaje.

En la sección~\ref{sec:students} se describe el flujo propuesto para los estudiantes. En la sección~\ref{sec:teachers} se explican las acciones disponibles para los profesores.

\section{Flujo para los estudiantes}\label{sec:students}

A continuación, se describe el flujo de uso del sistema propuesto para los estudiantes, detallando las etapas por las que pasa el estudiante y cómo cada etapa contribuye a su proceso de aprendizaje.

\subsection{Registro}

El primer paso para el estudiante es registrarse en el sistema. Para ello, debe proporcionar su nombre, apellidos y grupo. Este paso permite llevar un registro de los estudiantes y facilita su seguimiento a lo largo del curso. 

Una vez completado el registro, el estudiante puede comenzar a explorar los temas disponibles, lo que le permitirá alinear su estudio independiente con las clases que recibe.

\subsection{Listado de temas}

Cuando el usuario lo solicita, se le presenta un listado de los temas disponibles para su estudio. Esta funcionalidad permite al estudiante identificar rápidamente los temas que se ajustan al contenido visto en conferencias y clases prácticas. Por ejemplo, algunos de los temas disponibles pueden ser condicionales, \textit{arrays} y \textit{backtracking}.

Si al usuario no le queda claro de qué va el tema solamente por el nombre, también puede pedir una descripción del mismo, que incluye por qué se necesita estudiar el tema, definiciones básicas y ejemplos. Esta funcionalidad es útil para estudiantes que están comenzando y necesitan una comprensión más profunda de los conceptos básicos.

A medida que el estudiante explora los temas, ya sea en conferencias, clases prácticas o durante su estudio independiente, es común que surjan dudas o preguntas sobre el contenido. Para apoyar este proceso el sistema propuesto ofrece una sección dedicada a la resolución de dudas.

\subsection{Resolución de dudas}

Tradicionalmente, el estudiante podría recurrir a buscar respuestas a sus dudas en el libro de texto, pero este proceso puede ser lento y no siempre resuelve la duda. Otra opción sería utilizar herramientas como \textit{ChatGPT}; sin embargo, estas pueden generar alucinaciones, es decir, respuestas que parecen coherentes pero que son incorrectas, lo cual representa un riesgo, especialmente para estudiantes que aún no han desarrollado la capacidad crítica necesaria para distinguir entre información precisa y errónea.

Para simplificar este proceso, el sistema incluye una sección dedicada a la resolución de dudas. El estudiante puede formular preguntas en lenguaje natural, y el sistema responde con explicaciones, acompañadas de referencias a la bibliografía oficial del curso. Esta funcionalidad garantiza que las respuestas estén alineadas con el material de estudio, de esta manera el estudiante puede ir a consultar en el libro si lo necesita. 

Para implementar esta funcionalidad, se puede utilizar un sistema de Generación Aumentada por Recuperación (RAG, por sus siglas en inglés), que se presenta en el Capítulo~\ref{chapter:rag}. 

Además de resolver dudas, el sistema también ofrece herramientas para que los estudiantes practiquen y refuercen sus conocimientos a través de ejercicios adaptados a su nivel.

\subsection{Práctica con ejercicios sugeridos}

El usuario puede solicitar ejercicios relacionados con un tema específico. El sistema le sugiere ejercicios adaptados al nivel de conocimiento y progreso del estudiante, el cual se determina en función de la cantidad y complejidad de los ejercicios que ha resuelto previamente del tema que solicita. Estos ejercicios están diseñados para reforzar los conceptos aprendidos y preparar al estudiante para enfrentar problemas de mayor complejidad.

Los ejercicios se presentan en un orden progresivo, comenzando con problemas básicos y avanzando gradualmente hacia aquellos de mayor dificultad.

Una vez que el estudiante resuelve un ejercicio, envía el código y el sistema proporciona retroalimentación, lo que le permite identificar aciertos y errores para mejorar su desempeño.

\subsection{Retroalimentación}

Después de que el estudiante envía la solución a un ejercicio, el sistema evalúa la solución y le informa al estudiante si es correcta o incorrecta. Si la solución es correcta, se le puede sugerir al estudiante que explore formas de mejorar la eficiencia del código, fomentando el desarrollo de habilidades de programación más avanzadas.

En caso de que la solución sea incorrecta, el sistema identifica el error y se lo comunica al estudiante. Además, el sistema registra la cantidad de intentos que el estudiante ha realizado para resolver el ejercicio, lo que permite monitorear su progreso y adaptar futuras recomendaciones de ejercicios según su desempeño. Esta funcionalidad tiene como objetivo ayudar a identificar áreas de mejora y motivar al estudiante a persistir en la resolución de problemas.

Si el estudiante se encuentra con dificultades durante la resolución de un ejercicio, se le ofrecen pistas para guiarlo. A continuación se explica cómo funciona este proceso.

\subsection{Solicitud de pistas}

Durante la resolución de ejercicios, es posible que el estudiante enfrente dificultades y se sienta estancado. Para apoyarlo en estos momentos, el sistema ofrece la opción de solicitar pistas. Las pistas están organizadas comenzando con sugerencias generales y avanzando hacia indicaciones más específicas. 

Este sistema permite al estudiante recibir ayuda gradualmente, sin revelar la solución completa, con la intención de que lo motive a descubrirla por sí mismo mientras refuerza su comprensión del tema. Por ejemplo, si el ejercicio consiste en implementar un algoritmo de ordenamiento, la primera pista podría ser una descripción general del algoritmo, mientras que las pistas posteriores podrían detallar cómo implementar cada paso o cómo manejar casos específicos.

Este sistema de pistas escalonadas busca evitar que el estudiante se sienta abrumado, a la vez que le permite avanzar a su propio ritmo. Además, al descubrir la solución por sí mismo, el estudiante puede ganar confianza en sus habilidades y desarrollar una mayor capacidad para resolver problemas de manera independiente.

En caso de que el estudiante no logre resolver el ejercicio incluso con las pistas, se ofrece la opción de consultar la solución completa. En la siguiente sección se explica esta funcionalidad.

\subsection{Solicitud de soluciones}

Si el estudiante, a pesar de utilizar las pistas escalonadas, no logra resolver el ejercicio, el sistema ofrece la opción de solicitar la solución completa. Esta funcionalidad está pensada como último recurso, para evitar que el estudiante se frustre o pierda demasiado tiempo en un solo problema.

Al solicitar la solución, no solo se muestra el código o la respuesta final, sino que también se incluye una explicación detallada de cada paso. Esto permite al estudiante comprender el razonamiento detrás de la solución y aprender de sus errores. Esta solución no está generada por un LLM, sino que está escrita por profesores, basándose en su experiencia y en las particularidades de las clases de la facultad.

Además, incluso si el estudiante logra resolver el ejercicio por sí mismo, se le ofrece la opción de consultar la solución para que pueda comparar su enfoque con el propuesto. Consultar una solución alternativa puede aportar nuevas perspectivas, mostrar métodos más eficientes o simplemente reforzar los conceptos aprendidos. Por ejemplo, un estudiante que resolvió un problema usando iteración podría descubrir una solución recursiva que no había considerado, lo que amplía su comprensión del tema.

Además de apoyar a los estudiantes, el sistema también ofrece funcionalidades específicas para los profesores. En la siguiente sección, se detallan las herramientas disponibles para los docentes.

\section{Flujo para los profesores}\label{sec:teachers}

El sistema está diseñado para que los profesores puedan gestionar y organizar el contenido del curso, adaptándolo a las necesidades específicas de los estudiantes. A continuación, se detallan las funcionalidades disponibles para los docentes.

\subsection{Creación de contenido}

El profesor tiene la capacidad de crear y editar los temas que conforman el curso. Además, el profesor puede añadir ejercicios prácticos diseñados para reforzar los conceptos teóricos. Cada ejercicio cuenta con una solución modelo para que los estudiantes verifiquen su trabajo, y un conjunto de pistas escalonadas que orientan al estudiante.

Además, el profesor puede modificar la bibliografía referenciada por el módulo de resolución de dudas. No se limita al libro de texto de la asignatura, sino que puede añadir otras fuentes confiables, como libros o notas de clase. Esto asegura que los estudiantes tengan acceso a información actualizada y de calidad para resolver sus dudas.

Esta flexibilidad en la gestión de contenido permite al profesor adaptar el material al ritmo del curso y a las necesidades individuales de los estudiantes, asegurando que los recursos estén alineados con los objetivos del curso.

\subsection{Monitoreo y análisis del progreso}

El sistema recopila datos sobre el desempeño de los estudiantes, lo que permite a los profesores generar reportes. Estos reportes incluyen información sobre la actividad de los estudiantes, como el número de ejercicios completados, la cantidad de pistas solicitadas y las veces que se ha accedido a la solución. También se proporcionan estadísticas de desempeño, como el porcentaje de ejercicios resueltos correctamente y el tiempo promedio dedicado a cada tema.

Además, el sistema permite analizar el desempeño grupal, destacando áreas fuertes y débiles. Esto facilita la identificación de patrones comunes que puedan requerir refuerzo en clase. 

Estos reportes no solo permiten a los profesores evaluar el progreso individual y grupal, sino también tomar decisiones informadas para ajustar el contenido o la metodología de enseñanza, optimizando así el proceso de aprendizaje. Además, ayudan a identificar a estudiantes que necesiten apoyo adicional para ofrecerles recursos complementarios, promoviendo un aprendizaje más personalizado.

En este capítulo se ha presentado una propuesta diseñada para apoyar el estudio independiente en el curso de Programación de la carrera de Ciencia de la Computación en \mbox{MATCOM}.

El flujo propuesto está diseñado para guiar al estudiante a través de su proceso de aprendizaje. Cada etapa está planificada para fomentar la autonomía, el pensamiento crítico y el progreso continuo. Esta propuesta tiene el potencial de transformar la manera en que los estudiantes y profesores abordan el estudio de la programación, ofreciendo un apoyo personalizado y accesible en cualquier momento.

En el siguiente capítulo se detalla la implementación de este sistema mediante un bot de Telegram.
\chapter{Detalles de implementación}\label{chapter:implementation}

En este capítulo se describen los detalles de implementación de un bot de Telegram diseñado para apoyar el estudio independiente de los estudiantes de programación. En la Sección~\ref{sec:technical-details} se presentan las tecnologías y herramientas utilizadas, así como las razones detrás de su selección. En la Sección~\ref{sec:features} se analiza la arquitectura del sistema y se describen las funcionalidades principales, como el registro de usuarios, el listado de temas, la resolución de dudas y la práctica con ejercicios. Finalmente, se discute cómo los profesores pueden personalizar el contenido del curso.

\section{Detalles técnicos}\label{sec:technical-details}

Se desarrolla un bot de Telegram debido a que los estudiantes de la facultad ya están familiarizados con esta herramienta desde el primer año de la carrera. En \mbox{MATCOM}, Telegram se utiliza ampliamente como medio de comunicación, donde cada asignatura cuenta con al menos un grupo de Telegram para facilitar la interacción entre estudiantes y profesores, algunas asignaturas también disponen de canales. 

Esta adopción generalizada de Telegram elimina la necesidad de que los estudiantes aprendan a manejar una nueva herramienta o incorporen otra aplicación a su rutina, evitando distracciones adicionales. El bot propuesto simplemente se integra como un nuevo chat en una plataforma que los estudiantes ya utilizan con regularidad.

Otra razón importante para esta elección son las ventajas técnicas que ofrece Telegram. Su API es gratuita y está bien documentada, lo que facilita la implementación de funcionalidades. Asimismo, Telegram es accesible desde múltiples dispositivos (dispositivos móviles, web y desktop) e incluso permite sesiones simultáneas, garantizando una disponibilidad constante y comodidad para los estudiantes.

El bot se desarrolla utilizando Python como lenguaje de programación debido a su amplia gama de bibliotecas. Para la gestión de datos, se utilizó PostgreSQL~\cite{postgresql}, un sistema de bases de datos relacional escalable y de código abierto.

En la siguiente sección, se describe el diseño del sistema mediante diagramas que representan las funcionalidades principales, proporcionando una visión de cómo interactúan los componentes del bot.

\section{Arquitectura y funcionalidades del sistema}\label{sec:features}

El esquema de la base de datos está estructurado en varias tablas interrelacionadas \texttt{users}, \texttt{students}, \texttt{topics}, \texttt{exercises}, \texttt{exercise\_hints}, \texttt{student\_hints} y \texttt{student\_exercise} que gestionan la información de usuarios, estudiantes, temas, ejercicios y pistas.

En esta sección se describe el diseño del sistema mediante diagramas que representan las funcionalidades principales. La Figura~\ref{fig:legend} muestra una leyenda que explica los componentes utilizados en los diagramas. Las flechas indican el flujo de la interacción.

\begin{figure}[h!]
  \centering
      \begin{tikzpicture}[node distance=0.5cm and 1cm]
      \node (startstop) [startstop] {Estudiante ingresa\\ nombre y apellidos};
      \node (startstopLabel) [right=of startstop, font=\footnotesize] {Interacción del usuario};
  
      \node (process) [process, below=of startstop] {Mensaje de bienvenida y\\ lista de comandos disponibles};
      \node (processLabel) [right=of process, font=\footnotesize] {Respuesta del sistema};
  
      \node (decision) [decision, below=of process] {¿Quedan \\ ejercicios \\ disponibles?};
      \node (decisionLabel) [right=of decision, font=\footnotesize] {Punto de decisión};
      \end{tikzpicture}
      \caption{Leyenda de los componentes de los diagramas.}\label{fig:legend}
\end{figure}

Las funcionalidades disponibles son el registro de los estudiantes, listar los temas de la asignatura, responder dudas sobre el contenido y recomendar ejercicios, así como ofrecer pistas para guiar en la resolución de los ejercicios. El primer paso es el registro de los estudiantes.

\subsection{Registro}

Al acceder por primera vez, la herramienta solicita al estudiante ingresar su nombre completo. Luego, muestra un mensaje de bienvenida y una guía básica que explica las funcionalidades disponibles, siguiendo el flujo descrito en la Figura~\ref{fig:register}. Esta introducción tiene como objetivo que el usuario comprenda cómo aprovechar la aplicación para su estudio.

\begin{figure}[h!]
  \centering
  \begin{tikzpicture}[node distance=1cm and 1cm]
  \node (registro) [startstop] {/start};
  \node (sDatos) [process, below=of registro]{Solicitud de información personal};
  \node (datos) [startstop, below=of sDatos]{Estudiante ingresa nombre y apellidos};
  \node (bienvenida) [process, right=of datos]{Mensaje de bienvenida y\\ lista de comandos disponibles};

  \draw [arrow2] (registro) -- (sDatos);
  \draw [arrow2] (sDatos) -- (datos);
  \draw [arrow2] (datos) -- (bienvenida);
  \end{tikzpicture}
  \caption{Diagrama del flujo del registro.}\label{fig:register}
\end{figure}

Después del registro, lo más común es que el estudiante revise el listado de temas disponibles para comenzar su estudio.

\subsection{Listado de temas}

Cuando el usuario lo solicita, la aplicación muestra un listado organizado de los temas disponibles para su estudio. Los temas se presentan en el orden definido en el diseño del curso de Programación, que se describe en la Sección~\ref{sec:matcom}.

Si el nombre del tema no resulta suficientemente claro, el estudiante puede solicitar una descripción detallada del mismo. Este flujo se ilustra en la Figura~\ref{fig:topics} de la página~\pageref{fig:topics}.

\begin{figure}[h!]
  \centering
  \begin{tikzpicture}[node distance=1cm and 1cm]
    \node (solicitud) [startstop] {/topics};
    \node (listado) [process, below=of solicitud] {Listado de temas organizados};
    \node (explorar) [decision, below=of listado] {¿El estudiante \\ desea más detalles \\ sobre un tema?};
    \node (descripcionSolicitud) [startstop, right=of explorar]{/topic [título del tema]};
    \node (descripcion) [process, below=of descripcionSolicitud] {Descripción del tema};
    
    \draw [arrow2] (solicitud) -- (listado);
    \draw [arrow2] (listado) -- (explorar);
    \draw [arrow2] (explorar) -- node[above] {Sí} (descripcionSolicitud);
    \draw [arrow2] (descripcionSolicitud) -- (descripcion);
  \end{tikzpicture}
\caption{Diagrama del flujo del listado temas.}\label{fig:topics}
\end{figure}

Durante el estudio o después de una clase, es común que los estudiantes tengan dudas sobre el contenido, como por ejemplo: ``¿qué es un array de direcciones?''. Para resolver estas inquietudes, la aplicación cuenta con un módulo de resolución de dudas.

\subsection{Sistema de resolución de dudas basado en RAG}

La aplicación incluye un sistema de resolución de dudas que permite a los estudiantes formular preguntas en lenguaje natural. Utilizando un RAG, el sistema proporciona respuestas basadas en la bibliografía del curso. La Figura~\ref{fig:q-a-diagram} de la página~\pageref{fig:q-a-diagram} muestra el funcionamiento de este sistema.

\begin{figure}[h!]
  \centering
    \begin{tikzpicture}[
      font=\sffamily,
      every node/.style={align=center},
      box/.style={draw, rectangle, minimum width=3cm, minimum height=1cm},
      arrow/.style={-Latex},
      db/.style={cylinder, cylinder uses custom fill, cylinder body fill=blue!20, shape border rotate=90, aspect=0.25, minimum height=1.5cm, minimum width=1cm, draw}
    ]
    
    \node[box] (materials) {Libro de Katrib};
    \node[box, below=1cm of materials] (question) {/ask [pregunta]};
    \node[box, right=0.8cm of question] (embedding) {Función \\de \\Embedding};
    \node[db, right=2cm of embedding] (database) {Base de Datos \\Vectorial};
    \node[box, below=2.5cm of database] (prompt) {Prompt Extendido = \\Pregunta +\\Contenido del libro};
    \node[box, below=1.5cm of prompt] (llm) {LLM};
    \node[box, left=2.5cm of llm] (reply) {Respuesta \\con referencias al libro};
    \node[below=1.5cm of question] (user) {
      \begin{tikzpicture}[scale=0.8]
        \draw (0,0.7) circle [radius=0.3]; % Cabeza
        % Cuerpo
        \draw (0,0.4) -- (0,-0.5); % Línea del cuerpo
        \draw (-0.4,-1) -- (0,-0.5) -- (0.4,-1); % Piernas
        \draw (-0.3,-0.2) -- (0,0.2) -- (0.3,-0.2); % Brazos
      \end{tikzpicture}
      \\ Estudiante
    };
    
    \draw[arrow] (materials) -- (embedding);
    \draw[arrow] (user) -- (question);
    \draw[arrow] (question) -- (embedding);
    \draw[arrow, bend left=15] (embedding) to node[above] {Guardar} (database);
    \draw[arrow, bend right=15] (embedding) to node[below] {Búsqueda \\por \\similitud} (database);
    
    \draw[arrow] (database) -- node[right] {Fragmentos \\recuperados} (prompt);
    \draw[arrow] (prompt) -- (llm);
    \draw[arrow] (llm) -- (reply);
    \draw[arrow] (reply) -- ++(0,1) -| (user);
    
    \end{tikzpicture}
    \caption{Sistema de preguntas y respuestas basado en Generación Aumentada por Recuperación (RAG).}\label{fig:q-a-diagram}
\end{figure}

Como se explica en el Capítulo~\ref{chapter:rag} el proceso de construcción del \textit{retriever} consta de tres etapas: dividir el corpus en fragmentos, codificar los fragmentos y construir la base de datos vectorial.

El corpus del sistema está compuesto por los capítulos del libro ``Empezar a programar: Un enfoque multiparadigma con C\#'' en formato PDF. Este contenido se convierte a texto utilizando herramientas de extracción que preservan su estructura original. Para manejar el contenido, se divide en fragmentos de longitud fija de 2000 caracteres con un solapamiento de 200 caracteres, con el objetivo de que no se pierda el contexto entre fragmentos consecutivos. Cada fragmento se transforma en una representación numérica mediante el modelo de embeddings \textit{models/text-embedding-004}.

Durante este proceso, a cada fragmento se le asocian metadatos, como el nombre del documento y el número de página correspondiente. Estos metadatos se almacenan junto con los vectores en la base de datos vectorial, lo que permite, durante la fase de generación, incorporar referencias en las respuestas proporcionadas al usuario.

Cuando un estudiante formula una pregunta, esta se procesa mediante el mismo modelo \textit{models/text-embedding-004}, que convierte la pregunta en una representación vectorial. Para medir la relevancia entre una consulta y los fragmentos almacenados en la base de datos vectorial, se emplea la similitud coseno. El sistema utiliza los 5 fragmentos más similares para generar el \textit{prompt} extendido, que combina la pregunta con el contenido relevante del libro. El \textit{prompt} extendido se envía al modelo \textit{gemini-1.5-flash}, que genera una respuesta.

Este flujo de trabajo muestra cómo el sistema RAG combina la búsqueda de información con la generación de texto, con el objetivo de reducir el tiempo que los estudiantes invertirían en buscar respuestas a través de medios tradicionales, como la revisión manual del libro o la consulta directa al profesor. Al incorporar referencias específicas al libro, el sistema no solo proporciona respuestas más confiables, sino que también fomenta la capacidad crítica de los estudiantes, permitiéndoles contrastar la información generada con fuentes autorizadas. De esta manera, se busca minimizar la dependencia exclusiva de LLMs o fuentes en línea que podrían no ser fiables, especialmente para usuarios con poca experiencia.

Además de resolver dudas, la aplicación cuenta con un sistema de sugerencia de ejercicios, así como apoyo en cada paso de la resolución de estos ejercicios.

\subsection{Práctica con ejercicios sugeridos. Pistas y soluciones.}

La aplicación ofrece una experiencia de aprendizaje personalizada al sugerir ejercicios adaptados al nivel del estudiante. Durante la resolución, el estudiante puede solicitar pistas que se dan en orden. En cualquier momento, el estudiante puede acceder a la solución completa. Una vez que el estudiante completa su implementación la manda al bot. Este flujo está descrito en la Figura~\ref{fig:exercises} de la página~\pageref{fig:exercises}.

El nivel del estudiante se calcula en función de su desempeño previo en la aplicación, considerando dos factores principales: la cantidad de ejercicios resueltos y la complejidad de los ejercicios que ha completado. Por ejemplo, si un estudiante resuelve 5 ejercicios básicos sin pedir pistas ni soluciones, la aplicación comenzará a proponer problemas de nivel intermedio.

\begin{figure}[h!]
  \centering
  \begin{tikzpicture}[node distance=1cm and 1cm]

    \node (start) [startstop] {/exercise [título del tema]};
    \node (disponibles) [decision, below=of start] {¿Quedan \\ ejercicios \\ disponibles?};
    \node (sugerencia) [process, below=of disponibles] {Sugerencia de ejercicio};
    \node (completado) [process, right=of disponibles] {Felicidades, Completaste \\ todos los ejercicios \\ de este tema};
    \node (conversion) [decision, below=of sugerencia] {¿El estudiante \\ sabe resolverlo?};
    \node (submit) [startstop, right=of conversion] {/submit [código]};
    \node (pista) [startstop, below=of conversion] {/hint [número del ejercicio]};
    \node (pistasDisponibles) [decision, left=of pista] {¿Quedan \\ pistas \\ disponibles?};
    \node (noPistas) [process, below=of pistasDisponibles] {Lo siento, no quedan \\ pistas para este \\ ejercicio};
    \node (sugerenciaPista) [process, above=of pistasDisponibles] {Sugerencia de pista};
    
    \draw [arrow2] (start) -- (disponibles);
    \draw [arrow2] (disponibles) -- node[above] {No} (completado);
    \draw [arrow2] (disponibles) -- node[left] {Sí} (sugerencia);
    \draw [arrow2] (sugerencia) -- (conversion);
    \draw [arrow2] (conversion) -- node[above] {Sí} (submit);
    \draw [arrow2] (conversion) -- node[left] {No} (pista);
    \draw [arrow2] (pista) -- (pistasDisponibles);
    \draw [arrow2] (pistasDisponibles) -- node[left] {No} (noPistas);
    \draw [arrow2] (pistasDisponibles) -- node[left] {Sí} (sugerenciaPista);
    \draw [arrow2] (sugerenciaPista) --  (conversion);
    
  \end{tikzpicture}
\caption{Diagrama del flujo del sistema de ejercicios.}\label{fig:exercises}
\end{figure}

Para mejorar esta funcionalidad, se propone implementar un sistema de evaluación automática de la correctitud de la solución. Actualmente, cuando un estudiante envía su código, este se almacena en la base de datos, el ejercicio se marca como ``pendiente de revisión'' y queda a la espera de que un profesor lo revise. Si bien este enfoque es funcional, la retroalimentación que proporciona al estudiante puede demorar.

Con la nueva propuesta, el sistema de evaluación automática ejecutará el código del estudiante y comparará la salida generada con los casos de prueba predefinidos. Si la salida coincide con los resultados esperados, el ejercicio se marcará como resuelto. En caso contrario, el sistema proporcionará retroalimentación inmediata al estudiante sobre los errores encontrados. De todas formas, con este sistema de evaluación automática, también se almacenaría el código de las soluciones en la base de datos, dejando al profesor la posibilidad de revisarlos.

A continuación se describe cómo los profesores podrían cambiar el contenido del curso, ya sea los materiales de los que se recupera información para resolver dudas, como los ejercicios.

\subsection{Personalización del curso}

Si se desea cambiar los temas y ejercicios del curso, es necesario crear un archivo \texttt{topics.json} en la carpeta \texttt{data}. Este archivo debe seguir un formato específico para que el sistema pueda interpretarlo correctamente.

El contenido del archivo debe ser un objeto JSON con un campo llamado topics, que es un array de temas. Cada tema debe contener la siguiente información:
\begin{itemize}
    \item \texttt{title} (tipo \texttt{string}): Título del tema.
    \item \texttt{description} (tipo \texttt{string}): Descripción general del tema.
    \item \texttt{exercises} (tipo \texttt{array}): Lista de ejercicios asociados al tema. Cada ejercicio contiene:
    \begin{itemize}
        \item \texttt{title} (tipo \texttt{string}): Título del ejercicio.
        \item \texttt{content} (tipo \texttt{string}): Enunciado o descripción del ejercicio.
        \item \texttt{difficulty} (tipo \texttt{string}): Dificultad del ejercicio: \textit{basic}, \textit{intermediate} o \textit{advanced}.
        \item \texttt{solution} (tipo \texttt{string}): Solución completa del ejercicio.
        \item \texttt{hints} (tipo \texttt{array}): Lista de pistas asociadas al ejercicio, donde cada pista tiene:
        \begin{itemize}
            \item \texttt{content} (tipo \texttt{string}): Texto de la pista.
            \item \texttt{order} (tipo \texttt{number}): Nivel de detalle de la pista, que define el orden en que se entregarán al usuario. Las pistas se muestran de menor a mayor nivel (0, 1, 2, \ldots), se recomienda comenzar con sugerencias generales e ir avanzando hacia indicaciones más específicas.
        \end{itemize}
    \end{itemize}
\end{itemize}

En la Figura~\ref{fig:json} de la página~\pageref{fig:json}, se muestra un ejemplo del archivo \texttt{topics.json} con un tema y un ejercicio:

\begin{figure}[h!]
\begin{lstlisting}
{
  "topics": [
    {
        "title": "Ciclos",
        "description": "...",
        "exercises": [
            {
                "title": "Es primo",
                "content": "...",
                "difficulty": "Basic",
                "solution": "...",
                "hints": [
                    {
                        "order": 1,
                        "content": "Conocemos que un número primo es un número entero mayor que 1 que solo es divisible por 1 y por sí mismo. ¿Cómo puedes verificar si un número es divisible por otro?"
                    },
                    {
                        "order": 2,
                        "content": "Sea n un número entero, todos sus divisores son menores o iguales que n. ¿Cómo puedes verificar la divisibilidad desde 2 hasta n - 1?"
                    },
                    {
                        "order": 3,
                        "content": "Si el número es 2, es primo. Si es menor que 2 no es primo. Para números n mayores que 2, solo necesitas hacer un bucle desde d = 2 hasta d = n -1 verificando si n es divisible entre d, o sea comprobar que n % d == 0."
                    }
                ]
            }
        ]
    }
  ]
}
\end{lstlisting}
\caption{Ejemplo del formato del JSON para personalización de ejercicios}\label{fig:json}
\end{figure}

Para añadir más documentos al sistema de preguntas y respuestas, es necesario colocar los archivos en formato PDF dentro de la carpeta \texttt{data/corpus}. Cuando se ejecute, la aplicación procesará estos documentos.

En este capítulo se han descrito los detalles de implementación del bot de Telegram. Las funcionalidades implementadas, como el registro de usuarios, la resolución de dudas y la práctica con ejercicios, están diseñadas para mejorar la experiencia de aprendizaje de los estudiantes. Además, la posibilidad de personalizar el contenido del curso asegura que el sistema pueda adaptarse a las necesidades de diferentes asignaturas y profesores.

\backmatter

\begin{conclusions}
En este trabajo se diseñó un sistema para apoyar el estudio independiente de los estudiantes en el curso de Programación de la carrera de Ciencia de la Computación en MATCOM. El sistema, implementado como un bot de Telegram, busca integrarse en el entorno académico, aprovechando la familiaridad de los estudiantes con esta plataforma. Su objetivo principal es que los estudiantes no se sientan solos durante su estudio independiente, sino que cuenten con un tutor virtual accesible todo el tiempo.

El sistema está diseñado para determinar el nivel de conocimiento de cada estudiante en función de los ejercicios que ha resuelto, y, en base a ello, sugerir ejercicios adaptados a sus necesidades. Esta personalización tiene como objetivo fomentar un aprendizaje progresivo, permitiendo a los estudiantes avanzar a su propio ritmo y evitando la frustración que puede surgir al enfrentar problemas demasiado complejos o simples. Además, el registro del estado de las actividades proporciona una visión del progreso de los estudiantes, lo que permite a los profesores adaptar el curso según las necesidades detectadas; la información recuperada sobre el desempeño individual y grupal facilita la identificación de áreas que requieren mayor atención en clases.

Asimismo, se desarrollaron e incluyeron en el sistema ejercicios con pistas escalonadas y soluciones detalladas, con el objetivo de guiar a los estudiantes en su proceso de aprendizaje, ofreciendo ayuda gradual y evitando la frustración.

Se implementó un sistema de Generación Aumentada con Recuperación (RAG, por sus siglas en inglés) para resolver dudas, asegurando que las respuestas estén basadas en la bibliografía oficial del curso. Esta funcionalidad busca reducir la dependencia de fuentes externas no verificadas, un problema frecuente entre estudiantes inexpertos que puede generar confusión y desmotivación.

Este trabajo no solo ofrece una solución tecnológica, sino que también propone una metodología para mejorar la experiencia de aprendizaje en el curso de Programación. La integración del sistema en MATCOM establece las bases para futuras mejoras y ampliaciones, con el objetivo de continuar apoyando a estudiantes y profesores en el proceso de enseñanza-aprendizaje.
\end{conclusions}

\begin{recomendations}
    A partir del desarrollo del sistema, se derivan las siguientes recomendaciones:

    \begin{itemize}
        \item Evaluar la efectividad del sistema en el curso de Programación de Ciencia de la Computación en \mbox{MATCOM}, utilizando métricas cuantitativas y cualitativas para medir su impacto en el aprendizaje.
        
        \item Se recomienda agregar nuevas funcionalidades al sistema y al bot, basadas en la retroalimentación de los usuarios durante su utilización.
        
        \item Almacenar las preguntas y respuestas generadas en una base de datos vectorial, lo que permitirá reducir las llamadas innecesarias al LLM.
    
        \item Agrupar y analizar las dudas más frecuentes de los estudiantes con el objetivo de proporcionar datos estadísticos útiles para identificar temas críticos y mejorar las conferencias en los próximos cursos.
        
        \item Implementar el sistema de evaluación automática propuesto en el \Cref{chapter:implementation}.
        
        \item Incorporar un flujo para profesores en el bot incluyendo un nuevo tipo de usuario, esto simplificará las tareas del profesor y eliminará la dependencia de \textit{scripts}.
        
        \item Ajustar estas ideas para utilizarlas en otras asignaturas en las que se detecten situaciones similares.
    \end{itemize}
\end{recomendations}

\include{BackMatter/Bibliography}

\end{document}