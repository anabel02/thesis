\documentclass[12pt,oneside]{uhthesis}
\usepackage{subfigure}
\usepackage[ruled,lined,linesnumbered,titlenumbered,algochapter,spanish,onelanguage]{algorithm2e}
\usepackage{amsmath}
\usepackage{amssymb}
\usepackage{amsbsy}
\usepackage{caption,booktabs}
\captionsetup{ justification = centering }
%\usepackage{mathpazo}
\usepackage{float}
\setlength{\marginparwidth}{2cm}
\usepackage{todonotes}
\usepackage{listings}
\usepackage{xcolor}
\usepackage{multicol}
\usepackage{graphicx}
\floatstyle{plaintop}
\restylefloat{table}
\addbibresource{Bibliography.bib}
% \setlength{\parskip}{\baselineskip}%
\renewcommand{\tablename}{Tabla}
\renewcommand{\listalgorithmcfname}{Índice de Algoritmos}
%\dontprintsemicolon
\SetAlgoNoEnd

\definecolor{codegreen}{rgb}{0,0.6,0}
\definecolor{codegray}{rgb}{0.5,0.5,0.5}
\definecolor{codepurple}{rgb}{0.58,0,0.82}
\definecolor{backcolour}{rgb}{0.95,0.95,0.92}

\lstdefinestyle{mystyle}{
    backgroundcolor=\color{backcolour},   
    commentstyle=\color{codegreen},
    keywordstyle=\color{purple},
    numberstyle=\tiny\color{codegray},
    stringstyle=\color{codepurple},
    basicstyle=\ttfamily\footnotesize,
    breakatwhitespace=false,         
    breaklines=true,                 
    captionpos=b,                    
    keepspaces=true,                 
    numbers=left,                    
    numbersep=5pt,                  
    showspaces=false,                
    showstringspaces=false,
    showtabs=false,                  
    tabsize=4
}

\lstset{style=mystyle}

\title{Título de la tesis}
\author{\\\vspace{0.25cm}Nombre del autor}
\advisor{\\\vspace{0.25cm}Nombre del primer tutor\\\vspace{0.2cm}Nombre del segundo tutor}
\degree{Licenciado en (Matemática o Ciencia de la Computación)}
\faculty{Facultad de Matemática y Computación}
\date{Fecha\\\vspace{0.25cm}\href{https://github.com/username/repo}{github.com/username/repo}}
\logo{Graphics/uhlogo}
\makenomenclature

\renewcommand{\vec}[1]{\boldsymbol{#1}}
\newcommand{\diff}[1]{\ensuremath{\mathrm{d}#1}}
\newcommand{\me}[1]{\mathrm{e}^{#1}}
\newcommand{\pf}{\mathfrak{p}}
\newcommand{\qf}{\mathfrak{q}}
%\newcommand{\kf}{\mathfrak{k}}
\newcommand{\kt}{\mathtt{k}}
\newcommand{\mf}{\mathfrak{m}}
\newcommand{\hf}{\mathfrak{h}}
\newcommand{\fac}{\mathrm{fac}}
\newcommand{\maxx}[1]{\max\left\{ #1 \right\} }
\newcommand{\minn}[1]{\min\left\{ #1 \right\} }
\newcommand{\lldpcf}{1.25}
\newcommand{\nnorm}[1]{\left\lvert #1 \right\rvert }
\renewcommand{\lstlistingname}{Ejemplo de código}
\renewcommand{\lstlistlistingname}{Ejemplos de código}

\begin{document}

\frontmatter
\maketitle

\begin{dedication}
    Dedicación
\end{dedication}
\begin{acknowledgements}
    Agradecimientos
\end{acknowledgements}
\begin{opinion}
    La programación es vital para el desarrollo de la computación, para el mundo moderno y para los estudiantes de nuestras carreras.  Por otra parte, la enseñanza de la programación ha recibido muchísima atención de parte de los investigadores de todo el mundo, como se puede apreciar en el capítulo de los preliminares de esta tesis.  

    En este trabajo se propone una metodología para asistir a los estudiantes de MATCOM con el estudio independiente de esta habilidad.  También se propone una herramienta computacional que implementa la propuesta realizada.   Este trabajo está dirigido específicamente a los estudiantes de nuestras carreras, y toma en consideración las características del curso que se imparte, y la bibliografía básica del mismo.

    Para realizar este trabajo, Anabel tuvo que estudiar contenidos que no forman parte de su plan de estudio, y lo hizo de una manera excelente.  

    Creo que estamos en presencia de un trabajo excelente desarrollado por una excelente científica de la computación y una excelente persona.

    \vspace{1cm}


    \begin{flushright}
    \underline{\hspace{6.5cm}}\\
    MSc. Fernando Raul Rodriguez Flores

    Facultad de Matemática y Computación
    
    Universidad de la Habana

    Febrero, 2025
    \end{flushright}
\end{opinion}
\begin{resumen}
	La programación es una habilidad esencial en el mundo moderno. En el ámbito académico, los cursos introductorios de programación son fundamentales para la formación de profesionales en Ciencia de la Computación, ya que proporcionan las bases para el desarrollo de habilidades avanzadas. Sin embargo, el estudio independiente en estos cursos enfrenta desafíos, como la dificultad de los estudiantes para gestionar el tiempo, la acumulación de conceptos no comprendidos y la falta de orientación sobre por dónde comenzar, lo que puede generar frustración y desmotivación.

	Este trabajo propone el diseño e implementación de un sistema de apoyo al estudio independiente para la asignatura de Programación en la carrera de Ciencia de la Computación de la Facultad de Matemática y Computación de la Universidad de La Habana (\mbox{MATCOM}). El sistema aborda problemas identificados, como la dependencia excesiva del apoyo docente y la falta de retroalimentación oportuna. Incluye funcionalidades como un sistema de preguntas y respuestas basado en Generación Aumentada con Recuperación (RAG), que permite a los estudiantes resolver dudas en tiempo real, y un mecanismo de propuesta de ejercicios adaptados al nivel de conocimiento del estudiante. Además, ofrece pistas durante la resolución de problemas y soluciones detalladas una vez completados, con el objetivo de fomentar el pensamiento crítico y la autonomía.

	El sistema se implementa como un bot en Telegram, una plataforma ampliamente utilizada por los estudiantes de \mbox{MATCOM}. Esta herramienta busca mejorar el rendimiento académico y promover habilidades como la autogestión del aprendizaje y la resolución de problemas. Este trabajo ofrece una solución adaptada a las necesidades específicas de los estudiantes de programación, con potencial para ser replicada en otros contextos académicos.
\end{resumen}

\begin{abstract}
	Programming is an essential skill in the modern world. Introductory programming courses are fundamental for the training of professionals in Computer Science, as they provide the foundation for developing advanced skills. However, independent study in these courses faces challenges, such as students' difficulty in managing time, the accumulation of misunderstood concepts, and the lack of guidance on where to begin, which can lead to frustration and demotivation.

	This work proposes the design and implementation of a system to support independent study for the Programming course in the Computer Science program at the Faculty of Mathematics and Computing of the University of Havana (\mbox{MATCOM}). The system addresses identified issues, such as excessive reliance on instructor support and the lack of timely feedback. It includes functionalities such as a question-and-answer system based on Retrieval-Augmented Generation (RAG), which allows students to resolve doubts in real time, and a mechanism for proposing exercises tailored to the student's knowledge level. Additionally, it provides hints during problem-solving and detailed solutions upon completion, aiming to foster critical thinking and autonomy.

	The system is implemented as a bot on Telegram, a platform widely used by \mbox{MATCOM} students. This tool seeks to improve academic performance and promote skills such as self-regulated learning and problem-solving. This work offers a solution tailored to the specific needs of programming students, with the potential to be replicated in other academic contexts.
\end{abstract}
\tableofcontents
\listoffigures
% \listoftables
% \listofalgorithms
\lstlistoflistings

\mainmatter

\chapter*{Introducción}\label{chapter:introduction}
\addcontentsline{toc}{chapter}{Introducción}

La programación es una habilidad esencial en el mundo actual, con aplicaciones que van más allá del desarrollo tecnológico. Su importancia radica en su capacidad para resolver problemas en diversas áreas del conocimiento, desde la ciencia y la ingeniería hasta las humanidades y las ciencias sociales. En un entorno cada vez más digitalizado, la programación permite comprender y transformar procesos, sistemas y datos, lo que la convierte en una herramienta clave para el progreso y la innovación.

En el ámbito académico, la programación es un componente fundamental en la formación de profesionales en Ciencia de la Computación. Los cursos introductorios de programación son el primer contacto de los estudiantes con los conceptos y herramientas básicas, por lo que deben proporcionar una base sólida para el desarrollo de habilidades más avanzadas en este campo.

El estudio independiente en cursos introductorios de programación enfrenta desafíos que pueden afectar el rendimiento académico de los estudiantes. Según~\cite{proskuraLytvynova2020}, la falta de supervisión constante por parte del docente puede llevar a confusiones en la priorización de contenidos y en la gestión del tiempo, especialmente cuando los estudiantes deben equilibrar el aprendizaje de la programación con otras responsabilidades académicas. Además,~\cite{overklift2019} señala que la ausencia de un docente disponible para resolver dudas durante el estudio puede resultar en la acumulación de conceptos no comprendidos, generando frustración y desmotivación.

Varios estudios han abordado problemas relacionados con la enseñanza y el aprendizaje de la programación. Por ejemplo,~\cite{Gabbay2022, Hanafi2023, Messer2024} estudia el uso de plataformas de autoevaluación que permiten a los estudiantes recibir retroalimentación inmediata sobre sus códigos, lo que ha demostrado ser efectivo para el rendimiento del estudiante.

Por otro lado,~\cite{dong2025buildaitutoradapt} sugiere la implementación de sistemas de tutoría virtual que integren chatbots inteligentes para resolver dudas en tiempo real. Estos sistemas, basados en modelos de lenguaje de gran tamaño (LLMs), como \textit{ChatGPT}~\cite{chatgpt}, proporcionan respuestas adaptándose a las necesidades individuales de los estudiantes.

En la asignatura Programación de la carrera Ciencia de la Computación en la Facultad de Matemática y Computación de la Universidad de La Habana (\mbox{MATCOM}), los estudiantes enfrentan desafíos en el proceso de aprendizaje. Los profesores han observado que existe una dependencia significativa del apoyo docente para avanzar en la disciplina.

El objetivo general de este trabajo es diseñar e implementar un sistema que apoye a los estudiantes de Ciencia de la Computación de (\mbox{MATCOM}) en el estudio independiente de la asignatura Programación.

Los objetivos específicos de esta investigación son:
\begin{itemize}
    \item Realizar una revisión de la bibliografía existente sobre sistemas de apoyo al aprendizaje de programación y metodologías de estudio independiente.
    \item Identificar las necesidades específicas de los estudiantes de Matcom en relación con el estudio independiente de programación.
    \item Recopilar ejercicios de Programación organizados por temas y elaborar soluciones detalladas que expliquen los procedimientos a los estudiantes.
    \item Diseñar un sistema que incluya funcionalidades como proporcionar retroalimentación inmediata sobre los ejercicios, resolver dudas comunes de los estudiantes y guiar en la resolución de problemas.
    \item Implementar un prototipo del sistema a través de un bot de Telegram.
    \item Proponer recomendaciones para mejorar el sistema.
\end{itemize}

Este trabajo propone un sistema diseñado específicamente para apoyar el estudio independiente de la asignatura Programación en la carrera Ciencia de la Computación en \mbox{MATCOM}. A diferencia de las soluciones existentes, este sistema se enfoca en integrar recursos adaptados al plan de estudios de la facultad con mecanismos de retroalimentación personalizada y seguimiento del progreso del estudiante. Además, aborda problemas identificados en el proceso de estudio independiente, como la falta de retroalimentación oportuna y la dependencia excesiva del apoyo docente. El sistema busca no solo facilitar el aprendizaje de la programación, sino también motivar a los estudiantes a través de una herramienta que les permita avanzar a su propio ritmo y recibir apoyo constante.

Además, dado que la mayoría de los estudiantes de la facultad están familiarizados con el uso de Telegram, se realiza una implementación de este sistema como un bot en dicha plataforma.

La estructura de este documento es la siguiente: en el capítulo~\ref{chapter:background} se presentan algunos de los fundamentos de la enseñanza de programación, se analiza la importancia del estudio independiente en el aprendizaje de la programación y se describe el curso de Programación que se imparte en primer año de Ciencia de la Computación en \mbox{MATCOM}. En el capítulo~\ref{chapter:exercises} se describe el proceso de recopilación y diseño de un conjunto de ejercicios, así como el enfoque adoptado para presentar sus soluciones y pistas que guíen al estudiante durante el proceso de solución. En el capítulo~\ref{chapter:proposal} se presenta una propuesta del sistema detallando el flujo de cada funcionalidad. En el capítulo~\ref{chapter:implementation} se exponen los detalles de implementación del bot de Telegram.
\chapter{Preliminares}\label{chapter:background}

El objetivo de este trabajo es diseñar e implementar una herramienta que apoye el estudio independiente de los estudiantes de Programación. En este capítulo se presentan los elementos fundamentales que sustentan la investigación.

En la sección~\ref{sec:ensenanza_programacion} se presentan algunos de los fundamentos de la enseñanza de programación. En la sección~\ref{sec:contents}, se enumeran los contenidos que suelen incluir los cursos básicos de programación a nivel global. En la sección~\ref{sec:resources} se analizan los materiales y recursos empleados en la enseñanza de programación, desde los tradicionales libros de texto hasta aplicaciones interactivas.

En la sección~\ref{sec:study} se analiza el estudio independiente y su importancia en el aprendizaje de la programación. En la sección~\ref{sec:matcom} se describe el curso de Programación que se imparte en primer año de Ciencia de la Computación en la Facultad de Matemática y Computación de la Universidad de La Habana (\mbox{MATCOM}).

\section{Elementos sobre la enseñanza de la programación}\label{sec:ensenanza_programacion}

El objetivo de los cursos introductorios de programación en Ciencia de la Computación es que los estudiantes no solo aprendan a escribir código, sino que también adquieran una mentalidad analítica que les permita abordar problemas de manera estructurada y lógica~\cite{JOHNLEMAY2021100056}. Estos cursos combinan teoría y práctica~\cite{Sarsa_2022}, o sea, los estudiantes aprenden los conceptos aplicándolos directamente en ejercicios y proyectos.

Estos cursos y sus tareas suelen diseñarse introduciendo conceptos y actividades de forma gradual, aumentando su nivel de dificultad de manera escalonada para evitar que los estudiantes se sientan abrumados al recibir más contenido del que pueden procesar, previniendo así la sobrecarga cognitiva~\cite{duran2021clt}.

La carga cognitiva se refiere a la intensidad de la actividad mental necesaria para alcanzar un objetivo de aprendizaje en un tiempo limitado~\cite{duran2021clt}. La sobrecarga cognitiva ocurre cuando esta carga supera la capacidad del estudiante para procesar y asimilar la información, dificultando el aprendizaje y comprometiendo los resultados~\cite{duran2021clt}. 

Este proceso de introducir conceptos gradualmente permite que los estudiantes construyan una base sobre la que puedan abordar desafíos más difíciles, y ayuda a los estudiantes a desarrollar un sentido de logro a medida que avanzan. Este proceso se relaciona con la teoría de la zona de desarrollo próximo~\cite{vygotsky1978mind}, según la cual, los estudiantes pueden adquirir habilidades que aún no son capaces de dominar de forma independiente, siempre y cuando reciban la orientación adecuada. A medida que los estudiantes progresan, su zona de desarrollo próximo se expande, permitiéndoles superar nuevos obstáculos, mejorar sus habilidades de programación y abordar problemas cada vez más complejos.

De acuerdo con~\cite{rosenzweig2019expectancy} la motivación de los estudiantes está estrechamente ligada al nivel de dificultad de las tareas que se les orientan. Ejercicios demasiado simples pueden llevar al desinterés, mientras que aquellos que superan en complejidad las habilidades actuales de los estudiantes tienden a generar frustración y desaliento~\cite{rosenzweig2019expectancy}. Este desequilibrio no solo afecta la disposición de los estudiantes para continuar con el aprendizaje de la programación, sino que también puede debilitar su percepción de autoeficacia~\cite{bandura1977self}.

Para evitar que los estudiantes enfrenten tareas demasiado complejas de forma prematura, los cursos introductorios de programación suelen incluir numerosos ejercicios pequeños que promueven la adquisición de habilidades mediante la práctica regular~\cite{vihavainen2011extreme}. Dado que los estudiantes tienen diferentes antecedentes, habilidades y zonas de desarrollo próximo, un enfoque personalizado con tareas adaptadas a sus necesidades podría ser beneficioso~\cite{leinonen2021exploring}. Sin embargo, crear conjuntos de ejercicios personalizados representa una tarea extremadamente laboriosa~\cite{lobb2016coderunner, wrenn2018whotests}.

El diseño de los cursos de programación usualmente se basa en la práctica deliberada~\cite{ericsson1993deliberate}, un enfoque que atribuye el desarrollo de la experticia a la práctica sistemática, prolongada en el tiempo y enfocada en mejorar habilidades específicas. La práctica deliberada enfatiza que el desempeño sobresaliente depende menos del talento innato y más del esfuerzo y perseverancia~\cite{ericsson1993deliberate}.

Para comprender mejor cómo se estructuran estos cursos, en la siguiente sección se analizan los contenidos que se imparten en los cursos introductorios de programación.

\section{¿Qué se enseña en un curso introductorio de programación?}\label{sec:contents}

Los cursos introductorios de programación están diseñados para estudiantes sin experiencia previa en la materia. Generalmente, cubren los fundamentos de la programación, el desarrollo de habilidades para la resolución de problemas y el uso de un lenguaje de programación para implementar soluciones~\cite{Medeiros2019}.

Algunos cursos también incorporan conceptos más avanzados, como la recursividad, que permite resolver problemas complejos mediante su descomposición en subproblemas más simples~\cite{cs50_harvard,mit_60001}. Además, en algunos cursos se introducen principios de Programación Orientada a Objetos, incluyendo clases, objetos, herencia y polimorfismo, lo que facilita la construcción de software modular y escalable~\cite{cs50_harvard, mit_60001}.

Los instructores de cursos introductorios de programación a nivel universitario trabajan constantemente en mejorar sus planes de enseñanza y en seleccionar recursos de alta calidad para ayudar a los estudiantes a alcanzar los objetivos de aprendizaje previstos~\cite{wong2022}. En la próxima sección, se mencionan los diferentes recursos que se emplean para enseñar programación.

\section{Materiales utilizados para aprender a programar}\label{sec:resources}

El aprendizaje de la programación se apoya en una variedad de materiales diseñados para complementar las clases, fomentar el estudio independiente y facilitar la comprensión de conceptos complejos. Algunos de estos materiales son los libros de texto, presentaciones y conjuntos de ejercicios que desarrollan los docentes para los estudiantes. Además, existen recursos como plataformas en línea, videos en \textit{Youtube} y herramientas como \textit{ChatGPT}.

Los libros de texto constituyen la bibliografía básica en los cursos de programación~\cite{wong2022}. Estos libros sirven como referencia para estudiantes y docentes, ofreciendo teoría y ejemplos prácticos.

Para apoyar las clases, los docentes desarrollan presentaciones como herramientas visuales que facilitan la enseñanza de conceptos. Estas permiten ilustrar ideas, presentarlas de manera estructurada e interactuar con los estudiantes durante las explicaciones.

Como se menciona en la sección~\ref{sec:ensenanza_programacion}, la práctica es un elemento clave en los cursos de programación. Los conjuntos de ejercicios, como parte de los materiales utilizados en estos cursos, constituyen una de las principales formas de practicar. Diseñados para las sesiones de laboratorio, estos ejercicios permiten a los estudiantes aplicar los conceptos aprendidos y desarrollar habilidades en resolución de problemas.

Los recursos anteriores forman parte de los materiales oficiales de los cursos, aunque los estudiantes también suelen buscar apoyo en otras fuentes. A continuación, se enumeran algunas de estas fuentes:

\begin{itemize}
    \item Plataformas de ejercicios: 
    Plataformas como \textit{HackerRank}~\cite{hackerrank}, \textit{LeetCode}~\cite{leetcode}  y \textit{GeeksforGeeks}~\cite{geeksforgeeks} ofrecen una amplia variedad de problemas categorizados por nivel de dificultad y temática. Estos ejercicios pueden ayudar a los estudiantes a consolidar sus conocimientos y mejorar sus habilidades de resolución de problemas.

    \item Videotutoriales:
    Los videotutoriales son una herramienta educativa versátil y accesible que combina explicaciones visuales y prácticas. Están disponibles en plataformas como \textit{YouTube}~\cite{youtube}, y suelen abordar temas desde lo más básico hasta lo avanzado. Estos videos permiten a los estudiantes aprender a su propio ritmo, pausando y repitiendo secciones según sea necesario.

    \item Foros y comunidades: 
    Foros como \textit{Stack Overflow}~\cite{stackoverflow} y comunidades en redes sociales como \textit{Reddit}~\cite{reddit} permiten a los estudiantes interactuar directamente con expertos y compañeros. Aquí pueden resolver dudas, compartir conocimientos y aprender de los desafíos que otros enfrentan.
    
    \item Documentación oficial:
    La documentación oficial de los lenguajes, como  \textit{Java API}~\cite{java-api}, \textit{Python.org}~\cite{python} y \textit{Microsoft Docs}~\cite{microsoft-docs}, proporciona una referencia confiable y mantenida por los propios desarrolladores del lenguaje. Estos recursos incluyen ejemplos y buenas prácticas.
    
    \item Chatbots: 
    Entre los recursos que complementan el proceso de aprendizaje, los chatbots basados en modelos de lenguaje de gran tamaño (LLMs), como \textit{ChatGPT}~\cite{chatgpt} y \textit{Microsoft Copilot}~\cite{copilot}, destacan por su capacidad de ofrecer soporte personalizado y en tiempo real.
\end{itemize}

A pesar de la disponibilidad de múltiples recursos, la programación sigue enfrentando un alto índice de abandono de los cursos. Según~\cite{wong2022}, esto se debe en gran medida a que los estudiantes priorizan la búsqueda de soluciones inmediatas sobre la comprensión de los conceptos fundamentales. Muchos recurren a un ciclo de prueba y error, modificando el código repetidamente hasta que funciona, sin entender el razonamiento detrás de los cambios. Este enfoque, apoyado en foros de preguntas y respuestas, no solo consume tiempo, sino que también impide un aprendizaje profundo y significativo.

Otro factor que contribuye a este problema es la percepción de un ``alto costo social'' al pedir ayuda~\cite{wong2022}. Los estudiantes suelen evitar consultar a instructores o compañeros por temor a parecer poco competentes o dependientes. Esta ansiedad, generada por la presión de demostrar autonomía o por experiencias previas negativas, los lleva a depender excesivamente de recursos en línea. Sin embargo, esta dependencia rara vez resuelve sus dudas de fondo, perpetuando la frustración y el estancamiento en su progreso.

En la siguiente sección, se analizan algunos aspectos del estudio independiente de la programación, con el objetivo de identificar estrategias que permitan superar estos obstáculos y fomentar un aprendizaje más efectivo.

\section{Estudio independiente de la programación}\label{sec:study}

El estudio independiente es el proceso en el que los estudiantes gestionan su propio aprendizaje, generalmente siguiendo las instrucciones de un docente, pero sin su supervisión constante, aprovechando los recursos disponibles para desarrollar conocimientos y habilidades fuera del aula~\cite{proskuraLytvynova2020}.

Según~\cite{proskuraLytvynova2020}, el estudio independiente es clave en la formación de estudiantes en Ciencia de la Computación, quienes deben asumir el aprendizaje autónomo como base para su desarrollo académico y profesional. Esta práctica cobra importancia ante la evolución tecnológica, que exige a los profesionales en programación mantener un aprendizaje constante~\cite{proskuraLytvynova2020}.

El éxito en el estudio independiente depende en gran medida de la capacidad de los estudiantes para gestionar su tiempo y organizar sus actividades de estudio~\cite{overklift2019}. La falta de orientación directa por parte del personal docente puede llevar a confusión sobre cómo priorizar contenidos y equilibrar la programación con otras asignaturas o responsabilidades~\cite{proskuraLytvynova2020}. 

Sin un instructor disponible, los estudiantes pueden encontrarse con conceptos o problemas complejos que no logran resolver por sí mismos, lo que puede generar frustración y afectar su confianza~\cite{overklift2019}. Un mal estudio independiente puede tener consecuencias significativas en el rendimiento académico de los estudiantes. Según~\cite{overklift2019}, aquellos que enfrentan dificultades para organizar y estructurar su tiempo de estudio suelen experimentar problemas para cumplir con los requisitos de sus cursos.

De acuerdo con~\cite{overklift2019} cuando los estudiantes reconocen problemas en su proceso de aprendizaje autónomo, suelen buscar apoyo en el personal docente; sin embargo, los estudiantes a menudo tardan en identificar que necesitan ayuda, y la disponibilidad del docente puede ser insuficiente para brindar un apoyo personalizado a todos los alumnos.

En este sentido, la retroalimentación proporcionada durante el proceso de aprendizaje puede resultar fundamental~\cite{Sarsa_2022}. La retroalimentación fomenta habilidades como el aprendizaje autorregulado~\cite{Sarsa_2022}, que se refiere a la capacidad de los estudiantes para planificar, supervisar y evaluar su propio proceso de aprendizaje, estableciendo metas, seleccionando estrategias y ajustando su comportamiento según los resultados obtenidos

La retroalimentación también promueve la metacognición~\cite{Sarsa_2022}, es decir, la conciencia y comprensión que tienen los estudiantes de sus propios procesos de pensamiento. Estas habilidades permiten a los estudiantes reflexionar sobre sus métodos, identificar áreas de mejora y ajustar sus estrategias de aprendizaje. La retroalimentación ayuda a los estudiantes no solo a resolver problemas específicos, sino también a desarrollar la capacidad de aprender de manera autónoma, convirtiéndose en aprendices más efectivos y adaptables~\cite{shute2008focus}.

El estudio independiente, como se ha mencionado, representa una herramienta esencial para el crecimiento académico y profesional, particularmente en disciplinas como la Ciencia de la Computación. Sin embargo, su éxito depende en gran medida de un marco educativo bien definido que proporcione dirección y recursos adecuados. En la siguiente sección, se detalla el curso de Programación que se imparte en la carrera de Ciencia de la Computación en \mbox{MATCOM}.

\section{Curso de Programación en \mbox{MATCOM}}\label{sec:matcom}

El curso de Programación forma parte del primer año de la carrera de Ciencia de la Computación de la Universidad de La Habana. Programación ocupa un lugar central en el Plan de Estudio debido a su importancia en la formación de las capacidades y habilidades básicas en el perfil profesional de Computación~\cite{plan_estudio_e_2017}.

La programación es una habilidad fundamental en la formación de los estudiantes de Ciencia de la Computación, ya que les permite transformar ideas abstractas en soluciones prácticas. Además, fomenta el desarrollo de habilidades críticas como la resolución de problemas, la creatividad y la capacidad de trabajar en equipo~\cite{plan_estudio_e_2017}.

Las siguientes secciones presentan la estructura del curso de Programación, los recursos empleados y los contenidos impartidos a lo largo del mismo.

\subsection{Estructura del curso}

El curso está orientado a estudiantes sin experiencia previa en programación. Se organiza en dos semestres, cada uno con una duración de 16 semanas. En el Cuadro~\ref{tab:course_distribution} se presenta la distribución temporal de los contenidos, que ofrece una visión general del curso y de las evaluaciones.

\begin{table}[h!]
    \centering
    \begin{tabular}{|p{6.5cm}|p{2cm}|p{3.5cm}|}
    \hline
    \textbf{Temas} & \textbf{Semanas} & \textbf{Evaluaciones} \\ \hline
    \raggedright Fundamentos de Programación. \\ Programación Estructurada. 
    & 8 & Prueba parcial y Proyecto extraclase \\ \hline
    \raggedright Recursividad.
    & 8 & Prueba parcial \\ \hline
    \raggedright Programación Orientada a Objetos. \\ Programación Funcional. \\ Estructuras de Datos. 
    & 16 & Prueba parcial y Proyecto extraclase \\ \hline
    \end{tabular}
    \caption{Distribución del curso}\label{tab:course_distribution}
\end{table}
    

El curso combina teoría y práctica, con una conferencia semanal para la presentación de conceptos y dos clases prácticas que permiten a los estudiantes aplicar y consolidar lo aprendido en un entorno guiado. A continuación se describen cada una de estas sesiones.

\subsubsection{Conferencias}

Las conferencias constituyen el espacio destinado a la introducción de contenidos teóricos y la demostración práctica de su aplicación mediante códigos de ejemplo. Estas sesiones tienen como propósito principal establecer las bases conceptuales, al tiempo que muestran el uso adecuado del lenguaje de programación y del entorno de desarrollo integrado (IDE) seleccionado para el curso~\cite{plan_estudio_e_2017}.

Durante las conferencias, se utiliza un proyector para facilitar la presentación visual de conceptos, fragmentos de código y su ejecución en tiempo real. Este enfoque permite que los estudiantes observen cómo los conceptos se traducen en implementaciones prácticas y cómo interactúan las diferentes partes de un programa. Asimismo, estas sesiones fomentan la participación activa mediante preguntas y discusiones que consolidan el entendimiento de los contenidos.

\subsubsection{Clases Prácticas}

Las clases prácticas tienen como objetivo consolidar los conceptos presentados en las conferencias mediante la resolución de problemas y la implementación de programas~\cite{plan_estudio_e_2017}.

Cada semana, la primera clase práctica se realiza en el aula y está enfocada en vincular los conceptos teóricos y algoritmos presentados en la conferencia con su aplicación práctica. Esto se logra mediante ejercicios guiados, discusiones de problemas y la exploración de diversas estrategias de solución. Los estudiantes analizan problemas, debaten algoritmos y comparten enfoques para resolverlos, mientras el profesor supervisa y evalúa los intentos de solución, ofreciendo retroalimentación para consolidar el aprendizaje.

La segunda clase práctica, realizada en el laboratorio, está orientada a la implementación y los estudiantes trabajan en computadoras. Durante esta sesión, los estudiantes escriben, depuran y prueban código en tiempo real, enfrentándose a ejercicios que incluyen desde la creación de programas elementales hasta la resolución de problemas complejos. Además, en esta clase, los docentes proporcionan orientación técnica, ayudando a los estudiantes a resolver dudas sobre sintaxis, lógica, depuración, etc. 

A través de esta clase, se busca que los estudiantes desarrollen confianza en sus habilidades de programación y aprendan a aplicar conceptos de forma práctica, en un entorno de trabajo supervisado.

Las conferencias y clases prácticas, aunque fundamentales para el aprendizaje, no son los únicos recursos disponibles para los estudiantes. Además de estas actividades guiadas, el curso cuenta con una serie de recursos de enseñanza que enriquecen el proceso de aprendizaje. En la siguiente sección se mencionan estos recursos.

\subsection{Recursos de enseñanza}

El curso utiliza el lenguaje de programación C\# como vehículo para la enseñanza de los conceptos fundamentales de programación. Este lenguaje destaca por su versatilidad, enfoque en la programación orientada a objetos y su relevancia en la industria del desarrollo de software~\cite{microsoft-docs, Albahari2017, Troelsen2021}.

El IDE empleado es Visual Studio Code~\cite{vscode}, una herramienta reconocida por su accesibilidad y flexibilidad. El uso combinado de C\# y Visual Studio Code busca que los estudiantes se familiaricen con herramientas y prácticas estándar de la industria~\cite{microsoft-docs}, preparándolos para proyectos futuros, tanto académicos como profesionales.

El texto principal de la asignatura es el libro ``Empezar a programar. Un enfoque multiparadigma con C\#''~\cite{katrib_programar}. Este libro fue escrito por el profesor Miguel Katrib, quien ha dirigido la disciplina de Programación en la carrera de Ciencia de la Computación en \mbox{MATCOM} durante más de cuatro décadas.

En la siguiente sección se detalla el contenido del curso.

\subsection{Contenidos impartidos en el curso}  

A lo largo del curso se imparten los siguientes contenidos~\cite{plan_estudio_e_2017}:

\begin{enumerate}  
    \item Introducción a la programación y al entorno de desarrollo.  
    Se presentan las nociones básicas sobre cómo funcionan las computadoras, qué son los algoritmos y programas, y cómo los lenguajes de programación facilitan la comunicación con los sistemas. También se explora el entorno de desarrollo que se utilizará durante el curso.  

    \item Estructuras básicas: variables, tipos de datos y operaciones.
    Este tema abarca los fundamentos de la programación, como el uso de variables para almacenar información, los diferentes tipos de datos que se manejan en un programa y las operaciones disponibles para manipularlos.  

    \item Control de flujo: estructuras de control condicionales e iterativas. 
    Tiene como objetivo aprender a dirigir la ejecución del programa utilizando estructuras de decisión (\texttt{if}, \texttt{switch}) y ciclos (\texttt{while}, \texttt{for}).

    \item Métodos y modularización del código.
    Este tema introduce el concepto de dividir un programa en componentes más pequeños y reutilizables, lo que permite un desarrollo más organizado y eficiente.  

    \item Arrays: el concepto primario de colección.
    Se estudia cómo los arrays permiten almacenar y manipular múltiples elementos del mismo tipo. También se abordan las estructuras de control asociadas y el uso de arrays multidimensionales.

    \item Introducción a la complejidad algorítmica.
    Se analizan métodos de búsqueda y ordenación, ilustrando cómo el volumen de datos afecta el desempeño de los programas.  

    \item Recursividad.
    Se estudia la recursividad como técnica de resolución de problemas basada en dividirlos en subproblemas más pequeños.

    \item Estrategias de solución de problemas.
    Se abordan enfoques comunes como fuerza bruta, divide y vencerás, y \textit{backtracking}, con aplicaciones prácticas para resolver problemas complejos.  

    \item Combinatoria.
    Se introduce la combinatoria para resolver problemas relacionados con el conteo y las permutaciones. 

    \item Jerarquías de tipos: herencia y polimorfismo.
    Este tema introduce conceptos avanzados de la programación orientada a objetos, como la reutilización del código mediante herencia, el uso de polimorfismo para trabajar con tipos genéricos y las conversiones entre tipos (casting).  

    \item Clases abstractas. Interfaces. Genericidad.
    Los estudiantes aprenden a diseñar tipos y algoritmos genéricos utilizando interfaces y clases abstracta, además de manejar iteradores para recorrer colecciones de datos.

    \item Elementos de programación funcional.
    Se exploran conceptos básicos de programación funcional, como delegados y funciones como ciudadanos de primera clase, que permiten trabajar con un enfoque más declarativo y modular.

    \item Estructuras de datos.
    Se estudian estructuras clásicas implementadas con arrays y listas enlazadas, analizando el concepto de nodo y enlace como base para estructuras avanzadas.  

    \item Pilas, colas, diccionarios y árboles.
    Este tema profundiza en estructuras de datos y su uso en problemas prácticos.
\end{enumerate}

En este capítulo se ha explorado los fundamentos de los cursos introductorios de programación, los materiales disponibles para estudiar programación, la importancia del estudio independiente en programación, los desafíos que enfrentan los estudiantes y cómo la falta de retroalimentación oportuna y personalizada puede afectar su rendimiento. En particular, en \mbox{MATCOM} se ha observado que la efectividad del estudio independiente de los estudiantes deja mucho que desear, lo que resalta la necesidad de implementar estrategias que complementen el proceso de aprendizaje.

Para abordar los desafíos descritos en este capítulo, se propone un sistema que se integre al curso de Programación de la carrera de Ciencia de la Computación en \mbox{MATCOM}, con el objetivo de  mejorar la experiencia de estudio independiente. Este sistema proporcionará  a los estudiantes respuestas a preguntas sobre el contenido, sugerencias de ejercicios, pistas para resolverlos, y retroalimentación.

En el siguiente capítulo, se aborda la Generación Aumentada con Recuperación, una técnica que combina la recuperación de información con la generación de respuestas de los LLMs, esta técnica resulta conveniente para implementar el componente del sistema destinado a resolver dudas.

\chapter{Propuesta}\label{chapter:proposal}

\chapter{Detalles de implementación}\label{chapter:implementation}

En este capítulo se describen los detalles de implementación de un bot de Telegram diseñado para apoyar el estudio independiente de los estudiantes de programación. En la Sección~\ref{sec:technical-details} se presentan las tecnologías y herramientas utilizadas, así como las razones detrás de su selección. En la Sección~\ref{sec:features} se analiza la arquitectura del sistema y se describen las funcionalidades principales, como el registro de usuarios, el listado de temas, la resolución de dudas y la práctica con ejercicios. Finalmente, se discute cómo los profesores pueden personalizar el contenido del curso.

\section{Detalles técnicos}\label{sec:technical-details}

Se desarrolla un bot de Telegram debido a que los estudiantes de la facultad ya están familiarizados con esta herramienta desde el primer año de la carrera. En \mbox{MATCOM}, Telegram se utiliza ampliamente como medio de comunicación, donde cada asignatura cuenta con al menos un grupo de Telegram para facilitar la interacción entre estudiantes y profesores, algunas asignaturas también disponen de canales. 

Esta adopción generalizada de Telegram elimina la necesidad de que los estudiantes aprendan a manejar una nueva herramienta o incorporen otra aplicación a su rutina, evitando distracciones adicionales. El bot propuesto simplemente se integra como un nuevo chat en una plataforma que los estudiantes ya utilizan con regularidad.

Otra razón importante para esta elección son las ventajas técnicas que ofrece Telegram. Su API es gratuita y está bien documentada, lo que facilita la implementación de funcionalidades. Asimismo, Telegram es accesible desde múltiples dispositivos (dispositivos móviles, web y desktop) e incluso permite sesiones simultáneas, garantizando una disponibilidad constante y comodidad para los estudiantes.

El bot se desarrolla utilizando Python como lenguaje de programación debido a su amplia gama de bibliotecas. Para la gestión de datos, se utilizó PostgreSQL~\cite{postgresql}, un sistema de bases de datos relacional escalable y de código abierto.

En la siguiente sección, se describe el diseño del sistema mediante diagramas que representan las funcionalidades principales, proporcionando una visión de cómo interactúan los componentes del bot.

\section{Arquitectura y funcionalidades del sistema}\label{sec:features}

El esquema de la base de datos está estructurado en varias tablas interrelacionadas \texttt{users}, \texttt{students}, \texttt{topics}, \texttt{exercises}, \texttt{exercise\_hints}, \texttt{student\_hints} y \texttt{student\_exercise} que gestionan la información de usuarios, estudiantes, temas, ejercicios y pistas.

En esta sección se describe el diseño del sistema mediante diagramas que representan las funcionalidades principales. La Figura~\ref{fig:legend} muestra una leyenda que explica los componentes utilizados en los diagramas. Las flechas indican el flujo de la interacción.

\begin{figure}[h!]
  \centering
      \begin{tikzpicture}[node distance=0.5cm and 1cm]
      \node (startstop) [startstop] {Estudiante ingresa\\ nombre y apellidos};
      \node (startstopLabel) [right=of startstop, font=\footnotesize] {Interacción del usuario};
  
      \node (process) [process, below=of startstop] {Mensaje de bienvenida y\\ lista de comandos disponibles};
      \node (processLabel) [right=of process, font=\footnotesize] {Respuesta del sistema};
  
      \node (decision) [decision, below=of process] {¿Quedan \\ ejercicios \\ disponibles?};
      \node (decisionLabel) [right=of decision, font=\footnotesize] {Punto de decisión};
      \end{tikzpicture}
      \caption{Leyenda de los componentes de los diagramas.}\label{fig:legend}
\end{figure}

Las funcionalidades disponibles son el registro de los estudiantes, listar los temas de la asignatura, responder dudas sobre el contenido y recomendar ejercicios, así como ofrecer pistas para guiar en la resolución de los ejercicios. El primer paso es el registro de los estudiantes.

\subsection{Registro}

Al acceder por primera vez, la herramienta solicita al estudiante ingresar su nombre completo. Luego, muestra un mensaje de bienvenida y una guía básica que explica las funcionalidades disponibles, siguiendo el flujo descrito en la Figura~\ref{fig:register}. Esta introducción tiene como objetivo que el usuario comprenda cómo aprovechar la aplicación para su estudio.

\begin{figure}[h!]
  \centering
  \begin{tikzpicture}[node distance=1cm and 1cm]
  \node (registro) [startstop] {/start};
  \node (sDatos) [process, below=of registro]{Solicitud de información personal};
  \node (datos) [startstop, below=of sDatos]{Estudiante ingresa nombre y apellidos};
  \node (bienvenida) [process, right=of datos]{Mensaje de bienvenida y\\ lista de comandos disponibles};

  \draw [arrow2] (registro) -- (sDatos);
  \draw [arrow2] (sDatos) -- (datos);
  \draw [arrow2] (datos) -- (bienvenida);
  \end{tikzpicture}
  \caption{Diagrama del flujo del registro.}\label{fig:register}
\end{figure}

Después del registro, lo más común es que el estudiante revise el listado de temas disponibles para comenzar su estudio.

\subsection{Listado de temas}

Cuando el usuario lo solicita, la aplicación muestra un listado organizado de los temas disponibles para su estudio. Los temas se presentan en el orden definido en el diseño del curso de Programación, que se describe en la Sección~\ref{sec:matcom}.

Si el nombre del tema no resulta suficientemente claro, el estudiante puede solicitar una descripción detallada del mismo. Este flujo se ilustra en la Figura~\ref{fig:topics} de la página~\pageref{fig:topics}.

\begin{figure}[h!]
  \centering
  \begin{tikzpicture}[node distance=1cm and 1cm]
    \node (solicitud) [startstop] {/topics};
    \node (listado) [process, below=of solicitud] {Listado de temas organizados};
    \node (explorar) [decision, below=of listado] {¿El estudiante \\ desea más detalles \\ sobre un tema?};
    \node (descripcionSolicitud) [startstop, right=of explorar]{/topic [título del tema]};
    \node (descripcion) [process, below=of descripcionSolicitud] {Descripción del tema};
    
    \draw [arrow2] (solicitud) -- (listado);
    \draw [arrow2] (listado) -- (explorar);
    \draw [arrow2] (explorar) -- node[above] {Sí} (descripcionSolicitud);
    \draw [arrow2] (descripcionSolicitud) -- (descripcion);
  \end{tikzpicture}
\caption{Diagrama del flujo del listado temas.}\label{fig:topics}
\end{figure}

Durante el estudio o después de una clase, es común que los estudiantes tengan dudas sobre el contenido, como por ejemplo: ``¿qué es un array de direcciones?''. Para resolver estas inquietudes, la aplicación cuenta con un módulo de resolución de dudas.

\subsection{Sistema de resolución de dudas basado en RAG}

La aplicación incluye un sistema de resolución de dudas que permite a los estudiantes formular preguntas en lenguaje natural. Utilizando un RAG, el sistema proporciona respuestas basadas en la bibliografía del curso. La Figura~\ref{fig:q-a-diagram} de la página~\pageref{fig:q-a-diagram} muestra el funcionamiento de este sistema.

\begin{figure}[h!]
  \centering
    \begin{tikzpicture}[
      font=\sffamily,
      every node/.style={align=center},
      box/.style={draw, rectangle, minimum width=3cm, minimum height=1cm},
      arrow/.style={-Latex},
      db/.style={cylinder, cylinder uses custom fill, cylinder body fill=blue!20, shape border rotate=90, aspect=0.25, minimum height=1.5cm, minimum width=1cm, draw}
    ]
    
    \node[box] (materials) {Libro de Katrib};
    \node[box, below=1cm of materials] (question) {/ask [pregunta]};
    \node[box, right=0.8cm of question] (embedding) {Función \\de \\Embedding};
    \node[db, right=2cm of embedding] (database) {Base de Datos \\Vectorial};
    \node[box, below=2.5cm of database] (prompt) {Prompt Extendido = \\Pregunta +\\Contenido del libro};
    \node[box, below=1.5cm of prompt] (llm) {LLM};
    \node[box, left=2.5cm of llm] (reply) {Respuesta \\con referencias al libro};
    \node[below=1.5cm of question] (user) {
      \begin{tikzpicture}[scale=0.8]
        \draw (0,0.7) circle [radius=0.3]; % Cabeza
        % Cuerpo
        \draw (0,0.4) -- (0,-0.5); % Línea del cuerpo
        \draw (-0.4,-1) -- (0,-0.5) -- (0.4,-1); % Piernas
        \draw (-0.3,-0.2) -- (0,0.2) -- (0.3,-0.2); % Brazos
      \end{tikzpicture}
      \\ Estudiante
    };
    
    \draw[arrow] (materials) -- (embedding);
    \draw[arrow] (user) -- (question);
    \draw[arrow] (question) -- (embedding);
    \draw[arrow, bend left=15] (embedding) to node[above] {Guardar} (database);
    \draw[arrow, bend right=15] (embedding) to node[below] {Búsqueda \\por \\similitud} (database);
    
    \draw[arrow] (database) -- node[right] {Fragmentos \\recuperados} (prompt);
    \draw[arrow] (prompt) -- (llm);
    \draw[arrow] (llm) -- (reply);
    \draw[arrow] (reply) -- ++(0,1) -| (user);
    
    \end{tikzpicture}
    \caption{Sistema de preguntas y respuestas basado en Generación Aumentada por Recuperación (RAG).}\label{fig:q-a-diagram}
\end{figure}

Como se explica en el Capítulo~\ref{chapter:rag} el proceso de construcción del \textit{retriever} consta de tres etapas: dividir el corpus en fragmentos, codificar los fragmentos y construir la base de datos vectorial.

El corpus del sistema está compuesto por los capítulos del libro ``Empezar a programar: Un enfoque multiparadigma con C\#'' en formato PDF. Este contenido se convierte a texto utilizando herramientas de extracción que preservan su estructura original. Para manejar el contenido, se divide en fragmentos de longitud fija de 2000 caracteres con un solapamiento de 200 caracteres, con el objetivo de que no se pierda el contexto entre fragmentos consecutivos. Cada fragmento se transforma en una representación numérica mediante el modelo de embeddings \textit{models/text-embedding-004}.

Durante este proceso, a cada fragmento se le asocian metadatos, como el nombre del documento y el número de página correspondiente. Estos metadatos se almacenan junto con los vectores en la base de datos vectorial, lo que permite, durante la fase de generación, incorporar referencias en las respuestas proporcionadas al usuario.

Cuando un estudiante formula una pregunta, esta se procesa mediante el mismo modelo \textit{models/text-embedding-004}, que convierte la pregunta en una representación vectorial. Para medir la relevancia entre una consulta y los fragmentos almacenados en la base de datos vectorial, se emplea la similitud coseno. El sistema utiliza los 5 fragmentos más similares para generar el \textit{prompt} extendido, que combina la pregunta con el contenido relevante del libro. El \textit{prompt} extendido se envía al modelo \textit{gemini-1.5-flash}, que genera una respuesta.

Este flujo de trabajo muestra cómo el sistema RAG combina la búsqueda de información con la generación de texto, con el objetivo de reducir el tiempo que los estudiantes invertirían en buscar respuestas a través de medios tradicionales, como la revisión manual del libro o la consulta directa al profesor. Al incorporar referencias específicas al libro, el sistema no solo proporciona respuestas más confiables, sino que también fomenta la capacidad crítica de los estudiantes, permitiéndoles contrastar la información generada con fuentes autorizadas. De esta manera, se busca minimizar la dependencia exclusiva de LLMs o fuentes en línea que podrían no ser fiables, especialmente para usuarios con poca experiencia.

Además de resolver dudas, la aplicación cuenta con un sistema de sugerencia de ejercicios, así como apoyo en cada paso de la resolución de estos ejercicios.

\subsection{Práctica con ejercicios sugeridos. Pistas y soluciones.}

La aplicación ofrece una experiencia de aprendizaje personalizada al sugerir ejercicios adaptados al nivel del estudiante. Durante la resolución, el estudiante puede solicitar pistas que se dan en orden. En cualquier momento, el estudiante puede acceder a la solución completa. Una vez que el estudiante completa su implementación la manda al bot. Este flujo está descrito en la Figura~\ref{fig:exercises} de la página~\pageref{fig:exercises}.

El nivel del estudiante se calcula en función de su desempeño previo en la aplicación, considerando dos factores principales: la cantidad de ejercicios resueltos y la complejidad de los ejercicios que ha completado. Por ejemplo, si un estudiante resuelve 5 ejercicios básicos sin pedir pistas ni soluciones, la aplicación comenzará a proponer problemas de nivel intermedio.

\begin{figure}[h!]
  \centering
  \begin{tikzpicture}[node distance=1cm and 1cm]

    \node (start) [startstop] {/exercise [título del tema]};
    \node (disponibles) [decision, below=of start] {¿Quedan \\ ejercicios \\ disponibles?};
    \node (sugerencia) [process, below=of disponibles] {Sugerencia de ejercicio};
    \node (completado) [process, right=of disponibles] {Felicidades, Completaste \\ todos los ejercicios \\ de este tema};
    \node (conversion) [decision, below=of sugerencia] {¿El estudiante \\ sabe resolverlo?};
    \node (submit) [startstop, right=of conversion] {/submit [código]};
    \node (pista) [startstop, below=of conversion] {/hint [número del ejercicio]};
    \node (pistasDisponibles) [decision, left=of pista] {¿Quedan \\ pistas \\ disponibles?};
    \node (noPistas) [process, below=of pistasDisponibles] {Lo siento, no quedan \\ pistas para este \\ ejercicio};
    \node (sugerenciaPista) [process, above=of pistasDisponibles] {Sugerencia de pista};
    
    \draw [arrow2] (start) -- (disponibles);
    \draw [arrow2] (disponibles) -- node[above] {No} (completado);
    \draw [arrow2] (disponibles) -- node[left] {Sí} (sugerencia);
    \draw [arrow2] (sugerencia) -- (conversion);
    \draw [arrow2] (conversion) -- node[above] {Sí} (submit);
    \draw [arrow2] (conversion) -- node[left] {No} (pista);
    \draw [arrow2] (pista) -- (pistasDisponibles);
    \draw [arrow2] (pistasDisponibles) -- node[left] {No} (noPistas);
    \draw [arrow2] (pistasDisponibles) -- node[left] {Sí} (sugerenciaPista);
    \draw [arrow2] (sugerenciaPista) --  (conversion);
    
  \end{tikzpicture}
\caption{Diagrama del flujo del sistema de ejercicios.}\label{fig:exercises}
\end{figure}

Para mejorar esta funcionalidad, se propone implementar un sistema de evaluación automática de la correctitud de la solución. Actualmente, cuando un estudiante envía su código, este se almacena en la base de datos, el ejercicio se marca como ``pendiente de revisión'' y queda a la espera de que un profesor lo revise. Si bien este enfoque es funcional, la retroalimentación que proporciona al estudiante puede demorar.

Con la nueva propuesta, el sistema de evaluación automática ejecutará el código del estudiante y comparará la salida generada con los casos de prueba predefinidos. Si la salida coincide con los resultados esperados, el ejercicio se marcará como resuelto. En caso contrario, el sistema proporcionará retroalimentación inmediata al estudiante sobre los errores encontrados. De todas formas, con este sistema de evaluación automática, también se almacenaría el código de las soluciones en la base de datos, dejando al profesor la posibilidad de revisarlos.

A continuación se describe cómo los profesores podrían cambiar el contenido del curso, ya sea los materiales de los que se recupera información para resolver dudas, como los ejercicios.

\subsection{Personalización del curso}

Si se desea cambiar los temas y ejercicios del curso, es necesario crear un archivo \texttt{topics.json} en la carpeta \texttt{data}. Este archivo debe seguir un formato específico para que el sistema pueda interpretarlo correctamente.

El contenido del archivo debe ser un objeto JSON con un campo llamado topics, que es un array de temas. Cada tema debe contener la siguiente información:
\begin{itemize}
    \item \texttt{title} (tipo \texttt{string}): Título del tema.
    \item \texttt{description} (tipo \texttt{string}): Descripción general del tema.
    \item \texttt{exercises} (tipo \texttt{array}): Lista de ejercicios asociados al tema. Cada ejercicio contiene:
    \begin{itemize}
        \item \texttt{title} (tipo \texttt{string}): Título del ejercicio.
        \item \texttt{content} (tipo \texttt{string}): Enunciado o descripción del ejercicio.
        \item \texttt{difficulty} (tipo \texttt{string}): Dificultad del ejercicio: \textit{basic}, \textit{intermediate} o \textit{advanced}.
        \item \texttt{solution} (tipo \texttt{string}): Solución completa del ejercicio.
        \item \texttt{hints} (tipo \texttt{array}): Lista de pistas asociadas al ejercicio, donde cada pista tiene:
        \begin{itemize}
            \item \texttt{content} (tipo \texttt{string}): Texto de la pista.
            \item \texttt{order} (tipo \texttt{number}): Nivel de detalle de la pista, que define el orden en que se entregarán al usuario. Las pistas se muestran de menor a mayor nivel (0, 1, 2, \ldots), se recomienda comenzar con sugerencias generales e ir avanzando hacia indicaciones más específicas.
        \end{itemize}
    \end{itemize}
\end{itemize}

En la Figura~\ref{fig:json} de la página~\pageref{fig:json}, se muestra un ejemplo del archivo \texttt{topics.json} con un tema y un ejercicio:

\begin{figure}[h!]
\begin{lstlisting}
{
  "topics": [
    {
        "title": "Ciclos",
        "description": "...",
        "exercises": [
            {
                "title": "Es primo",
                "content": "...",
                "difficulty": "Basic",
                "solution": "...",
                "hints": [
                    {
                        "order": 1,
                        "content": "Conocemos que un número primo es un número entero mayor que 1 que solo es divisible por 1 y por sí mismo. ¿Cómo puedes verificar si un número es divisible por otro?"
                    },
                    {
                        "order": 2,
                        "content": "Sea n un número entero, todos sus divisores son menores o iguales que n. ¿Cómo puedes verificar la divisibilidad desde 2 hasta n - 1?"
                    },
                    {
                        "order": 3,
                        "content": "Si el número es 2, es primo. Si es menor que 2 no es primo. Para números n mayores que 2, solo necesitas hacer un bucle desde d = 2 hasta d = n -1 verificando si n es divisible entre d, o sea comprobar que n % d == 0."
                    }
                ]
            }
        ]
    }
  ]
}
\end{lstlisting}
\caption{Ejemplo de JSON para personalización de ejercicios.}\label{fig:json}
\end{figure}

Para añadir más documentos al sistema de preguntas y respuestas, es necesario colocar los archivos en formato PDF dentro de la carpeta \texttt{data/corpus}. Cuando se ejecute, la aplicación procesará estos documentos.

En este capítulo se han descrito los detalles de implementación del bot de Telegram. Las funcionalidades implementadas, como el registro de usuarios, la resolución de dudas y la práctica con ejercicios, están diseñadas para mejorar la experiencia de aprendizaje de los estudiantes. Además, la posibilidad de personalizar el contenido del curso asegura que el sistema pueda adaptarse a las necesidades de diferentes asignaturas y profesores.

\backmatter

\begin{conclusions}
    Conclusiones
\end{conclusions}

\begin{recomendations}
    Recomendaciones
\end{recomendations}

\printbibliography


\end{document}