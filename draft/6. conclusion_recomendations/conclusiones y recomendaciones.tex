\documentclass{article}
\usepackage{graphicx}
\usepackage[spanish]{babel}
\usepackage{amsmath}
\usepackage{url}

%%%{{{ Comments and the like
\usepackage[textwidth=4cm]{todonotes}
\usepackage{soul}
\usepackage{xcolor}
\newcounter{todocounter}
\newcommand{\comment}[2]{\stepcounter{todocounter}
  {\color{green!50!blue}{(#1$^{{\color{black}\textbf{\thetodocounter}}}$)}}
  \todo[color=green,noline,size=\tiny]{\textbf{\thetodocounter:} #2

  }}
\newcommand{\quitaesto}[1]{{\color{red}(\st{#1})}}

\newcommand{\cambio}[2]{{\color{cyan}{{#2}}}{\color{red}{(\st{#1})}}}

\newcommand{\agregaesto}[1]{{\color{cyan}{{#1}}}}

\newcommand{\notaparaelautor}[1]{{\color{brown}{\textbf{#1}}}}

\newcommand{\errorortografico}[1]{{\fcolorbox{gray}{magenta}{\textcolor{yellow}{\bf #1}}}}
    
%%%}}}

\title{Conclusiones y recomendaciones}
\author{Anabel Benítez González}
\date{}

\begin{document}

\maketitle

\section{Conclusiones}

En este trabajo se diseñó un sistema para apoyar el estudio independiente de los estudiantes en el curso de Programación de la carrera de Ciencia de la Computación en MATCOM. El sistema, implementado como un bot de Telegram, busca integrarse de manera natural en el entorno académico, aprovechando la familiaridad de los estudiantes con esta plataforma. Su objetivo principal es que los estudiantes no se sientan solos durante su estudio independiente, sino que cuenten con un tutor virtual accesible todo el tiempo.

El sistema está diseñado para determinar el nivel de conocimiento de cada estudiante en función de los ejercicios que ha resuelto, y, en base a ello, sugerir ejercicios adaptados a sus necesidades. Esta personalización tiene como objetivo fomentar un aprendizaje progresivo, permitiendo a los estudiantes avanzar a su propio ritmo y evitando la frustración que puede surgir al enfrentar problemas demasiado complejos o simples. Además, el registro del estado de las actividades proporciona una visión del progreso de los estudiantes, lo que permite a los profesores adaptar el curso según las necesidades detectadas; la información recuperada sobre el desempeño individual y grupal facilita la identificación de áreas que requieren mayor atención en clases.

Asimismo, se desarrollaron e incluyeron en el sistema ejercicios con pistas escalonadas y soluciones detalladas, con el objetivo de guiar a los estudiantes en su proceso de aprendizaje, ofreciendo ayuda gradual y evitando la frustración.

Se implementó un sistema de Generación Aumentada con Recuperación (RAG, por sus siglas en inglés) para resolver dudas, asegurando que las respuestas estén basadas en la bibliografía oficial del curso. Esta funcionalidad busca reducir la dependencia de fuentes externas no verificadas, un problema frecuente entre estudiantes inexpertos que puede generar confusión y desmotivación.

Aunque el sistema aún no ha sido utilizado por estudiantes, se espera que su puesta en práctica tenga un impacto positivo en el proceso de enseñanza-aprendizaje. Este trabajo no solo ofrece una solución tecnológica, sino que también propone una metodología para mejorar la experiencia de aprendizaje en el curso de Programación. La integración del sistema en MATCOM establece las bases para futuras mejoras y ampliaciones, con el objetivo de continuar apoyando a estudiantes y profesores en el proceso de enseñanza-aprendizaje.

\section{Recomendaciones}

A partir del desarrollo y análisis del sistema, se derivan las siguientes recomendaciones:

\begin{itemize}
    \item Evaluar la efectividad del sistema en el curso de Programación en Ciencias de la Computación en \mbox{MATCOM}, utilizando métricas cuantitativas y cualitativas para medir su impacto en el aprendizaje.
    
    \item Se recomienda agregar nuevas funcionalidades al sistema y al bot, basadas en la retroalimentación de los usuarios durante su utilización.
    
    \item Almacenar las preguntas y respuestas generadas en una base de datos vectorial, lo que permitirá reducir las llamadas innecesarias al LLM.

    \item Agrupar y analizar las dudas más frecuentes de los estudiantes con el objetivo de proporcionar datos estadísticos útiles para identificar temas críticos y mejorar las conferencias en los próximos cursos.
    
    \item Implementar el sistema de evaluación automática de la solución de los ejercicios propuesto en el Capítulo~\ref{chapter:implementation}.
    
    \item Incorporar un flujo para profesores en el bot incluyendo un nuevo tipo de usuario, esto simplificará las tareas del profesor y eliminará la dependencia de \textit{scripts}.
\end{itemize}

\end{document}