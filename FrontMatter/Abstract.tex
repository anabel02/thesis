\begin{resumen}
	La programación es una habilidad esencial en el mundo moderno. En el ámbito académico, los cursos introductorios de programación son fundamentales para la formación de profesionales en Ciencia de la Computación, ya que proporcionan las bases para el desarrollo de habilidades avanzadas. Sin embargo, el estudio independiente en estos cursos enfrenta desafíos, como la dificultad de los estudiantes para gestionar el tiempo, la acumulación de conceptos no comprendidos y la falta de orientación sobre por dónde comenzar, lo que puede generar frustración y desmotivación.

	Este trabajo propone el diseño e implementación de un sistema de apoyo al estudio independiente para la asignatura de Programación en la carrera de Ciencia de la Computación de la Facultad de Matemática y Computación de la Universidad de La Habana (\mbox{MATCOM}). El sistema aborda problemas identificados, como la dependencia excesiva del apoyo docente y la falta de retroalimentación oportuna. Incluye funcionalidades como un sistema de preguntas y respuestas basado en Generación Aumentada con Recuperación (RAG), que permite a los estudiantes resolver dudas en tiempo real, y un mecanismo de propuesta de ejercicios adaptados al nivel de conocimiento del estudiante. Además, ofrece pistas durante la resolución de problemas y soluciones detalladas una vez completados, con el objetivo de fomentar el pensamiento crítico y la autonomía.

	El sistema se implementa como un bot en Telegram, una plataforma ampliamente utilizada por los estudiantes de \mbox{MATCOM}. Esta herramienta busca mejorar el rendimiento académico y promover habilidades como la autogestión del aprendizaje y la resolución de problemas. Este trabajo ofrece una solución adaptada a las necesidades específicas de los estudiantes de programación, con potencial para ser replicada en otros contextos académicos.
\end{resumen}

\begin{abstract}
	Programming is an essential skill in the modern world. Introductory programming courses are fundamental for the training of professionals in Computer Science, as they provide the foundation for developing advanced skills. However, independent study in these courses faces challenges, such as students' difficulty in managing time, the accumulation of misunderstood concepts, and the lack of guidance on where to begin, which can lead to frustration and demotivation.

	This work proposes the design and implementation of a system to support independent study for the Programming course in the Computer Science program at the Faculty of Mathematics and Computing of the University of Havana (\mbox{MATCOM}). The system addresses identified issues, such as excessive reliance on instructor support and the lack of timely feedback. It includes functionalities such as a question-and-answer system based on Retrieval-Augmented Generation (RAG), which allows students to resolve doubts in real time, and a mechanism for proposing exercises tailored to the student's knowledge level. Additionally, it provides hints during problem-solving and detailed solutions upon completion, aiming to foster critical thinking and autonomy.

	The system is implemented as a bot on Telegram, a platform widely used by \mbox{MATCOM} students. This tool seeks to improve academic performance and promote skills such as self-directed learning and problem-solving. This work offers a solution tailored to the specific needs of programming students, with the potential to be replicated in other academic contexts.
\end{abstract}