\documentclass{article}
\usepackage{graphicx}
\usepackage[spanish]{babel}
\usepackage{amsmath}
\usepackage{url}

\input{word-comments.tex}

\title{Descripción de la aplicación propuesta}
\author{Anabel Benítez González}
\date{}

\begin{document}

% \tableofcontents
\maketitle

En este capítulo se presenta una propuesta de aplicación diseñada para apoyar el estudio independiente en el curso de Programación en la carrera Ciencia de La Computación de \mbox{MATCOM}. En la sección \ref{sec:students} se describe el flujo de la aplicación para los estudiantes. En la sección \ref{sec:teachers} se explican las acciones disponibles para los profesores.

\section{Flujo para los estudiantes}\label{sec:students}

A continuación, se describe el flujo de uso de la aplicación, detallando las etapas por las que pasa el estudiante y cómo cada funcionalidad contribuye a su proceso de aprendizaje.

\subsection{Bienvenida}

El flujo comienza cuando el estudiante accede a la aplicación por primera vez. En este punto, la herramienta solicita al usuario que ingrese su nombre y grupo, lo que permite personalizar la experiencia de aprendizaje desde el inicio. A continuación, se muestra un mensaje de bienvenida junto con una guía básica que explica las funcionalidades disponibles. Esta introducción permite que el usuario comprenda cómo puede aprovechar la aplicación para su estudio.

Una vez que el estudiante se ha familiarizado con la aplicación, puede comenzar a explorar los temas disponibles, lo que le permitirá alinear su estudio independiente con las clases que recibe.

\subsection{Listado de temas}

Cuando el usuario lo solicita, la aplicación presenta un listado organizado de los temas disponibles para su estudio. Esta funcionalidad permite al estudiante identificar rápidamente los temas que se ajustan al contenido visto en conferencias y clases prácticas. Por ejemplo, algunos de los temas disponibles incluyen condicionales, arreglos y \textit{backtracking}.

La aplicación muestra los temas en el orden de estudio definido en el diseño del curso de Programación descrito en~\ref{sac:matcom}.

Si al usuario no le queda claro de qué va el tema solamente por el nombre, también puede pedir una descripción del mismo. La aplicación mostrará una explicación más detallada, incluyendo por qué se necesita estudiar el tema, definiciones básicas y ejemplos. Esta funcionalidad es útil para estudiantes que están comenzando y necesitan una comprensión más profunda de los conceptos básicos.

A medida que el estudiante explora los temas, ya sea en conferencias, clases prácticas o durante su estudio independiente, es común que surjan dudas o preguntas sobre el contenido. Para apoyar este proceso, la aplicación ofrece una sección dedicada a la resolución de dudas.

\subsection{Resolución de dudas}

Tradicionalmente, el estudiante podría recurrir a buscar respuestas en el libro de texto, pero este proceso puede ser lento y no siempre resuelve la duda de manera inmediata. Otra opción sería consultar al profesor, pero esto depende de la disponibilidad del docente. Incluso, podría utilizar herramientas como \textit{ChatGPT}, pero estas no siempre están alineadas con el contenido específico del curso.

Para simplificar y agilizar este proceso, la aplicación incluye una sección dedicada a la resolución de dudas. El estudiante puede formular preguntas en lenguaje natural, y el sistema responde con explicaciones claras, acompañadas de referencias a la bibliografía oficial del curso. Esta funcionalidad no solo ahorra tiempo, sino que también garantiza que las respuestas estén alineadas con el material de estudio, facilitando un aprendizaje más eficiente y autónomo.

Además de resolver dudas, la aplicación también ofrece herramientas para que los estudiantes practiquen y refuercen sus conocimientos a través de ejercicios adaptados a su nivel.

\subsection{Práctica con ejercicios sugeridos}

El usuario puede solicitar a la aplicación ejercicios relacionados con un tema específico. La aplicación sugiere ejercicios adaptados al nivel de conocimiento y progreso del estudiante, el cual se determina en función de la cantidad y complejidad de los ejercicios que ha resuelto previamente del tema que solicita. Estos ejercicios están diseñados para reforzar los conceptos aprendidos y preparar al estudiante para enfrentar problemas de mayor complejidad.

Los ejercicios se presentan en un orden progresivo, comenzando con problemas básicos y avanzando gradualmente hacia aquellos de mayor dificultad.

Una vez que el estudiante resuelve un ejercicio, envía el código y la aplicación proporciona retroalimentación inmediata, lo que le permite identificar aciertos y errores para mejorar su desempeño.

\subsection{Retroalimentación inmediata}

Después de que el estudiante envía la solución a un ejercicio, la aplicación proporciona retroalimentación inmediata. El sistema evalúa la solución y le informa al estudiante si es correcta o incorrecta. Si la solución es correcta, la aplicación puede sugerir al usuario que explore formas de mejorar la eficiencia del código, fomentando el desarrollo de habilidades de programación más avanzadas.

En caso de que la solución sea incorrecta, la aplicación identifica el error y lo comunica al estudiante. Además, el sistema registra la cantidad de intentos que el estudiante ha realizado para resolver el ejercicio, lo que permite monitorear su progreso y adaptar futuras recomendaciones de ejercicios según su desempeño. Esta funcionalidad no solo ayuda a identificar áreas de mejora, sino que también motiva al estudiante a persistir en la resolución de problemas.

Si el estudiante se encuentra con dificultades durante la resolución de un ejercicio, la aplicación ofrece pistas para guiarlo. A continuación se explica cómo funciona este proceso.

\subsection{Solicitud de pistas}

Durante la resolución de ejercicios, es posible que el estudiante enfrente dificultades y se sienta estancado. Para apoyarlo en estos momentos, la aplicación ofrece la opción de solicitar pistas. Las pistas están organizadas comenzando con sugerencias generales y avanzando hacia indicaciones más específicas. 

Este sistema permite al estudiante recibir ayuda gradualmente, sin revelar la solución completa, con la intención de que lo motive a descubrirla por sí mismo mientras refuerza su comprensión del tema. Por ejemplo, si el ejercicio consiste en implementar un algoritmo de ordenamiento, la primera pista podría ser una descripción general del algoritmo, mientras que las pistas posteriores podrían detallar cómo implementar cada paso o cómo manejar casos específicos.

Este sistema de pistas escalonadas no solo evita que el estudiante se sienta abrumado, sino que también le permite avanzar a su propio ritmo. Además, al descubrir la solución por sí mismo, el estudiante gana confianza en sus habilidades y desarrolla una mayor capacidad para resolver problemas de manera independiente.

En caso de que el estudiante no logre resolver el ejercicio incluso con las pistas, la aplicación ofrece la opción de consultar la solución completa. En la siguiente sección se explica esta funcionalidad.

\subsection{Solicitud de soluciones}

Si el estudiante, a pesar de utilizar las pistas escalonadas, no logra resolver el ejercicio, la aplicación ofrece la opción de solicitar la solución completa. Esta funcionalidad está pensada como último recurso, para evitar que el estudiante se frustre o pierda demasiado tiempo en un solo problema.

Al solicitar la solución, la aplicación no solo muestra el código o la respuesta final, sino que también incluye una explicación detallada de cada paso. Esto permite al estudiante comprender el razonamiento detrás de la solución y aprender de sus errores.

Además, incluso si el estudiante logra resolver el ejercicio por sí mismo, la aplicación le ofrece la opción de consultar la solución para que pueda comparar su enfoque con el propuesto. Ver una solución alternativa puede aportar nuevas perspectivas, mostrar métodos más eficientes o simplemente reforzar los conceptos aprendidos. Por ejemplo, un estudiante que resolvió un problema usando iteración podría descubrir una solución recursiva que no había considerado, lo que amplía su comprensión del tema.

Además de apoyar a los estudiantes, la aplicación también ofrece funcionalidades específicas para los profesores. En la siguiente sección, se detallan las herramientas disponibles para los docentes.

\section{Flujo para los profesores}\label{sec:teachers}

La aplicación no solo está diseñada para apoyar a los estudiantes, sino también para ser una herramienta útil para los profesores. Los profesores pueden utilizar la plataforma para crear y gestionar contenido, lo que les permite enriquecer el material disponible para los estudiantes y adaptarlo a las necesidades del curso.

El profesor tiene la capacidad de agregar y organizar contenido dentro de la aplicación. Esto incluye la creación de temas, donde el docente puede definir los contenidos del curso y proporcionar descripciones detalladas que expliquen su relevancia, objetivos de aprendizaje y conexiones con otros temas. Además, el profesor puede añadir ejercicios prácticos asociados a cada tema. Cada ejercicio cuenta con un nivel de dificultad definido, una solución y un conjunto de pistas escalonadas que guían al estudiante en caso de que necesite ayuda.

Esta funcionalidad de creación de contenido permite al profesor personalizar el material según el ritmo y las necesidades del curso, asegurando que los estudiantes tengan acceso a recursos adecuados y alineados con los objetivos del curso.

En este capítulo se ha presentado una propuesta de aplicación diseñada para apoyar el estudio independiente en el curso de Programación de la carrera de Ciencia de la Computación en \mbox{MATCOM}. La aplicación está estructurada en dos flujos principales: uno para los estudiantes y otro para los profesores, cada uno con funcionalidades específicas que buscan optimizar el proceso de enseñanza y aprendizaje.

El flujo propuesto está diseñado para guiar al estudiante a través de su proceso de aprendizaje. Cada etapa está planificada para fomentar la autonomía, el pensamiento crítico y el progreso continuo. Para los profesores, la aplicación se convierte en un aliado, permitiéndoles ajustar las estrategias de enseñanza basado en los datos recopilados de la aplicación. Esta propuesta tiene el potencial de transformar la manera en que los estudiantes y profesores abordan el estudio de la programación, ofreciendo un apoyo personalizado y accesible en cualquier momento.

Para construir esta herramienta, fue necesario recopilar ejercicios agrupados por los temas del curso de Programación de la carrera de Ciencia de la Computación en \mbox{MATCOM}, diseñar soluciones y desarrollar un conjunto de pistas progresivas para cada ejercicio. En el siguiente capítulo, se aborda esta parte del trabajo.

\end{document}