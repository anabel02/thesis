\begin{conclusions}
En este trabajo se diseñó un sistema para apoyar el estudio independiente de los estudiantes en el curso de Programación de la carrera de Ciencia de la Computación en MATCOM. El sistema, implementado como un bot de Telegram, busca integrarse en el entorno académico, aprovechando la familiaridad de los estudiantes con esta plataforma. Su objetivo principal es que los estudiantes no se sientan solos durante su estudio independiente, sino que cuenten con un tutor virtual accesible todo el tiempo.

El sistema está diseñado para determinar el nivel de conocimiento de cada estudiante en función de los ejercicios que ha resuelto, y, en base a ello, sugerir ejercicios adaptados a sus necesidades. Esta personalización tiene como objetivo fomentar un aprendizaje progresivo, permitiendo a los estudiantes avanzar a su propio ritmo y evitando la frustración que puede surgir al enfrentar problemas demasiado complejos o simples. Además, el registro del estado de las actividades proporciona una visión del progreso de los estudiantes, lo que permite a los profesores adaptar el curso según las necesidades detectadas; la información recuperada sobre el desempeño individual y grupal facilita la identificación de áreas que requieren mayor atención en clases.

Asimismo, se desarrollaron e incluyeron en el sistema ejercicios con pistas escalonadas y soluciones detalladas, con el objetivo de guiar a los estudiantes en su proceso de aprendizaje, ofreciendo ayuda gradual y evitando la frustración.

Se implementó un sistema de Generación Aumentada con Recuperación (RAG, por sus siglas en inglés) para resolver dudas, asegurando que las respuestas estén basadas en la bibliografía oficial del curso. Esta funcionalidad busca reducir la dependencia de fuentes externas no verificadas, un problema frecuente entre estudiantes inexpertos que puede generar confusión y desmotivación.

Este trabajo no solo ofrece una solución tecnológica, sino que también propone una metodología para mejorar la experiencia de aprendizaje en el curso de Programación. La integración del sistema en MATCOM establece las bases para futuras mejoras y ampliaciones, con el objetivo de continuar apoyando a estudiantes y profesores en el proceso de enseñanza-aprendizaje.
\end{conclusions}
