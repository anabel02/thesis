\documentclass{article}
\usepackage{graphicx}
\usepackage[spanish]{babel}
\usepackage{amsmath}
\usepackage{url}

%%%{{{ Comments and the like
\usepackage[textwidth=4cm]{todonotes}
\usepackage{soul}
\usepackage{xcolor}
\newcounter{todocounter}
\newcommand{\comment}[2]{\stepcounter{todocounter}
  {\color{green!50!blue}{(#1$^{{\color{black}\textbf{\thetodocounter}}}$)}}
  \todo[color=green,noline,size=\tiny]{\textbf{\thetodocounter:} #2

  }}
\newcommand{\quitaesto}[1]{{\color{red}(\st{#1})}}

\newcommand{\cambio}[2]{{\color{cyan}{{#2}}}{\color{red}{(\st{#1})}}}

\newcommand{\agregaesto}[1]{{\color{cyan}{{#1}}}}

\newcommand{\notaparaelautor}[1]{{\color{brown}{\textbf{#1}}}}

\newcommand{\errorortografico}[1]{{\fcolorbox{gray}{magenta}{\textcolor{yellow}{\bf #1}}}}
    
%%%}}}

\title{Introducción}
\author{Anabel Benítez González}
\date{}

\begin{document}

\maketitle

La programación se ha convertido en una habilidad fundamental en el mundo moderno, no solo para el desarrollo tecnológico, sino también para la resolución de problemas en prácticamente todas las áreas del conocimiento. En un universo cada vez más digitalizado, la capacidad de programar es esencial para comprender y transformar la realidad que nos rodea. En particular, en el ámbito académico, el dominio de la programación es crucial para formar profesionales capaces de innovar y adaptarse a los desafíos del futuro. Por ello, el desarrollo de herramientas que faciliten el aprendizaje autónomo de la programación no solo es relevante, sino también una necesidad imperante para garantizar que las nuevas generaciones estén preparadas para enfrentar los retos del siglo XXI.

\todo{estudio independiente}


Diversos investigadores y educadores han abordado el desafío de mejorar el aprendizaje de la programación a través de herramientas tecnológicas. Por ejemplo, se han desarrollado plataformas como Codecademy y Khan Academy, que ofrecen cursos interactivos de programación. Además, sistemas basados en inteligencia artificial, como AutoTutor, han demostrado ser efectivos para personalizar el aprendizaje en otras disciplinas. Sin embargo, estas soluciones no están específicamente diseñadas para el contexto de Matcom, ni consideran las particularidades del plan de estudios y las necesidades de los estudiantes de esta facultad.

A diferencia de las soluciones existentes, este trabajo propone un sistema diseñado específicamente para apoyar el estudio independiente de programación en Matcom. Este sistema no solo integra recursos adaptados al plan de estudios de la facultad, sino que también incorpora mecanismos de retroalimentación personalizada y seguimiento del progreso del estudiante. Además, se enfoca en resolver problemas comunes identificados en el proceso de aprendizaje autónomo, como la falta de motivación.

El objetivo general de este trabajo es diseñar e implementar un sistema de apoyo al estudio independiente de programación para los estudiantes de Matcom, que facilite su proceso de aprendizaje mediante recursos personalizados, retroalimentación automática y seguimiento del progreso.

Los objetivos específicos son:
\begin{itemize}
    \item Realizar una revisión de la bibliografía existente sobre sistemas de apoyo al aprendizaje de programación y metodologías de estudio independiente.
    \item Identificar las necesidades específicas de los estudiantes de Matcom en relación con el estudio autónomo de programación.
    \item Diseñar un sistema que integre recursos educativos, ejercicios prácticos y mecanismos de retroalimentación personalizada.
    \item Implementar un prototipo del sistema.
    \item Proponer recomendaciones para la mejora del sistema.
\end{itemize}

Como la mayoría de los estudiantes de la facultad están familiarizados con el uso de Telegram, se diseñó un bot que combine contenidos teóricos y ejercicios prácticos. Esta propuesta busca no solo facilitar el aprendizaje de programación, sino también motivar a los estudiantes a través de una experiencia de usuario intuitiva y disponible todo el tiempo.

\begin{thebibliography}{99}
  
\end{thebibliography}

\end{document}