\chapter{Preliminares}\label{chapter:background}

El objetivo de este trabajo es diseñar e implementar una herramienta que apoye el estudio independiente de los estudiantes de Programación. En este capítulo se presentan los elementos fundamentales que sustentan la investigación.

En la \Cref{sec:ensenanza_programacion} se presentan algunos de los fundamentos de la enseñanza de programación. En la \Cref{sec:contents}, se enumeran los contenidos que suelen incluir los cursos básicos de programación a nivel global. En la \Cref{sec:resources} se analizan los materiales y recursos empleados en la enseñanza de programación, desde los tradicionales libros de texto hasta aplicaciones interactivas.

En la \Cref{sec:study} se analiza el estudio independiente y su importancia en el aprendizaje de la programación. En la \Cref{sec:matcom} se describe el curso de Programación que se imparte en primer año de Ciencia de la Computación en la Facultad de Matemática y Computación de la Universidad de La Habana (\mbox{MATCOM}).

\section{Elementos sobre la enseñanza de la programación}\label{sec:ensenanza_programacion}

El objetivo de los cursos introductorios de programación en Ciencia de la Computación es que los estudiantes no solo aprendan a escribir código, sino que también adquieran una mentalidad analítica que les permita abordar problemas de manera estructurada y lógica~\cite{JOHNLEMAY2021100056}. Estos cursos combinan teoría y práctica~\cite{Sarsa_2022}, o sea, los estudiantes aprenden los conceptos aplicándolos directamente en ejercicios y proyectos.

Estos cursos y sus tareas suelen diseñarse introduciendo conceptos y actividades de forma gradual, aumentando su nivel de dificultad de manera escalonada para evitar que los estudiantes se sientan abrumados al recibir más contenido del que pueden procesar, previniendo así la sobrecarga cognitiva~\cite{duran2021clt}.

La carga cognitiva se refiere a la intensidad de la actividad mental necesaria para alcanzar un objetivo de aprendizaje en un tiempo limitado~\cite{duran2021clt}. La sobrecarga cognitiva ocurre cuando esta carga supera la capacidad del estudiante para procesar y asimilar la información, dificultando el aprendizaje y comprometiendo los resultados~\cite{duran2021clt}. 

Este proceso de introducir conceptos gradualmente permite que los estudiantes construyan una base sobre la que puedan abordar desafíos más difíciles, y ayuda a los estudiantes a desarrollar un sentido de logro a medida que avanzan. Este proceso se relaciona con la teoría de la zona de desarrollo próximo~\cite{vygotsky1978mind}, según la cual, los estudiantes pueden adquirir habilidades que aún no son capaces de dominar de forma independiente, siempre y cuando reciban la orientación adecuada. A medida que los estudiantes progresan, su zona de desarrollo próximo se expande, permitiéndoles superar nuevos obstáculos, mejorar sus habilidades de programación y abordar problemas cada vez más complejos.

De acuerdo con~\cite{rosenzweig2019expectancy} la motivación de los estudiantes está estrechamente ligada al nivel de dificultad de las tareas que se les orientan. Ejercicios demasiado simples pueden llevar al desinterés, mientras que aquellos que superan en complejidad las habilidades actuales de los estudiantes tienden a generar frustración y desaliento~\cite{rosenzweig2019expectancy}. Este desequilibrio no solo afecta la disposición de los estudiantes para continuar con el aprendizaje de la programación, sino que también puede debilitar su percepción de autoeficacia~\cite{bandura1977self}.

Para evitar que los estudiantes enfrenten tareas demasiado complejas de forma prematura, los cursos introductorios de programación suelen incluir numerosos ejercicios pequeños que promueven la adquisición de habilidades mediante la práctica regular~\cite{vihavainen2011extreme}. Dado que los estudiantes tienen diferentes antecedentes, habilidades y zonas de desarrollo próximo, un enfoque personalizado con tareas adaptadas a sus necesidades podría ser beneficioso~\cite{leinonen2021exploring}. Sin embargo, crear conjuntos de ejercicios personalizados representa una tarea extremadamente laboriosa~\cite{lobb2016coderunner, wrenn2018whotests}.

El diseño de los cursos de programación usualmente se basa en la práctica deliberada~\cite{ericsson1993deliberate}, un enfoque que atribuye el desarrollo de la experticia a la práctica sistemática, prolongada en el tiempo y enfocada en mejorar habilidades específicas. La práctica deliberada enfatiza que el desempeño sobresaliente depende menos del talento innato y más del esfuerzo y perseverancia~\cite{ericsson1993deliberate}.

Para comprender mejor cómo se estructuran estos cursos, en la siguiente sección se analizan los contenidos que se imparten en los cursos introductorios de programación.

\section{¿Qué se enseña en un curso introductorio de programación?}\label{sec:contents}

Los cursos introductorios de programación están diseñados para estudiantes sin experiencia previa en la materia. Generalmente, cubren los fundamentos de la programación, el desarrollo de habilidades para la resolución de problemas y el uso de un lenguaje de programación para implementar soluciones~\cite{Medeiros2019}.

Algunos cursos también incorporan conceptos más avanzados, como la recursividad, que permite resolver problemas complejos mediante su descomposición en subproblemas más simples~\cite{cs50_harvard,mit_60001}. Además, en algunos cursos se introducen principios de Programación Orientada a Objetos, incluyendo clases, objetos, herencia y polimorfismo, lo que facilita la construcción de software modular y escalable~\cite{cs50_harvard, mit_60001}.

Los instructores de cursos introductorios de programación a nivel universitario trabajan constantemente en mejorar sus planes de enseñanza y en seleccionar recursos de alta calidad para ayudar a los estudiantes a alcanzar los objetivos de aprendizaje previstos~\cite{wong2022}. En la próxima sección, se mencionan los diferentes recursos que se emplean para enseñar programación.

\section{Materiales utilizados para aprender a programar}\label{sec:resources}

El aprendizaje de la programación se apoya en una variedad de materiales diseñados para complementar las clases, fomentar el estudio independiente y facilitar la comprensión de conceptos complejos. Algunos de estos materiales son los libros de texto, presentaciones y conjuntos de ejercicios que desarrollan los docentes para los estudiantes. Además, existen recursos como plataformas en línea, videos en \textit{Youtube} y herramientas como \textit{ChatGPT}.

Los libros de texto constituyen la bibliografía básica en los cursos de programación~\cite{wong2022}. Estos libros sirven como referencia para estudiantes y docentes, ofreciendo teoría y ejemplos prácticos.

Para apoyar las clases, los docentes desarrollan presentaciones como herramientas visuales que facilitan la enseñanza de conceptos. Estas permiten ilustrar ideas, presentarlas de manera estructurada e interactuar con los estudiantes durante las explicaciones.

Como se menciona en la \Cref{sec:ensenanza_programacion}, la práctica es un elemento clave en los cursos de programación. Los conjuntos de ejercicios, como parte de los materiales utilizados en estos cursos, constituyen una de las principales formas de practicar. Diseñados para las sesiones de laboratorio, estos ejercicios permiten a los estudiantes aplicar los conceptos aprendidos y desarrollar habilidades en resolución de problemas.

Los recursos anteriores forman parte de los materiales oficiales de los cursos, aunque los estudiantes también suelen buscar apoyo en otras fuentes. A continuación, se enumeran algunas de estas fuentes:

\begin{itemize}
    \item Plataformas de ejercicios: 
    Plataformas como \textit{HackerRank}~\cite{hackerrank}, \textit{LeetCode}~\cite{leetcode}  y \textit{GeeksforGeeks}~\cite{geeksforgeeks} ofrecen una amplia variedad de problemas categorizados por nivel de dificultad y temática. Estos ejercicios pueden ayudar a los estudiantes a consolidar sus conocimientos y mejorar sus habilidades de resolución de problemas.

    \item Videotutoriales:
    Los videotutoriales son una herramienta educativa versátil y accesible que combina explicaciones visuales y prácticas. Están disponibles en plataformas como \textit{YouTube}~\cite{youtube}, y suelen abordar temas desde lo más básico hasta lo avanzado. Estos videos permiten a los estudiantes aprender a su propio ritmo, pausando y repitiendo secciones según sea necesario.

    \item Foros y comunidades: 
    Foros como \textit{Stack Overflow}~\cite{stackoverflow} y comunidades en redes sociales como \textit{Reddit}~\cite{reddit} permiten a los estudiantes interactuar directamente con expertos y compañeros. Aquí pueden resolver dudas, compartir conocimientos y aprender de los desafíos que otros enfrentan.
    
    \item Documentación oficial:
    La documentación oficial de los lenguajes, como  \textit{Java API}~\cite{java-api}, \textit{Python.org}~\cite{python} y \textit{Microsoft Docs}~\cite{microsoft-docs}, proporciona una referencia confiable y mantenida por los propios desarrolladores del lenguaje. Estos recursos incluyen ejemplos y buenas prácticas.
    
    \item Chatbots: 
    Entre los recursos que complementan el proceso de aprendizaje, los chatbots basados en modelos de lenguaje de gran tamaño (LLMs), como \textit{ChatGPT}~\cite{chatgpt} y \textit{Microsoft Copilot}~\cite{copilot}, destacan por su capacidad de ofrecer soporte personalizado y en tiempo real.
\end{itemize}

A pesar de la disponibilidad de múltiples recursos, la programación sigue enfrentando un alto índice de abandono de los cursos. Según~\cite{wong2022}, esto se debe en gran medida a que los estudiantes priorizan la búsqueda de soluciones inmediatas sobre la comprensión de los conceptos fundamentales. Muchos recurren a un ciclo de prueba y error, modificando el código repetidamente hasta que funciona, sin entender el razonamiento detrás de los cambios. Este enfoque, apoyado en foros de preguntas y respuestas, no solo consume tiempo, sino que también impide un aprendizaje profundo y significativo.

Otro factor que contribuye a este problema es la percepción de un ``alto costo social'' al pedir ayuda~\cite{wong2022}. Los estudiantes suelen evitar consultar a instructores o compañeros por temor a parecer poco competentes o dependientes. Esta ansiedad, generada por la presión de demostrar autonomía o por experiencias previas negativas, los lleva a depender excesivamente de recursos en línea. Sin embargo, esta dependencia rara vez resuelve sus dudas de fondo, perpetuando la frustración y el estancamiento en su progreso.

En la siguiente sección, se analizan algunos aspectos del estudio independiente de la programación, con el objetivo de identificar estrategias que permitan superar estos obstáculos y fomentar un aprendizaje más efectivo.

\section{Estudio independiente de la programación}\label{sec:study}

El estudio independiente es el proceso en el que los estudiantes gestionan su propio aprendizaje, generalmente siguiendo las instrucciones de un docente, pero sin su supervisión constante, aprovechando los recursos disponibles para desarrollar conocimientos y habilidades fuera del aula~\cite{proskuraLytvynova2020}.

Según~\cite{proskuraLytvynova2020}, el estudio independiente es clave en la formación de estudiantes en Ciencia de la Computación, quienes deben asumir el aprendizaje autónomo como base para su desarrollo académico y profesional. Esta práctica cobra importancia ante la evolución tecnológica, que exige a los profesionales en programación mantener un aprendizaje constante~\cite{proskuraLytvynova2020}.

El éxito en el estudio independiente depende en gran medida de la capacidad de los estudiantes para gestionar su tiempo y organizar sus actividades de estudio~\cite{overklift2019}. La falta de orientación directa por parte del personal docente puede llevar a confusión sobre cómo priorizar contenidos y equilibrar la programación con otras asignaturas o responsabilidades~\cite{proskuraLytvynova2020}. 

Sin un instructor disponible, los estudiantes pueden encontrarse con conceptos o problemas complejos que no logran resolver por sí mismos, lo que puede generar frustración y afectar su confianza~\cite{overklift2019}. Un mal estudio independiente puede tener consecuencias significativas en el rendimiento académico de los estudiantes. Según~\cite{overklift2019}, aquellos que enfrentan dificultades para organizar y estructurar su tiempo de estudio suelen experimentar problemas para cumplir con los requisitos de sus cursos.

De acuerdo con~\cite{overklift2019} cuando los estudiantes reconocen problemas en su proceso de aprendizaje autónomo, suelen buscar apoyo en el personal docente; sin embargo, los estudiantes a menudo tardan en identificar que necesitan ayuda, y la disponibilidad del docente puede ser insuficiente para brindar un apoyo personalizado a todos los alumnos.

En este sentido, la retroalimentación proporcionada durante el proceso de aprendizaje puede resultar fundamental~\cite{Sarsa_2022}. La retroalimentación fomenta habilidades como el aprendizaje autorregulado~\cite{Sarsa_2022}, que se refiere a la capacidad de los estudiantes para planificar, supervisar y evaluar su propio proceso de aprendizaje, estableciendo metas, seleccionando estrategias y ajustando su comportamiento según los resultados obtenidos

La retroalimentación también promueve la metacognición~\cite{Sarsa_2022}, es decir, la conciencia y comprensión que tienen los estudiantes de sus propios procesos de pensamiento. Estas habilidades permiten a los estudiantes reflexionar sobre sus métodos, identificar áreas de mejora y ajustar sus estrategias de aprendizaje. La retroalimentación ayuda a los estudiantes no solo a resolver problemas específicos, sino también a desarrollar la capacidad de aprender de manera autónoma, convirtiéndose en aprendices más efectivos y adaptables~\cite{shute2008focus}.

El estudio independiente, como se ha mencionado, representa una herramienta esencial para el crecimiento académico y profesional, particularmente en disciplinas como la Ciencia de la Computación. Sin embargo, su éxito depende en gran medida de un marco educativo bien definido que proporcione dirección y recursos adecuados. En la siguiente sección, se detalla el curso de Programación que se imparte en la carrera de Ciencia de la Computación en \mbox{MATCOM}.

\section{Curso de Programación en \mbox{MATCOM}}\label{sec:matcom}

El curso de Programación forma parte del primer año de la carrera de Ciencia de la Computación de la Universidad de La Habana. Programación ocupa un lugar central en el Plan de Estudio debido a su importancia en la formación de las capacidades y habilidades básicas en el perfil profesional de Computación~\cite{plan_estudio_e_2017}.

La programación es una habilidad fundamental en la formación de los estudiantes de Ciencia de la Computación, ya que les permite transformar ideas abstractas en soluciones prácticas. Además, fomenta el desarrollo de habilidades críticas como la resolución de problemas, la creatividad y la capacidad de trabajar en equipo~\cite{plan_estudio_e_2017}.

Las siguientes secciones presentan la estructura del curso de Programación, los recursos empleados y los contenidos impartidos a lo largo del mismo.

\subsection{Estructura del curso}

El curso está orientado a estudiantes sin experiencia previa en programación. Se organiza en dos semestres, cada uno con una duración de 16 semanas. En la \Cref{tab:course_distribution} se presenta la distribución temporal de los contenidos, que ofrece una visión general del curso y de las evaluaciones.

\begin{table}[h!]
    \centering
    \begin{tabular}{|p{6.5cm}|p{2cm}|p{3.5cm}|}
    \hline
    \textbf{Temas} & \textbf{Semanas} & \textbf{Evaluaciones} \\ \hline
    \raggedright Fundamentos de Programación. \\ Programación Estructurada. 
    & 8 & Prueba parcial y Proyecto extraclase \\ \hline
    \raggedright Recursividad.
    & 8 & Prueba parcial \\ \hline
    \raggedright Programación Orientada a Objetos. \\ Programación Funcional. \\ Estructuras de Datos. 
    & 16 & Prueba parcial y Proyecto extraclase \\ \hline
    \end{tabular}
    \caption{Distribución del curso}\label{tab:course_distribution}
\end{table}
    

El curso combina teoría y práctica, con una conferencia semanal para la presentación de conceptos y dos clases prácticas que permiten a los estudiantes aplicar y consolidar lo aprendido en un entorno guiado. A continuación se describen cada una de estas sesiones.

\subsubsection{Conferencias}

Las conferencias constituyen el espacio destinado a la introducción de contenidos teóricos y la demostración práctica de su aplicación mediante códigos de ejemplo. Estas sesiones tienen como propósito principal establecer las bases conceptuales, al tiempo que muestran el uso adecuado del lenguaje de programación y del entorno de desarrollo integrado (IDE) seleccionado para el curso~\cite{plan_estudio_e_2017}.

Durante las conferencias, se utiliza un proyector para facilitar la presentación visual de conceptos, fragmentos de código y su ejecución en tiempo real. Este enfoque permite que los estudiantes observen cómo los conceptos se traducen en implementaciones prácticas y cómo interactúan las diferentes partes de un programa. Asimismo, estas sesiones fomentan la participación activa mediante preguntas y discusiones que consolidan el entendimiento de los contenidos.

\subsubsection{Clases Prácticas}

Las clases prácticas tienen como objetivo consolidar los conceptos presentados en las conferencias mediante la resolución de problemas y la implementación de programas~\cite{plan_estudio_e_2017}.

Cada semana, la primera clase práctica se realiza en el aula y está enfocada en vincular los conceptos teóricos y algoritmos presentados en la conferencia con su aplicación práctica. Esto se logra mediante ejercicios guiados, discusiones de problemas y la exploración de diversas estrategias de solución. Los estudiantes analizan problemas, debaten algoritmos y comparten enfoques para resolverlos, mientras el profesor supervisa y evalúa los intentos de solución, ofreciendo retroalimentación para consolidar el aprendizaje.

La segunda clase práctica, realizada en el laboratorio, está orientada a la implementación y los estudiantes trabajan en computadoras. Durante esta sesión, los estudiantes escriben, depuran y prueban código en tiempo real, enfrentándose a ejercicios que incluyen desde la creación de programas elementales hasta la resolución de problemas complejos. Además, en esta clase, los docentes proporcionan orientación técnica, ayudando a los estudiantes a resolver dudas sobre sintaxis, lógica, depuración, etc. 

A través de esta clase, se busca que los estudiantes desarrollen confianza en sus habilidades de programación y aprendan a aplicar conceptos de forma práctica, en un entorno de trabajo supervisado.

Las conferencias y clases prácticas, aunque fundamentales para el aprendizaje, no son los únicos recursos disponibles para los estudiantes. Además de estas actividades guiadas, el curso cuenta con una serie de recursos de enseñanza que enriquecen el proceso de aprendizaje. En la siguiente sección se mencionan estos recursos.

\subsection{Recursos de enseñanza}

El curso utiliza el lenguaje de programación C\# como vehículo para la enseñanza de los conceptos fundamentales de programación. Este lenguaje destaca por su versatilidad, enfoque en la programación orientada a objetos y su relevancia en la industria del desarrollo de software~\cite{microsoft-docs, Albahari2017, Troelsen2021}.

El IDE empleado es Visual Studio Code~\cite{vscode}, una herramienta reconocida por su accesibilidad y flexibilidad. El uso combinado de C\# y Visual Studio Code busca que los estudiantes se familiaricen con herramientas y prácticas estándar de la industria~\cite{microsoft-docs}, preparándolos para proyectos futuros, tanto académicos como profesionales.

El texto principal de la asignatura es el libro ``Empezar a programar. Un enfoque multiparadigma con C\#''~\cite{katrib_programar}. Este libro fue escrito por el profesor Miguel Katrib, quien ha dirigido la disciplina de Programación en la carrera de Ciencia de la Computación en \mbox{MATCOM} durante más de cuatro décadas.

En la siguiente sección se detalla el contenido del curso.

\subsection{Contenidos impartidos en el curso}  

A lo largo del curso se imparten los siguientes contenidos~\cite{plan_estudio_e_2017}:

\begin{enumerate}  
    \item Introducción a la programación y al entorno de desarrollo.  
    Se presentan las nociones básicas sobre cómo funcionan las computadoras, qué son los algoritmos y programas, y cómo los lenguajes de programación facilitan la comunicación con los sistemas. También se explora el entorno de desarrollo que se utilizará durante el curso.  

    \item Estructuras básicas: variables, tipos de datos y operaciones.
    Este tema abarca los fundamentos de la programación, como el uso de variables para almacenar información, los diferentes tipos de datos que se manejan en un programa y las operaciones disponibles para manipularlos.  

    \item Control de flujo: estructuras de control condicionales e iterativas. 
    Tiene como objetivo aprender a dirigir la ejecución del programa utilizando estructuras de decisión (\texttt{if}, \texttt{switch}) y ciclos (\texttt{while}, \texttt{for}).

    \item Métodos y modularización del código.
    Este tema introduce el concepto de dividir un programa en componentes más pequeños y reutilizables, lo que permite un desarrollo más organizado y eficiente.  

    \item Arrays: el concepto primario de colección.
    Se estudia cómo los arrays permiten almacenar y manipular múltiples elementos del mismo tipo. También se abordan las estructuras de control asociadas y el uso de arrays multidimensionales.

    \item Introducción a la complejidad algorítmica.
    Se analizan métodos de búsqueda y ordenación, ilustrando cómo el volumen de datos afecta el desempeño de los programas.  

    \item Recursividad.
    Se estudia la recursividad como técnica de resolución de problemas basada en dividirlos en subproblemas más pequeños.

    \item Estrategias de solución de problemas.
    Se abordan enfoques comunes como fuerza bruta, divide y vencerás, y \textit{backtracking}, con aplicaciones prácticas para resolver problemas complejos.  

    \item Combinatoria.
    Se introduce la combinatoria para resolver problemas relacionados con el conteo y las permutaciones. 

    \item Jerarquías de tipos: herencia y polimorfismo.
    Este tema introduce conceptos avanzados de la programación orientada a objetos, como la reutilización del código mediante herencia, el uso de polimorfismo para trabajar con tipos genéricos y las conversiones entre tipos (casting).  

    \item Clases abstractas. Interfaces. Genericidad.
    Los estudiantes aprenden a diseñar tipos y algoritmos genéricos utilizando interfaces y clases abstracta, además de manejar iteradores para recorrer colecciones de datos.

    \item Elementos de programación funcional.
    Se exploran conceptos básicos de programación funcional, como delegados y funciones como ciudadanos de primera clase, que permiten trabajar con un enfoque más declarativo y modular.

    \item Estructuras de datos.
    Se estudian estructuras clásicas implementadas con arrays y listas enlazadas, analizando el concepto de nodo y enlace como base para estructuras avanzadas.  

    \item Pilas, colas, diccionarios y árboles.
    Este tema profundiza en estructuras de datos y su uso en problemas prácticos.
\end{enumerate}

En este capítulo se ha explorado los fundamentos de los cursos introductorios de programación, los materiales disponibles para estudiar programación, la importancia del estudio independiente en programación, los desafíos que enfrentan los estudiantes y cómo la falta de retroalimentación oportuna y personalizada puede afectar su rendimiento. En particular, en \mbox{MATCOM} se ha observado que la efectividad del estudio independiente de los estudiantes deja mucho que desear, lo que resalta la necesidad de implementar estrategias que complementen el proceso de aprendizaje.

Para abordar los desafíos descritos en este capítulo, se propone un sistema que se integre al curso de Programación de la carrera de Ciencia de la Computación en \mbox{MATCOM}, con el objetivo de  mejorar la experiencia de estudio independiente. Este sistema proporcionará  a los estudiantes respuestas a preguntas sobre el contenido, sugerencias de ejercicios, pistas para resolverlos, y retroalimentación.

En el siguiente capítulo, se aborda la Generación Aumentada con Recuperación, una técnica que combina la recuperación de información con la generación de respuestas de los LLMs, esta técnica resulta conveniente para implementar el componente del sistema destinado a resolver dudas.
