\chapter{Descripción del sistema propuesto}\label{chapter:proposal}

En este capítulo se presenta una propuesta de un sistema diseñado para apoyar el estudio independiente en el curso de Programación de la carrera de Ciencia de la Computación de \mbox{MATCOM}.

El sistema determina el nivel de cada estudiante y, en función de este, propone ejercicios adaptados a sus necesidades. Los estudiantes pueden solicitar pistas para resolver los ejercicios. El sistema registra el estado de cada actividad: si el estudiante ha comenzado a trabajar en ella, si tiene dudas o si ha completado el ejercicio, permitiendo un seguimiento detallado de su progreso.

Por otro lado, los profesores pueden monitorear el avance de los estudiantes, identificar dificultades comunes y ajustar el contenido o las estrategias de enseñanza según sea necesario. Esta información les ayuda a brindar retroalimentación oportuna y a personalizar el apoyo ofrecido.

La implementación de estas funcionalidades se realizó mediante un bot de Telegram. Sin embargo, es importante destacar que estas tareas también podrían realizarse con herramientas más simples, como hojas de cálculo o archivos de texto. De hecho, durante varios años, algunos profesores de \mbox{MATCOM} han gestionado procesos similares utilizando estas herramientas básicas~\cite{rodriguez}. En este sentido, la aplicación no es solo una solución tecnológica, sino una propuesta para mejorar la experiencia de aprendizaje.

En la \Cref{sec:students} se describe el flujo propuesto para los estudiantes. En la \Cref{sec:teachers} se explican las acciones disponibles para los profesores.

\section{Flujo para los estudiantes}\label{sec:students}

A continuación, se describe el flujo de uso del sistema propuesto para los estudiantes, detallando las etapas por las que pasa el estudiante y cómo cada etapa contribuye a su proceso de aprendizaje.

\subsection{Registro}

El primer paso para el estudiante es registrarse en el sistema. Para ello, debe proporcionar su nombre, apellidos y grupo. Este paso permite llevar un registro de los estudiantes y facilita su seguimiento a lo largo del curso. 

Una vez completado el registro, el estudiante puede comenzar a explorar los temas disponibles, lo que le permitirá alinear su estudio independiente con las clases que recibe.

\subsection{Listado de temas}

Cuando el usuario lo solicita, se le presenta un listado de los temas disponibles para su estudio. Esta funcionalidad permite al estudiante identificar rápidamente los temas que se ajustan al contenido visto en conferencias y clases prácticas. Por ejemplo, algunos de los temas disponibles pueden ser condicionales, \textit{arrays} y \textit{backtracking}.

Si al usuario no le queda claro de qué va el tema solamente por el nombre, también puede pedir una descripción del mismo, que incluye por qué se necesita estudiar el tema, definiciones básicas y ejemplos. Esta funcionalidad es útil para estudiantes que están comenzando y necesitan una comprensión más profunda de los conceptos básicos.

A medida que el estudiante explora los temas, ya sea en conferencias, clases prácticas o durante su estudio independiente, es común que surjan dudas o preguntas sobre el contenido. Para apoyar este proceso el sistema propuesto ofrece una sección dedicada a la resolución de dudas.

\subsection{Resolución de dudas}

Tradicionalmente, el estudiante podría recurrir a buscar respuestas a sus dudas en el libro de texto, pero este proceso puede ser lento y no siempre resuelve la duda. Otra opción sería utilizar herramientas como \textit{ChatGPT}; sin embargo, estas pueden generar alucinaciones, es decir, respuestas que parecen coherentes pero que son incorrectas, lo cual representa un riesgo, especialmente para estudiantes que aún no han desarrollado la capacidad crítica necesaria para distinguir entre información precisa y errónea.

Para simplificar este proceso, el sistema incluye una sección dedicada a la resolución de dudas. El estudiante puede formular preguntas en lenguaje natural, y el sistema responde con explicaciones, acompañadas de referencias a la bibliografía oficial del curso. Esta funcionalidad garantiza que las respuestas estén alineadas con el material de estudio, de esta manera el estudiante puede ir a consultar en el libro si lo necesita. 

Para implementar esta funcionalidad, se puede utilizar un sistema de Generación Aumentada por Recuperación (RAG, por sus siglas en inglés), que se presenta en el \Cref{chapter:rag}. 

Además de resolver dudas, el sistema también ofrece herramientas para que los estudiantes practiquen y refuercen sus conocimientos a través de ejercicios adaptados a su nivel.

\subsection{Práctica con ejercicios sugeridos}

El usuario puede solicitar ejercicios relacionados con un tema específico. El sistema le sugiere ejercicios adaptados al nivel de conocimiento y progreso del estudiante, el cual se determina en función de la cantidad y complejidad de los ejercicios que ha resuelto previamente del tema que solicita. Estos ejercicios están diseñados para reforzar los conceptos aprendidos y preparar al estudiante para enfrentar problemas de mayor complejidad.

Los ejercicios se presentan en un orden progresivo, comenzando con problemas básicos y avanzando gradualmente hacia aquellos de mayor dificultad.

Una vez que el estudiante resuelve un ejercicio, envía el código y el sistema proporciona retroalimentación, lo que le permite identificar aciertos y errores para mejorar su desempeño.

\subsection{Retroalimentación}

Después de que el estudiante envía la solución a un ejercicio, el sistema evalúa la solución y le informa al estudiante si es correcta o incorrecta. Si la solución es correcta, se le puede sugerir al estudiante que explore formas de mejorar la eficiencia del código, fomentando el desarrollo de habilidades de programación más avanzadas.

En caso de que la solución sea incorrecta, el sistema identifica el error y se lo comunica al estudiante. Además, el sistema registra la cantidad de intentos que el estudiante ha realizado para resolver el ejercicio, lo que permite monitorear su progreso y adaptar futuras recomendaciones de ejercicios según su desempeño. Esta funcionalidad tiene como objetivo ayudar a identificar áreas de mejora y motivar al estudiante a persistir en la resolución de problemas.

Si el estudiante se encuentra con dificultades durante la resolución de un ejercicio, se le ofrecen pistas para guiarlo. A continuación se explica cómo funciona este proceso.

\subsection{Solicitud de pistas}

Durante la resolución de ejercicios, es posible que el estudiante enfrente dificultades y se sienta estancado. Para apoyarlo en estos momentos, el sistema ofrece la opción de solicitar pistas. Las pistas están organizadas comenzando con sugerencias generales y avanzando hacia indicaciones más específicas. 

Este sistema permite al estudiante recibir ayuda gradualmente, sin revelar la solución completa, con la intención de que lo motive a descubrirla por sí mismo mientras refuerza su comprensión del tema. Por ejemplo, si el ejercicio consiste en implementar un algoritmo de ordenamiento, la primera pista podría ser una descripción general del algoritmo, mientras que las pistas posteriores podrían detallar cómo implementar cada paso o cómo manejar casos específicos.

Este sistema de pistas escalonadas busca evitar que el estudiante se sienta abrumado, a la vez que le permite avanzar a su propio ritmo. Además, al descubrir la solución por sí mismo, el estudiante puede ganar confianza en sus habilidades y desarrollar una mayor capacidad para resolver problemas de manera independiente.

En caso de que el estudiante no logre resolver el ejercicio incluso con las pistas, se ofrece la opción de consultar la solución completa. En la siguiente sección se explica esta funcionalidad.

\subsection{Solicitud de soluciones}

Si el estudiante, a pesar de utilizar las pistas escalonadas, no logra resolver el ejercicio, el sistema ofrece la opción de solicitar la solución completa. Esta funcionalidad está pensada como último recurso, para evitar que el estudiante se frustre o pierda demasiado tiempo en un solo problema.

Al solicitar la solución, no solo se muestra el código o la respuesta final, sino que también se incluye una explicación detallada de cada paso. Esto permite al estudiante comprender el razonamiento detrás de la solución y aprender de sus errores. Esta solución no está generada por un LLM, sino que está escrita por profesores, basándose en su experiencia y en las particularidades de las clases de la facultad.

Además, incluso si el estudiante logra resolver el ejercicio por sí mismo, se le ofrece la opción de consultar la solución para que pueda comparar su enfoque con el propuesto. Consultar una solución alternativa puede aportar nuevas perspectivas, mostrar métodos más eficientes o simplemente reforzar los conceptos aprendidos. Por ejemplo, un estudiante que resolvió un problema usando iteración podría descubrir una solución recursiva que no había considerado, lo que amplía su comprensión del tema.

Además de apoyar a los estudiantes, el sistema también ofrece funcionalidades específicas para los profesores. En la siguiente sección, se detallan las herramientas disponibles para los docentes.

\section{Flujo para los profesores}\label{sec:teachers}

El sistema está diseñado para que los profesores puedan gestionar y organizar el contenido del curso, adaptándolo a las necesidades específicas de los estudiantes. A continuación, se detallan las funcionalidades disponibles para los docentes.

\subsection{Creación de contenido}

El profesor tiene la capacidad de crear y editar los temas que conforman el curso. Además, el profesor puede añadir ejercicios prácticos diseñados para reforzar los conceptos teóricos. Cada ejercicio cuenta con una solución modelo para que los estudiantes verifiquen su trabajo, y un conjunto de pistas escalonadas que orientan al estudiante.

Además, el profesor puede modificar la bibliografía referenciada por el módulo de resolución de dudas. No se limita al libro de texto de la asignatura, sino que puede añadir otras fuentes confiables, como libros o notas de clase. Esto asegura que los estudiantes tengan acceso a información actualizada y de calidad para resolver sus dudas.

Esta flexibilidad en la gestión de contenido permite al profesor adaptar el material al ritmo del curso y a las necesidades individuales de los estudiantes, asegurando que los recursos estén alineados con los objetivos del curso.

\subsection{Monitoreo y análisis del progreso}

El sistema recopila datos sobre el desempeño de los estudiantes, lo que permite a los profesores generar reportes. Estos reportes incluyen información sobre la actividad de los estudiantes, como el número de ejercicios completados, la cantidad de pistas solicitadas y las veces que se ha accedido a la solución. También se proporcionan estadísticas de desempeño, como el porcentaje de ejercicios resueltos correctamente y el tiempo promedio dedicado a cada tema.

Además, el sistema permite analizar el desempeño grupal, destacando áreas fuertes y débiles. Esto facilita la identificación de patrones comunes que puedan requerir refuerzo en clase. 

Estos reportes no solo permiten a los profesores evaluar el progreso individual y grupal, sino también tomar decisiones informadas para ajustar el contenido o la metodología de enseñanza, optimizando así el proceso de aprendizaje. Además, ayudan a identificar a estudiantes que necesiten apoyo adicional para ofrecerles recursos complementarios, promoviendo un aprendizaje más personalizado.

En este capítulo se ha presentado una propuesta diseñada para apoyar el estudio independiente en el curso de Programación de la carrera de Ciencia de la Computación en \mbox{MATCOM}.

El flujo propuesto está diseñado para guiar al estudiante a través de su proceso de aprendizaje. Cada etapa está planificada para fomentar la autonomía, el pensamiento crítico y el progreso continuo. Esta propuesta tiene el potencial de transformar la manera en que los estudiantes y profesores abordan el estudio de la programación, ofreciendo un apoyo personalizado y accesible en cualquier momento.

En el siguiente capítulo se detalla la implementación de este sistema mediante un bot de Telegram.