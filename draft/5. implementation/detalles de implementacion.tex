\documentclass{article}
\usepackage{graphicx}
\usepackage[spanish]{babel}
\usepackage{amsmath}
\usepackage{url}
\usepackage{tikz}
\usetikzlibrary{arrows.meta, positioning}
\usetikzlibrary{3d, shapes.geometric}


\input{word-comments.tex}

\title{Detalles de implementación}
\author{Anabel Benítez González}
\date{}

\begin{document}

% \tableofcontents
\maketitle

En este capítulo se describe una herramienta diseñada para apoyar el estudio independiente de  la asignatura Programación en \mbox{MATCOM}. La propuesta consiste en un bot de Telegram que funciona como asistente educativo interactivo, proporcionando respuestas a dudas sobre el contenido y ejercicios ajustados a las necesidades de los estudiantes. Este capítulo describe los componentes de la arquitectura del bot y sus funcionalidades principales.

\section{Selección de la plataforma: ¿Por qué Telegram?}

Telegram fue seleccionado como la plataforma para el desarrollo de la aplicación debido a varias ventajas. Su API es gratuita y está documentada, lo que facilita la implementación de funcionalidades y el desarrollo de características específicas. Asimismo, Telegram es accesible desde múltiples dispositivos, como teléfonos móviles y laptops, e incluso permite sesiones simultáneas en ambos, garantizando así una experiencia fluida, disponibilidad constante y comodidad para los estudiantes.

Otra razón importante para esta elección es que los estudiantes de la facultad ya están familiarizados con Telegram desde el primer año de la carrera, ya que la aplicación se utiliza ampliamente como herramienta de comunicación en \mbox{MATCOM}. Esto elimina la necesidad de aprender a manejar una nueva herramienta o de incorporar otra aplicación a su rutina, evitando distracciones adicionales. El bot propuesto simplemente se integra como un nuevo chat en una plataforma que los estudiantes ya utilizan con regularidad.

\section{Resolución de dudas}

El diagrama de la Figura~\ref{fig:q-a-diagram} ilustra cómo se utiliza el sistema basado en Generación Aumentada por Recuperación (RAG) para resolver las dudas de los estudiantes.

Como se explica en el Capítulo~\ref{chapter:rag} el proceso de construcción del recuperador consta de tres etapas: dividir el corpus en fragmentos, codificar los fragmentos y construir la base de datos vectorial.

El corpus del sistema está compuesto por los capítulos del libro ``Empezar a programar: Un enfoque multiparadigma con C\#'' en formato PDF. Este contenido se convierte a texto utilizando herramientas de extracción que preservan su estructura original. Para manejar el contenido, se divide en fragmentos de longitud fija de 2000 caracteres con un solapamiento de 200 caracteres, lo que garantiza que no se pierda el contexto entre fragmentos consecutivos. Cada fragmento se transforma en una representación numérica mediante el modelo de embeddings \textit{models/text-embedding-004} de Google.

Durante este proceso, a cada fragmento se le asocian metadatos, como el nombre del documento y el número de página correspondiente. Estos metadatos se almacenan junto con los vectores en la base de datos vectorial, lo que permite, durante la fase de generación, incorporar referencias en las respuestas proporcionadas al usuario.

Cuando un estudiante formula una pregunta, esta se procesa mediante el mismo modelo \textit{models/text-embedding-004}, que convierte la pregunta en una representación vectorial. Para medir la relevancia entre una consulta y los fragmentos almacenados en la base de datos vectorial, se emplea la similitud coseno. El sistema utiliza los 5 fragmentos más similares para generar el \textit{prompt} extendido, que combina la pregunta con el contenido relevante del libro.

El \textit{prompt} extendido se envía al LLM \textit{gemini-1.5-flash}, que genera una respuesta detallada. Además, las respuestas están enriquecidas con referencias específicas al libro, permitiendo al estudiante verificar la información en una fuente confiable.

Este flujo de trabajo muestra cómo el sistema RAG integra la búsqueda de información con la generación de texto, reduciendo el tiempo que el estudiante dedicaría a buscar respuestas por otros medios, como el libro o la consulta directa al profesor. Al incluir referencias al libro de Katrib, el sistema permite al estudiante verificar la información directamente en una fuente confiable, evitando la dependencia exclusiva LLMs o fuentes en línea que podrían no ser fiables, especialmente para estudiantes inexpertos. Este enfoque busca no solo fomentar la autonomía del estudiante, sino también reforzar su capacidad para validar la información de manera independiente.

\begin{figure}[h!]
    \centering
    \begin{tikzpicture}[
      font=\sffamily,
      every node/.style={align=center},
      box/.style={draw, rectangle, minimum width=3cm, minimum height=1cm},
      arrow/.style={-Latex},
      db/.style={cylinder, cylinder uses custom fill, cylinder body fill=blue!20, shape border rotate=90, aspect=0.25, minimum height=1.5cm, minimum width=1cm, draw}
    ]
    
    \node[box] (materials) {Libro de Katrib};
    \node[box, below=1cm of materials] (question) {Pregunta};
    \node[box, right=0.8cm of question] (embedding) {Función \\de \\Embedding};
    \node[db, right=2cm of embedding] (database) {Base de Datos \\Vectorial};
    \node[box, below=2.5cm of database] (prompt) {Prompt Extendido = \\Pregunta +\\Contenido del libro};
    \node[box, below=1.5cm of prompt] (llm) {LLM};
    % \node[box, left=2.5cm of llm] (professor) {Profesor};
    \node[box, left=2.5cm of llm] (reply) {Respuesta \\con referencias al libro};
    \node[below=1.5cm of question] (user) {
      \begin{tikzpicture}[scale=0.8] % Escala más grande
        % Cabeza
        \draw (0,0.7) circle [radius=0.3]; % Cabeza más grande y sin relleno
        % Cuerpo
        \draw (0,0.4) -- (0,-0.5); % Línea del cuerpo
        \draw (-0.4,-1) -- (0,-0.5) -- (0.4,-1); % Piernas
        \draw (-0.3,-0.2) -- (0,0.2) -- (0.3,-0.2); % Brazos
      \end{tikzpicture}
      \\ Estudiante
    };
    
    % Arrows
    \draw[arrow] (materials) -- (embedding);
    \draw[arrow] (user) -- (question);
    \draw[arrow] (question) -- (embedding);
    % \draw[arrow] (embedding) -- node[above] {Guardar} (database);
    % Two distinct arrows between nodes
    \draw[arrow, bend left=15] (embedding) to node[above] {Guardar} (database);
    \draw[arrow, bend right=15] (embedding) to node[below] {Búsqueda \\por \\similitud} (database);
    
    \draw[arrow] (database) -- node[right] {Fragmentos \\recuperados} (prompt);
    \draw[arrow] (prompt) -- (llm);
    \draw[arrow] (llm) -- (reply);
    \draw[arrow] (reply) -- ++(0,1) -| (user);
    
    \end{tikzpicture}
    \caption{Diagrama de flujo del sistema de preguntas y respuestas basado en Generación Aumentada por Recuperación (RAG)}\label{fig:q-a-diagram}
\end{figure}

% \begin{thebibliography}{99}
% \end{thebibliography}

\end{document}