\chapter{Detalles de implementación}\label{chapter:implementation}

En este capítulo se describen los detalles de implementación de un bot de Telegram diseñado para apoyar el estudio independiente de los estudiantes de programación. En la Sección~\ref{sec:technical-details} se presentan las tecnologías y herramientas utilizadas, así como las razones detrás de su selección. En la Sección~\ref{sec:features} se analiza la arquitectura del sistema y se describen las funcionalidades principales, como el registro de usuarios, el listado de temas, la resolución de dudas y la práctica con ejercicios. Finalmente, se discute cómo los profesores pueden personalizar el contenido del curso.

\section{Detalles técnicos}\label{sec:technical-details}

Se desarrolla un bot de Telegram debido a que los estudiantes de la facultad ya están familiarizados con esta herramienta desde el primer año de la carrera. En \mbox{MATCOM}, Telegram se utiliza ampliamente como medio de comunicación, donde cada asignatura cuenta con al menos un grupo de Telegram para facilitar la interacción entre estudiantes y profesores, algunas asignaturas también disponen de canales. 

Esta adopción generalizada de Telegram elimina la necesidad de que los estudiantes aprendan a manejar una nueva herramienta o incorporen otra aplicación a su rutina, evitando distracciones adicionales. El bot propuesto simplemente se integra como un nuevo chat en una plataforma que los estudiantes ya utilizan con regularidad.

Otra razón importante para esta elección son las ventajas técnicas que ofrece Telegram. Su API es gratuita y está bien documentada, lo que facilita la implementación de funcionalidades. Asimismo, Telegram es accesible desde múltiples dispositivos (dispositivos móviles, web y desktop) e incluso permite sesiones simultáneas, garantizando una disponibilidad constante y comodidad para los estudiantes.

El bot se desarrolla utilizando Python como lenguaje de programación debido a su amplia gama de bibliotecas disponibles. Para la gestión de datos, se utilizó PostgreSQL~\cite{postgresql}, un sistema gestor de bases de datos relacional escalable y de código abierto.

En la siguiente sección, se describe el diseño del sistema mediante diagramas que representan las funcionalidades principales, proporcionando una visión de cómo interactúan los componentes del bot.

\section{Arquitectura y funcionalidades del sistema}\label{sec:features}

El esquema de la base de datos está estructurado en varias tablas interrelacionadas \texttt{users}, \texttt{students}, \texttt{topics}, \texttt{exercises}, \texttt{exercise\_hints}, \texttt{student\_hints} y \texttt{student\_exercise} que gestionan la información de usuarios, estudiantes, temas, ejercicios y pistas.

En esta sección se describe el diseño del sistema mediante diagramas que representan las funcionalidades principales. La Figura~\ref{fig:legend} muestra una leyenda que explica los componentes utilizados en los diagramas. Las flechas indican el flujo de la interacción.

\begin{figure}[h!]
  \centering
      \begin{tikzpicture}[node distance=0.5cm and 1cm]
      \node (startstop) [startstop] {Estudiante ingresa\\ nombre y apellidos};
      \node (startstopLabel) [right=of startstop, font=\footnotesize] {Interacción del usuario};
  
      \node (process) [process, below=of startstop] {Mensaje de bienvenida y\\ lista de comandos disponibles};
      \node (processLabel) [right=of process, font=\footnotesize] {Respuesta del sistema};
  
      \node (decision) [decision, below=of process] {¿Quedan \\ ejercicios \\ disponibles?};
      \node (decisionLabel) [right=of decision, font=\footnotesize] {Punto de decisión};
      \end{tikzpicture}
      \caption{Leyenda de los componentes de los diagramas.}\label{fig:legend}
\end{figure}

\subsection{Registro}

Al acceder por primera vez, la herramienta solicita al estudiante ingresar su nombre completo. Luego, muestra un mensaje de bienvenida y una guía básica que explica las funcionalidades disponibles, siguiendo el flujo descrito en la Figura~\ref{fig:register}. Esta introducción tiene como objetivo que el usuario comprenda cómo aprovechar la aplicación para su estudio.

\begin{figure}[h!]
  \centering
  \begin{tikzpicture}[node distance=1cm and 1cm]
  \node (registro) [startstop] {/start};
  \node (sDatos) [process, below=of registro]{Solicitud de información personal};
  \node (datos) [startstop, below=of sDatos]{Estudiante ingresa nombre y apellidos};
  \node (bienvenida) [process, right=of datos]{Mensaje de bienvenida y\\ lista de comandos disponibles};

  \draw [arrow2] (registro) -- (sDatos);
  \draw [arrow2] (sDatos) -- (datos);
  \draw [arrow2] (datos) -- (bienvenida);
  \end{tikzpicture}
  \caption{Diagrama del flujo del registro.}\label{fig:register}
\end{figure}

Después del registro, lo más común es que el estudiante revise el listado de temas disponibles para comenzar su estudio.

\subsection{Listado de temas}

Cuando el usuario lo solicita, la aplicación muestra un listado organizado de los temas disponibles para su estudio. Los temas se presentan en el orden de estudio definido en el diseño del curso de Programación, como se describe en la Sección~\ref{sec:matcom}.

Si el nombre del tema no resulta suficientemente claro, el estudiante puede solicitar una descripción detallada del mismo. Este flujo se ilustra en la Figura~\ref{fig:topics}.

\begin{figure}[h!]
  \centering
  \begin{tikzpicture}[node distance=1cm and 1cm]
    \node (solicitud) [startstop] {/topics};
    \node (listado) [process, below=of solicitud] {Listado de temas organizados};
    \node (explorar) [decision, below=of listado] {¿El estudiante \\ desea más detalles \\ sobre un tema?};
    \node (descripcionSolicitud) [startstop, right=of explorar]{/topic [título del tema]};
    \node (descripcion) [process, below=of descripcionSolicitud] {Descripción del tema};
    
    \draw [arrow2] (solicitud) -- (listado);
    \draw [arrow2] (listado) -- (explorar);
    \draw [arrow2] (explorar) -- node[above] {Sí} (descripcionSolicitud);
    \draw [arrow2] (descripcionSolicitud) -- (descripcion);
  \end{tikzpicture}
\caption{Diagrama del flujo del listado temas.}\label{fig:topics}
\end{figure}

Durante el estudio o después de una clase, es común que los estudiantes tengan dudas sobre el contenido, como por ejemplo: ``¿qué es un array de direcciones?''. Para resolver estas inquietudes, la aplicación cuenta con un módulo de resolución de dudas.

\subsection{Sistema de resolución de dudas basado en RAG}

La aplicación incluye un sistema de resolución de dudas que permite a los estudiantes formular preguntas en lenguaje natural. Utilizando un RAG, el sistema proporciona respuestas basadas en la bibliografía del curso. La Figura~\ref{fig:q-a-diagram} muestra el funcionamiento de este sistema.

\begin{figure}[h!]
  \centering
    \begin{tikzpicture}[
      font=\sffamily,
      every node/.style={align=center},
      box/.style={draw, rectangle, minimum width=3cm, minimum height=1cm},
      arrow/.style={-Latex},
      db/.style={cylinder, cylinder uses custom fill, cylinder body fill=blue!20, shape border rotate=90, aspect=0.25, minimum height=1.5cm, minimum width=1cm, draw}
    ]
    
    \node[box] (materials) {Libro de Katrib};
    \node[box, below=1cm of materials] (question) {/ask [pregunta]};
    \node[box, right=0.8cm of question] (embedding) {Función \\de \\Embedding};
    \node[db, right=2cm of embedding] (database) {Base de Datos \\Vectorial};
    \node[box, below=2.5cm of database] (prompt) {Prompt Extendido = \\Pregunta +\\Contenido del libro};
    \node[box, below=1.5cm of prompt] (llm) {LLM};
    \node[box, left=2.5cm of llm] (reply) {Respuesta \\con referencias al libro};
    \node[below=1.5cm of question] (user) {
      \begin{tikzpicture}[scale=0.8]
        \draw (0,0.7) circle [radius=0.3]; % Cabeza
        % Cuerpo
        \draw (0,0.4) -- (0,-0.5); % Línea del cuerpo
        \draw (-0.4,-1) -- (0,-0.5) -- (0.4,-1); % Piernas
        \draw (-0.3,-0.2) -- (0,0.2) -- (0.3,-0.2); % Brazos
      \end{tikzpicture}
      \\ Estudiante
    };
    
    \draw[arrow] (materials) -- (embedding);
    \draw[arrow] (user) -- (question);
    \draw[arrow] (question) -- (embedding);
    \draw[arrow, bend left=15] (embedding) to node[above] {Guardar} (database);
    \draw[arrow, bend right=15] (embedding) to node[below] {Búsqueda \\por \\similitud} (database);
    
    \draw[arrow] (database) -- node[right] {Fragmentos \\recuperados} (prompt);
    \draw[arrow] (prompt) -- (llm);
    \draw[arrow] (llm) -- (reply);
    \draw[arrow] (reply) -- ++(0,1) -| (user);
    
    \end{tikzpicture}
    \caption{Sistema de preguntas y respuestas basado en Generación Aumentada por Recuperación (RAG).}\label{fig:q-a-diagram}
\end{figure}

Como se explica en el Capítulo~\ref{chapter:rag} el proceso de construcción del componente de recuperación consta de tres etapas: dividir el corpus en fragmentos, codificar los fragmentos y construir la base de datos vectorial.

El corpus del sistema está compuesto por los capítulos del libro ``Empezar a programar: Un enfoque multiparadigma con C\#'' en formato PDF. Este contenido se convierte a texto utilizando herramientas de extracción que preservan su estructura original. Para manejar el contenido, se divide en fragmentos de longitud fija de 2000 caracteres con un solapamiento de 200 caracteres, con el objetivo de que no se pierda el contexto entre fragmentos consecutivos. Cada fragmento se transforma en una representación numérica mediante el modelo de embeddings \textit{models/text-embedding-004} de Google.

Durante este proceso, a cada fragmento se le asocian metadatos, como el nombre del documento y el número de página correspondiente. Estos metadatos se almacenan junto con los vectores en la base de datos vectorial, lo que permite, durante la fase de generación, incorporar referencias en las respuestas proporcionadas al usuario.

Cuando un estudiante formula una pregunta, esta se procesa mediante el mismo modelo \textit{models/text-embedding-004}, que convierte la pregunta en una representación vectorial. Para medir la relevancia entre una consulta y los fragmentos almacenados en la base de datos vectorial, se emplea la similitud coseno. El sistema utiliza los 5 fragmentos más similares para generar el \textit{prompt} extendido, que combina la pregunta con el contenido relevante del libro.

El \textit{prompt} extendido se envía al LLM \textit{gemini-1.5-flash}, que genera una respuesta detallada. Además, las respuestas están enriquecidas con referencias específicas al libro, permitiendo al estudiante verificar la información en una fuente confiable.

Este flujo de trabajo muestra cómo el sistema RAG combina la búsqueda de información con la generación de texto, con el objetivo de reducir el tiempo que los estudiantes invertirían en buscar respuestas a través de medios tradicionales, como la revisión manual del libro o la consulta directa al profesor. Al incorporar referencias específicas al libro de Katrib, el sistema no solo proporciona respuestas más confiables, sino que también fomenta la capacidad crítica de los estudiantes, permitiéndoles contrastar la información generada con fuentes autorizadas. De esta manera, se busca minimizar la dependencia exclusiva de LLMs o fuentes en línea que podrían no ser del todo fiables, especialmente para usuarios con poca experiencia.

Además de resolver dudas, la aplicación ofrece una experiencia de aprendizaje activo mediante la práctica con ejercicios.

\subsection{Práctica con ejercicios sugeridos. Pistas y soluciones.}

La aplicación ofrece una experiencia de aprendizaje personalizada al sugerir ejercicios adaptados al nivel del estudiante. Durante la resolución, el estudiante puede solicitar pistas que se dan en orden. En cualquier momento, el estudiante puede acceder a la solución completa. Una vez que el estudiante completa su implementación la manda al bot. Este flujo está descrito en la Figura~\ref{fig:exercises} de la página~\pageref{fig:exercises}.

El nivel del estudiante se calcula en función de su desempeño previo en la aplicación, considerando dos factores principales: la cantidad de ejercicios resueltos y la complejidad de los ejercicios que ha completado. Por ejemplo, si un estudiante resuelve 5 ejercicios básicos sin pedir pistas ni soluciones, la aplicación comenzará a proponer problemas de nivel intermedio.

\begin{figure}[h!]
  \centering
  \begin{tikzpicture}[node distance=1cm and 1cm]

    \node (start) [startstop] {/exercise [título del tema]};
    \node (disponibles) [decision, below=of start] {¿Quedan \\ ejercicios \\ disponibles?};
    \node (sugerencia) [process, below=of disponibles] {Sugerencia de ejercicio};
    \node (completado) [process, right=of disponibles] {Felicidades, Completaste \\ todos los ejercicios \\ de este tema};
    \node (conversion) [decision, below=of sugerencia] {¿El estudiante \\ sabe resolverlo?};
    \node (submit) [startstop, right=of conversion] {/submit [código]};
    \node (pista) [startstop, below=of conversion] {/hint [número del ejercicio]};
    \node (pistasDisponibles) [decision, left=of pista] {¿Quedan \\ pistas \\ disponibles?};
    \node (noPistas) [process, below=of pistasDisponibles] {Lo siento, no quedan \\ pistas para este \\ ejercicio};
    \node (sugerenciaPista) [process, above=of pistasDisponibles] {Sugerencia de pista};
    
    \draw [arrow2] (start) -- (disponibles);
    \draw [arrow2] (disponibles) -- node[above] {No} (completado);
    \draw [arrow2] (disponibles) -- node[left] {Sí} (sugerencia);
    \draw [arrow2] (sugerencia) -- (conversion);
    \draw [arrow2] (conversion) -- node[above] {Sí} (submit);
    \draw [arrow2] (conversion) -- node[left] {No} (pista);
    \draw [arrow2] (pista) -- (pistasDisponibles);
    \draw [arrow2] (pistasDisponibles) -- node[left] {No} (noPistas);
    \draw [arrow2] (pistasDisponibles) -- node[left] {Sí} (sugerenciaPista);
    \draw [arrow2] (sugerenciaPista) --  (conversion);
    
  \end{tikzpicture}
\caption{Diagrama del flujo del sistema de ejercicios.}\label{fig:exercises}
\end{figure}

Para mejorar esta funcionalidad, se propone implementar un sistema de evaluación automática de la correctitud de la solución. Actualmente, cuando un estudiante envía su código, este se almacena en la base de datos, el ejercicio se marca como ``pendiente de revisión'' y queda a la espera de que un profesor lo revise. Si bien este enfoque es funcional, la retroalimentación que proporciona al estudiante puede demorar.

Con la nueva propuesta, el sistema de evaluación automática ejecutará el código del estudiante y comparará la salida generada con los casos de prueba predefinidos. Si la salida coincide con los resultados esperados, el ejercicio se marcará como resuelto. En caso contrario, el sistema proporcionará retroalimentación inmediata al estudiante sobre los errores encontrados. De todas formas, con este sistema de evaluación automática, también se almacenaría el código de las soluciones en la base de datos, dejando al profesor la posibilidad de revisarlos.

A continuación se describe cómo los profesores podrían cambiar el contenido del curso, ya sea los materiales de los que se recupera información para resolver dudas, como de los ejercicios.

\subsection{Personalización del curso}

Si se desea cambiar los temas y ejercicios del curso, es necesario crear un archivo JSON en la carpeta \texttt{data} con el siguiente formato:

\begin{itemize}
    \item \texttt{title} (tipo \texttt{string}): Título del tema.
    \item \texttt{description} (tipo \texttt{string}): Descripción general del tema.
    \item \texttt{exercises} (tipo \texttt{array}): Lista de ejercicios asociados al tema. Cada ejercicio contiene:
    \begin{itemize}
        \item \texttt{title} (tipo \texttt{string}): Título del ejercicio.
        \item \texttt{content} (tipo \texttt{string}): Enunciado o descripción del ejercicio.
        \item \texttt{difficulty} (tipo \texttt{string}): Dificultad del ejercicio: \textit{basic}, \textit{intermediate} o \textit{advanced}.
        \item \texttt{solution} (tipo \texttt{string}): Solución completa del ejercicio.
        \item \texttt{hints} (tipo \texttt{array}): Lista de pistas asociadas al ejercicio, donde cada pista tiene:
        \begin{itemize}
            \item \texttt{text} (tipo \texttt{string}): Texto de la pista.
            \item \texttt{level} (tipo \texttt{number}): Nivel de detalle de la pista, que define el orden en que se entregarán al usuario. Las pistas se muestran de menor a mayor nivel (0, 1, 2, \ldots), se recomienda comenzar con sugerencias generales e ir avanzando hacia indicaciones más específicas.
        \end{itemize}
    \end{itemize}
\end{itemize}

Para añadir más documentos al sistema de preguntas y respuestas, es necesario colocar los archivos en formato PDF dentro de la carpeta \texttt{data/corpus}. Cuando se ejecute, la aplicación procesará estos documentos.

En este capítulo se han descrito los detalles de implementación del bot de Telegram. Las funcionalidades implementadas, como el registro de usuarios, la resolución de dudas y la práctica con ejercicios, están diseñadas para mejorar la experiencia de aprendizaje de los estudiantes. Además, la posibilidad de personalizar el contenido del curso asegura que el sistema pueda adaptarse a las necesidades de diferentes asignaturas y profesores.